%!TEX root = forallxsol.tex
%\part{Natural deduction for TFL}
%\label{ch.NDTFL}
%\addtocontents{toc}{\protect\mbox{}\protect\hrulefill\par}

\renewenvironment{proof} %TB: make proofs smaller, since space is a premium here
	{\noindent\par\noindent\small$\begin{nd}}
	{\end{nd}$\noindent\normalsize\ignorespacesafterend}

\setcounter{chapter}{14}
\chapter{Basic rules for TFL}\setcounter{ProbPart}{0}
\problempart
The following two `proofs' are \emph{incorrect}. Explain the mistakes they make.
\begin{multicols}{2}
\begin{proof}
\hypo{abc}{(\enot L \land A) \eor L}
\open
\hypo{nla}{\enot L \eand A}
\have{nl}{\enot L}\ae{nl}
        \have{a}{A}\ae{abc}
\close
\open
        \hypo{l}{L}
        \have{red}{\ered}\ri{l, nl}
        \have{a2}{A}\re{red}
\close
\have{con}{A}\oe{abc, nla-a, l-a2}
\end{proof}
\vfill
\noindent\myanswer{$\eand$E on line 5 can't be applied to line~1, since it is not of the form $\meta{A} \eand \meta{B}$.  `$A$' could be obtained by $\eand$E, but from line~2.}

\myanswer{$\ered$I on line 5 illicitly refers to a line from a closed subproof (line~3).}
\hfill
\columnbreak
\begin{proof}
\hypo{abc}{A \eand (B \eand C)}
\hypo{bcd}{(B \eor C) \eif D}
\have{b}{B}\ae{abc}
\have{bc}{B \eor C}\oi{b}
\have{d}{D}\ce{bc, bcd}
\end{proof}
\vfill
\noindent\myanswer{$\eand$E on line 3 should yield `$B \eand C$'.  `$B$' could then be obtained by $\eand$E again.}

\myanswer{The citation for line 5 is the wrong way round: it should be `$\eif$E 2, 4'.}
\end{multicols}

\problempart
The following three proofs are missing their citations (rule and line numbers). Add them, to turn them into \textit{bona fide} proofs. Additionally, write down the argument that corresponds to each proof.
\newpage
\begin{multicols}{2}
\begin{proof}
\hypo{ps}{P \eand S}
\hypo{nsor}{S \eif R}
\have{p}{P}\ae{ps}
\have{s}{S}\ae{ps}
\have{r}{R}\ce{nsor, s}
\have{re}{R \eor E}\oi{r}
\end{proof}
\\\myanswer{Corresponding argument: \\$P \eand S, S \eif R \therefore R \eor E$}

\bigskip

\begin{proof}
\hypo{ad}{A \eif D}
\open
	\hypo{ab}{A \eand B}
	\have{a}{A}\ae{ab}
	\have{d}{D}\ce{ad, a}
	\have{de}{D \eor E}\oi{d}
\close
\have{conc}{(A \eand B) \eif (D \eor E)}\ci{ab-de}
\end{proof}
\\\myanswer{Corresponding argument: \\$A \eif D \therefore (A \eand B) \eif (D \eor E)$}

\columnbreak

\begin{proof}
\hypo{nlcjol}{\enot L \eif (J \eor L)}
\hypo{nl}{\enot L}
\have{jol}{J \eor L}\ce{nlcjol, nl}
\open
	\hypo{j}{J}
	\have{jj}{J \eand J}\ai{j, j}
	\have{j2}{J}\ae{jj}
\close
\open
	\hypo{l}{L}
	\have{red}{\ered}\ri{l, nl}
	\have{j3}{J}\re{red}
\close
\have{conc}{J}\oe{jol, j-j2, l-j3}
\end{proof}
\\\myanswer{Corresponding argument:\\
$\enot L \eif (J \eor L), \enot L \therefore J$}
\end{multicols}

\solutions
\problempart
\label{pr.solvedTFLproofs}
Give a proof for each of the following arguments:
\begin{earg}
\item $J\eif\enot J \therefore \enot J$
\myanswer{\begin{proof}
\hypo{jnj}{J \eif \enot J}
\open
	\hypo{j}{J}
	\have{nj}{\enot J}\ce{jnj, j}
	\have{red}{\ered}\ri{j, nj}
\close
\have{con}{\enot J}\ni{j-red}
\end{proof}}

\item $Q\eif(Q\eand\enot Q) \therefore \enot Q$
\myanswer{\begin{proof}
\hypo{qq}{Q \eif  (Q \eand \enot Q)}
\open
	\hypo{q}{Q}
	\have{con}{Q \eand \enot Q}\ce{qq, q}
	\have{nq}{\enot Q}\ae{con}
	\have{red}{\ered}\ri{q, nq}
\close
\have{con}{\enot Q}\ni{q-red}
\end{proof}}

\item $A\eif (B\eif C) \therefore (A\eand B)\eif C$
\myanswer{\begin{proof}
\hypo{abc}{A \eif (B \eif C)}
\open
	\hypo{ab}{A \eand B}
	\have{a}{A}\ae{ab}
	\have{bc}{B \eif C}\ce{abc, a}
	\have{b}{B}\ae{ab}
	\have{c}{C}\ce{bc, b}	
\close
\have{con}{(A \eand B) \eif C}\ci{ab-c}
\end{proof}}

\item $K\eand L \therefore K\eiff L$
\myanswer{\begin{proof}
\hypo{kl}{K \eand L}
\open
	\hypo{k1}{K}
	\have{l1}{L}\ae{kl}
\close
\open
	\hypo{l2}{L}
	\have{k2}{K}\ae{kl}
\close
\have{con}{K \eiff L}\bi{k1-l1, l2-k2}
\end{proof}}

\item $(C\eand D)\eor E \therefore E\eor D$
\myanswer{\begin{proof}
\hypo{cde}{(C \eand D) \eor E}
\open
	\hypo{cd}{C \eand D}
	\have{d}{D}\ae{cd}
	\have{ed}{E \eor D}\oi{d}
\close
\open
	\hypo{e}{E}
	\have{ed1}{E \eor D}\oi{e}
\close
\have{con}{E \eor D}\oe{cde, cd-ed, e-ed1}
\end{proof}}

\item $A\eiff B, B\eiff C \therefore A\eiff C$
\myanswer{\begin{proof}
\hypo{ab}{A \eiff B}
\hypo{bc}{B \eiff C}
\open
	\hypo{a}{A}
	\have{b}{B}\be{ab, a}
	\have{c}{C}\be{bc, b}
\close
\open
	\hypo{c1}{C}
	\have{b1}{B}\be{bc,c1}
	\have{a1}{A}\be{ab, b1}
\close
\have{con}{A \eiff C}\bi{a-c, c1-a1}
\end{proof}}

\item $\enot F\eif G, F\eif H \therefore G\eor H$
\myanswer{\begin{proof}
\hypo{nfg}{\enot F \eif G}
\hypo{fh}{F \eif H}
\open
	\hypo{f}{F}
	\have{h}{H}\ce{fh, f}
	\have{gh}{G \eor H}\oi{h}
\close
\open
	\hypo{nf}{\enot F}
	\have{g}{G}\ce{nfg, nf}
	\have{gh1}{G \eor H}\oi{g}
\close
\have{con}{G \eor H}\tnd{f-gh, nf-gh1}
\end{proof}}

\item $(Z\eand K) \eor (K\eand M), K \eif D \therefore D$
\myanswer{\begin{proof}
\hypo{zkkm}{(Z \eand K) \eor (K \eand M)}
\hypo{kd}{K \eif D}
\open
	\hypo{zk}{Z \eand K}
	\have{k}{K}\ae{zk}
\close
\open
	\hypo{km}{K \eand M}
	\have{k1}{K}\ae{km}
\close
\have{k2}{K}\oe{zkkm, zk-k, km-k1}
\have{d}{D}\ce{kd, k2}
\end{proof}}

\item $P \eand (Q\eor R), P\eif \enot R \therefore Q\eor E$
\myanswer{\begin{proof}
\hypo{pqr}{P \eand (Q \eor R)}
\hypo{pnr}{P \eif \enot R}
\have{p}{P}\ae{pqr}
\have{nr}{\enot R}\ce{pnr, p}
\have{qr}{Q \eor R}\ae{pqr}
\open
	\hypo{q}{Q}
	\have{qe}{Q \eor E}\oi{q}
\close
\open
	\hypo{r}{R}
	\have{red}{\ered}\ri{r, nr}
	\have{qe1}{Q \eor E}\re{red}
\close
\have{con}{Q \eor E}\oe{qr, q-qe, r-qe1}
\end{proof}}

\newpage
\item $S\eiff T \therefore S\eiff (T\eor S)$
\myanswer{\begin{proof}
\hypo{st}{S \eiff T}
\open
	\hypo{s}{S}
	\have{t}{T}\be{st, s}
	\have{ts}{T \eor S}\oi{t}
\close
\open
	\hypo{ts1}{T \eor S}
	\open
		\hypo{t1}{T}
		\have{s3}{S}\be{st, t1}
	\close
	\open
		\hypo{s1}{S}
		\have{ss}{S \eand S}\ai{s1, s1}
		\have{s2}{S}\ae{ss}
	\close
	\have{s4}{S}\oe{ts1, t1-s3, s1-s2}
\close
\have{con}{S \eiff (T \eor S)}\bi{s-ts, ts1-s4}
\end{proof}}

\item $\enot (P \eif Q) \therefore \enot Q$
\myanswer{\begin{proof}
\hypo{npq}{\enot (P \eif Q)}
\open
	\hypo{q}{Q}
	\open
		\hypo{p}{P}
		\have{qq}{Q \eand Q}\ai{q, q}
		\have{q2}{Q}\ae{qq}
	\close
	\have{pq}{P \eif Q}\ci{p-q2}
	\have{red}{\ered}\ri{pq, npq}
\close
\have{con}{\enot Q}\ni{q-red}
\end{proof}}

\item $\enot (P \eif Q) \therefore P$
\myanswer{\begin{proof}
\hypo{npq}{\enot (P \eif Q)}
\open
	\hypo{np}{\enot P}
	\open
		\hypo{p2}{P}
		\have{red}{\ered}\ri{p2, np}
		\have{q}{Q}\re{red}
	\close
	\have{pq}{P \eif Q}\ci{p2-q}
	\have{red1}{\ered}\ri{pq, npq}
\close
\have{con}{P}\ip{np-red1}
\end{proof}}
\end{earg}


\chapter{Additional rules for TFL}
\label{s:Further}
\setcounter{ProbPart}{0}
\problempart
\label{pr.justifyTFLproof}
The following proofs are missing their citations (rule and line numbers). Add them wherever they are required:
\begin{multicols}{2}
\begin{proof}
\hypo{1}{W \eif \enot B}
\hypo{2}{A \eand W}
\hypo{2b}{B \eor (J \eand K)}
\have{3}{W}\ae{2}
\have{4}{\enot B}\ce{1, 3}
\have{5}{J \eand K}\ds{2b, 4}
\have{6}{K}\ae{5}
\end{proof}
\vfill
\begin{proof}
\hypo{1}{L \eiff \enot O}
\hypo{2}{L \eor \enot O}
\open
	\hypo{a1}{\enot L}
	\have{a2}{\enot O}\ds{2, a1}
	\have{a3}{L}\be{1, a2}
	\have{a4}{\ered}\ri{a3, a1}
\close
\have{3b}{\enot\enot L}\ni{a1-a4}
\have{3}{L}\dne{3b}
\end{proof}
\columnbreak
\begin{proof}
\hypo{1}{Z \eif (C \eand \enot N)}
\hypo{2}{\enot Z \eif (N \eand \enot C)}
\open
	\hypo{a1}{\enot(N \eor  C)}
	\have{a2}{\enot N \eand \enot C}\dem{a1}
	\have{a6}{\enot N}\ae{a2}
	\have{b4}{\enot C}\ae{a2}
		\open
		\hypo{b1}{Z}
		\have{b2}{C \eand \enot N}\ce{1, b1}
		\have{b3}{C}\ae{b2}
		\have{red}{\ered}\ri{b3, b4}
	\close
	\have{a3}{\enot Z}\ni{b1-red}
	\have{a4}{N \eand \enot C}\ce{2,a3}
	\have{a5}{N}\ae{a4}
	\have{a7}{\ered}\ri{a5, a6}
\close
\have{3b}{\enot\enot(N \eor C)}\ni{a1-a7}
\have{3}{N \eor C}\dne{3b}
\end{proof}
\end{multicols}

\problempart 
Give a proof for each of these arguments:
\begin{earg}
\item $E\eor F$, $F\eor G$, $\enot F \therefore E \eand G$
\myanswer{\begin{proof}
\hypo{ef}{E \eor F}
\hypo{fg}{F \eor G}
\hypo{nf}{\enot F}
\have{e}{E}\ds{ef, nf}
\have{g}{G}\ds{fg, nf}
\have{eg}{E \eand G}\ai{e, g}
\end{proof}}

\item $M\eor(N\eif M) \therefore \enot M \eif \enot N$
\myanswer{\begin{proof}
\hypo{mnm}{M \eor (N \eif M)}
\open
	\hypo{nm}{\enot M}
	\have{nim}{N \eif M}\ds{mnm, nm}
	\have{nn}{\enot N}\mt{nim, nm}
\close
\have{nminn}{\enot M \eif \enot N}\ci{nm-nn}
\end{proof}}

\item $(M \eor N) \eand (O \eor P)$, $N \eif P$, $\enot P \therefore M\eand O$
\myanswer{\begin{proof}
\hypo{mnop}{(M \eor N) \eand (O \eor P)}
\hypo{nip}{N \eif P}
\hypo{np}{\enot P}
\have{nn}{\enot N}\mt{nip, np}
\have{mn}{M \eor N}\ae{mnop}
\have{m}{M}\ds{mn, nn}
\have{op}{O \eor P}\ae{mnop}
\have{o}{O}\ds{op, np}
\have{mo}{M \eand O}\ai{m, o}
\end{proof}}

\item $(X\eand Y)\eor(X\eand Z)$, $\enot(X\eand D)$, $D\eor M$ \therefore $M$
\myanswer{\begin{proof}
\hypo{xyxz}{(X\eand Y)\eor(X\eand Z)}
\hypo{nxd}{\enot (X \eand D)}
\hypo{dm}{D \eor M}
\open
	\hypo{xy}{X \eand Y}
	\have{x}{X}\ae{xy}
\close
\open
	\hypo{xz}{X \eand Z}
	\have{x2}{X}\ae{xz}
\close
\have{x3}{X}\oe{xyxz, xy-x, xz-x2}
\open
	\hypo{d}{D}
	\have{xd}{X \eand D}\ai{x3, d}
	\have{red}{\ered}\ri{xd, nxd}
\close
\have{nd}{\enot D}\ni{d-red}
\have{m}{M}\ds{dm, nd}
\end{proof}}

\end{earg}

\chapter{Proof-theoretic concepts}
\label{s:ProofTheoreticConcepts}

\setcounter{ProbPart}{0}
\problempart
Show that each of the following sentences is a theorem:
\begin{earg}
\item $O \eif O$
\myanswer{\begin{proof}
\open
	\hypo{o}{O}
	\have{o2}{O}\by{R}{o}
\close
\have{con}{O \eif O}\ci{o-o2}
\end{proof}}
\item $N \eor \enot N$
\myanswer{\begin{proof}
\open
	\hypo{n}{N}
	\have{nnn}{N \eor \enot N}\oi{n}
\close
\open
	\hypo{nn}{\enot N}
	\have{nnn1}{N \eor \enot N}\oi{nn}
\close
\have{con}{N \eor \enot N}\tnd{n-nnn, nn-nnn1}
\end{proof}}

\newpage
\item $J \eiff [J\eor (L\eand\enot L)]$
\myanswer{\begin{proof}
\open
	\hypo{j1}{J}
	\have{jlnl1}{J \eor (L \eand \enot L)}\oi{j1}
\close
\open
	\hypo{jlnl2}{J \eor (L \eand \enot L)}
%	\open
%		\hypo{j2}{J}
%		\have{j3}{J}\by{R}{j2}
%	\close
	\open
		\hypo{lnl}{L \eand \enot L}
		\have{l}{L}\ae{lnl}
		\have{nl}{\enot L}\ae{lnl}
		\have{red}{\ered}\ri{l,nl}
%		\have{j4}{J}\re{red}
	\close
	\have{nlnl}{\enot (L \eand \enot L)}\ni{lnl-red}
	\have{j5}{J}\ds{jlnl2, nlnl}
\close
\have{con}{J \eiff [J\eor (L\eand\enot L)]}\bi{j1-jlnl1, jlnl2-j5}
\end{proof}}
\item $((A \eif B) \eif A) \eif A$ 
\myanswer{\begin{proof}
\open
	\hypo{aba}{(A \eif B) \eif A}
	\open 
		\hypo{na}{\enot A}
		\have{nab}{\enot (A \eif B)}\mt{aba, na}
		\open
			\hypo{a}{A}
			\have{red}{\ered}\ri{a, na}
			\have{b}{B}\re{red}
		\close
		\have{ab}{A \eif B}\ci{a-b}
		\have{red1}{\ered}\ri{ab, nab}
	\close
	\have{a2}{A}\ip{aba-red1}
\close
\have{con}{((A \eif B) \eif A) \eif A}\ci{aba-a2}
\end{proof}}
\end{earg}


\problempart
Provide proofs to show each of the following:
\begin{earg}
\item $C\eif(E\eand G), \enot C \eif G \proves G$
\myanswer{\begin{proof}
\hypo{ceg}{C \eif (E \eand G)}
\hypo{ncG}{\enot C \eif G}
\open
	\hypo{c}{C}
	\have{eg}{E \eand G}\ce{ceg, c}
	\have{g}{G}\ae{eg}
\close
\open
	\hypo{nc}{\enot C}
	\have{g1}{G}\ce{ncG, nc}
\close
\have{g2}{G}\tnd{c-g, nc-g1}
\end{proof}}

\item $M \eand (\enot N \eif \enot M) \proves (N \eand M) \eor \enot M$
\myanswer{\begin{proof}
\hypo{MnNnM}{M \eand (\enot N \eif \enot M)}
\have{M}{M}\ae{MnNnM}
\have{nNnM}{\enot N \eif \enot M}\ae{MnNnM}
\open
	\hypo{nN}{\enot N}
	\have{nM}{\enot M}\ce{nNnM, nN}
	\have{red}{\ered}\ri{M, nM}
\close
\have{N}{N}\ip{nN-red}
\have{NM}{N \eand M}\ai{N, M}
\have{con}{(N \eand M) \eor \enot M}\oi{NM}	
\end{proof}}


\item $(Z\eand K)\eiff(Y\eand M), D\eand(D\eif M) \proves Y\eif Z$
\myanswer{\begin{proof}
\hypo{zkym}{(Z \eand K) \eiff (Y \eand M)}
\hypo{ddm}{D \eand (D \eif M)}
\have{d}{D}\ae{ddm}
\have{dm}{D \eif M}\ae{ddm}
\have{m}{M}\ce{dm, d}
\open
	\hypo{y}{Y}
	\have{ym}{Y \eand M}\ai{y, m}
	\have{zk}{Z \eand K}\be{zkym, ym}
	\have{z}{Z}\ae{zk}
\close
\have{yz}{Y \eif Z}\ci{y-z}
\end{proof}}


\item $(W \eor X) \eor (Y \eor Z), X\eif Y, \enot Z \proves W\eor Y$
\myanswer{\begin{proof}
\hypo{wxyz}{(W \eor X) \eor (Y \eor Z)}
\hypo{xy}{X \eif Y}
\hypo{nz}{\enot Z}
\open
	\hypo{wx}{W \eor X}
	\open
		\hypo{w}{W}
		\have{wy1}{W \eor Y}\oi{w}
	\close
	\open
		\hypo{x}{X}
		\have{y1}{Y}\ce{xy, x}
		\have{wy2}{W \eor Y}\oi{y1}
	\close
	\have{wy3}{W \eor Y}\oe{wx, w-wy1, x-wy2}
\close
\open
	\hypo{yz}{Y \eor Z}
	\have{y}{Y}\ds{yz, nz}
	\have{wy}{W \eor Y}\oi{y}
\close
\have{wy4}{W \eor Y}\oe{wxyz, wx-wy3, yz-wy}
\end{proof}}
\end{earg}



\problempart
Show that each of the following pairs of sentences are provably equivalent: %TB: a bit of a hack here, to try to get the multicols thing working nicer.

\

1. $R \eiff E$, $E \eiff R$
\myanswer{\ignorespaces\begin{multicols}{2}
\begin{proof}
\hypo{re}{R \eiff E}
\open
	\hypo{e}{E}
	\have{r}{R}\be{re, e}
\close
\open
	\hypo{r1}{R}
	\have{e1}{E}\be{re, r1}
\close
\have{con}{E \eiff R}\bi{e-r, r1-e1}
\end{proof}\begin{proof}
\hypo{re}{E \eiff R}
\open
	\hypo{e}{E}
	\have{r}{R}\be{re, e}
\close
\open
	\hypo{r1}{R}
	\have{e1}{E}\be{re, r1}
\close
\have{con}{R \eiff E}\bi{r1-e1, e-r}
\end{proof}
\end{multicols}}
2. $G$, $\enot\enot\enot\enot G$
\myanswer{\begin{multicols}{2}
\begin{proof}
\hypo{g}{G}
\open
	\hypo{nnng}{\enot \enot \enot G}
	\have{ng}{\enot G}\dne{nnng}
	\have{red}{\ered}\ri{g,ng}
\close
\have{nnnng}{\enot\enot\enot\enot G}\ni{nnng-red}
\end{proof}\begin{proof}
\hypo{nnnng}{\enot\enot\enot\enot G}
\have{nng}{\enot \enot G}\dne{nnnng}
\have{g}{G}\dne{nng}
\end{proof}
\end{multicols}}
3. $T\eif S$, $\enot S \eif \enot T$
\myanswer{\begin{multicols}{2}
\begin{proof}
\hypo{ts}{T \eif S}
\open
	\hypo{ns}{\enot S}
	\have{nt}{\enot T}\mt{ts, ns}
\close
\have{nsnt}{\enot S \eif \enot T}\ci{ns-nt}
\end{proof}\begin{proof}
\hypo{nsnt}{\enot S \eif \enot T}
\open
	\hypo{t}{T}
	\open
		\hypo{ns}{\enot S}
		\have{nt}{\enot T}\ce{nsnt, ns}
		\have{red}{\ered}\ri{t, nt}
	\close
	\have{nns}{\enot \enot S}\ni{ns-red}
	\have{s}{S}\dne{nns}
\close
\have{ts}{T \eif S}\ci{t-s}
\end{proof}\end{multicols}}
4. $U \eif I$, $\enot(U \eand \enot I)$
\myanswer{
\begin{multicols}{2}
\begin{proof}
\hypo{ui}{U \eif I}
\open
	\hypo{uni}{U \eand \enot I}
	\have{u}{U}\ae{uni}
	\have{ni}{\enot I}\ae{uni}
	\have{i}{I}\ce{ui, u}
	\have{red}{\ered}\ri{i, ni}
\close
\have{con}{\enot (U \eand \enot I)}\ni{uni-red}
\end{proof}
\begin{proof}
\hypo{con}{\enot (U \eand \enot I)}
\open
	\hypo{u}{U}
	\open
		\hypo{ni}{\enot I}
		\have{uni}{U \eand \enot I}\ai{u, ni}
		\have{red}{\ered}\ri{uni, con}
	\close
	\have{nni}{\enot\enot I}\ni{ni-red}
	\have{i}{I}\dne{nni}
\close
\have{ui}{U \eif I}\ci{u-i}
\end{proof}
\end{multicols}}
\newpage 5. $\enot (C \eif D), C \eand \enot D$
\myanswer{\begin{multicols}{2}
\begin{proof}
\hypo{cnd}{C \eand \enot D}
\have{c}{C}\ae{cnd}
\have{nd}{\enot D}\ae{cnd}
\open
	\hypo{cd}{C \eif D}
	\have{d}{D}\ce{cd, c}
	\have{red}{\ered}\ri{d, nd}
\close
\have{ncd}{\enot (C \eif D)}\ni{cd-red}
\end{proof}
\begin{proof}
\hypo{ncd}{\enot (C \eif D)}
\open
	\hypo{d}{D}
	\open
		\hypo{c}{C}
		\have{d1}{D}\by{R}{d}
	\close
	\have{cd}{C \eif D}\ci{c-d1}
	\have{red}{\ered}\ri{cd, ncd}
\close
\have{nd}{\enot D}\ni{d-red}
\open
	\hypo{nc}{\enot C}
	\open
		\hypo{c1}{C}
		\have{red1}{\ered}\ri{c1, nc}
		\have{d2}{D}\re{red1}
	\close
	\have{cd1}{C \eif D}\ci{c1-d2}
	\have{red2}{\ered}\ri{cd1, ncd}
\close
\have{nnc}{\enot \enot C}\ni{nc-red2}
\have{c2}{C}\dne{nnc}
\have{cnd}{C \eand \enot D}\ai{c2, nd}
\end{proof}
\end{multicols}}
6. $\enot G \eiff H$, $\enot(G \eiff H)$ 
\myanswer{\begin{multicols}{2}
\begin{proof}
\hypo{ngh}{\enot G \eiff H}
\open
	\hypo{gh}{G \eiff H}
	\open
		\hypo{g}{G}
		\have{h}{H}\be{gh, g}
		\have{ng}{\enot G}\be{ngh, h}
		\have{red}{\ered}\ri{g, ng}
	\close
	\open
		\hypo{ng1}{\enot G}
		\have{h1}{H}\be{ngh, ng1}
		\have{g1}{G}\be{gh, h1}
		\have{red1}{\ered}\ri{g1, ng1}
	\close
	\have{red2}{\ered}\tnd{g-red, ng1-red1}
\close
\have{con}{\enot (G \eiff H)}\ni{gh-red2}
\end{proof}
\begin{proof}
\hypo{ngh}{\enot (G \eiff H)}
\open
	\hypo{ng}{\enot G}
	\open
		\hypo{nh}{\enot H}
		\open
			\hypo{g}{G}
			\have{red}{\ered}\ri{g, ng}
			\have{h}{H}\re{red}
		\close
		\open
			\hypo{h1}{H}
			\have{red1}{\ered}\ri{h1, nh}
			\have{g1}{G}\re{red1}
		\close
		\have{gh}{G \eiff H}\bi{g-h, h1-g1}
		\have{red2}{\ered}\ri{gh, ngh}
	\close
	\have{h2}{H}\ip{nh-red2}
\close
\open
	\hypo{h3}{H}
	\open
		\hypo{g2}{G}
		\open
			\hypo{g4}{G}
			\have{h5}{H}\by{R}{h3}
		\close
		\open
			\hypo{h4}{H}
			\have{g3}{G}\by{R}{g2}
		\close
		\have{gh1}{G \eiff H}\bi{g4-h5, h4-g3}
		\have{red3}{\ered}\ri{gh1, ngh}
	\close
	\have{ng1}{\enot G}\ni{g2-red3}
\close
\have{con}{\enot G \eiff H}\bi{ng-h2, h3-ng1}
\end{proof}
\end{multicols}}

\problempart
If you know that $\meta{A}\proves\meta{B}$, what can you say about $(\meta{A}\eand\meta{C})\proves\meta{B}$? What about $(\meta{A}\eor\meta{C})\proves\meta{B}$? Explain your answers.
\\\myanswer{If $\meta{A} \proves \meta{B}$, then $(\meta{A} \eand \meta{C})\proves \meta{B}$. After all, if $\meta{A} \proves \meta{B}$, then there is some proof with assumption $\meta{A}$ that ends with $\meta{B}$, and no undischarged assumptions other than $\meta{A}$. Now, if we start a proof with assumption $(\meta{A} \eand \meta{C})$, we can obtain $\meta{A}$ by $\eand$E. We can now copy and paste the original proof of $\meta{B}$ from $\meta{A}$, adding 1 to every line number and line number citation. The result will be a proof of $\meta{B}$ from assumption $\meta{A}$.}

\myanswer{However, we cannot prove much from $(\meta{A} \eor \meta{C})$. After all, it might be impossible to prove $\meta{B}$ from $\meta{C}$.}

\

\problempart In this chapter, we claimed that it is just as hard to show that two sentences are not provably equivalent, as it is to show that a sentence is not a theorem. Why did we claim this? (\emph{Hint}: think of a sentence that would be a theorem iff \meta{A} and \meta{B} were provably equivalent.)
\\\myanswer{Consider the sentence $\meta{A} \eiff \meta{B}$. Suppose we can show that this is a theorem. So we can prove it, with no assumptions, in $m$ lines, say. Then if we assume $\meta{A}$ and copy and paste the proof of $\meta{A} \eiff \meta{B}$ (changing the line numbering), we will have a deduction of this shape:
\begin{proof}
\hypo{a}{\meta{A}}
\have[m][1]{m}{\meta{A} \eiff \meta{B}}
\have{b}{\meta{B}}\be{m, a}
\end{proof}
\\This will show that $\meta{A} \proves \meta{B}$. In exactly the same way, we can show that $\meta{B} \proves \meta{A}$. So if we can show that $\meta{A} \eiff \meta{B}$ is a theorem, we can show that $\meta{A}$ and $\meta{B}$ are provably equivalent.}

\myanswer{Conversely, suppose we can show that $\meta{A}$ and $\meta{B}$ are provably equivalent. Then we can prove $\meta{B}$ from the assumption of $\meta{A}$ in $m$ lines, say, and prove $\meta{A}$ from the assumption of $\meta{B}$ in $n$ lines, say. Copying and pasting these proofs together (changing the line numbering where appropriate), we obtain:
\begin{proof}
\open
	\hypo{a}{\meta{A}}
	\have[m]{b1}{\meta{B}}
\close 
\open
	\hypo{b}{\meta{B}}
	\have[m+n]{a1}{\meta{A}}
\close
\have{ab}{\meta{A} \eiff \meta{B}}\bi{a-b1, b-a1}
\end{proof}
\\Thus showing that $\meta{A} \eiff \meta{B}$ is a theorem.}

\myanswer{There was nothing special about $\meta{A}$ and $\meta{B}$ in this. So what this shows is that the problem of showing that two sentences are provably equivalent is, essentially, the same problem as showing that a certain kind of sentence (a biconditional) is a theorem.}


\setcounter{chapter}{18}
\chapter{Derived rules}\setcounter{ProbPart}{0}
\problempart
Provide proof schemes that justify the addition of the third and fourth De Morgan rules as derived rules. 

\
\\\myanswer{Third rule:
\begin{proof}
		\have[m]{nab}{\enot \meta{A} \eand \enot \meta{B}}
		\have[k]{na}{\enot \meta{A}}\ae{nab}
		\have{nb}{\enot \meta{B}}\ae{nab}
		\open
			\hypo{aob}{\meta{A} \eor \meta{B}}
			\open
				\hypo{a}{\meta{A}}
				\have{con1}{\ered}\ri{a, na}
			\close
			\open
				\hypo{b}{\meta{B}}
				\have{con2}{\ered}\ri{b, nb}
			\close			
			\have{con3}{\ered}\oe{aob, a-con1, b-con2}
		\close
		\have{con}{\enot (\meta{A} \eor \meta{B})}\ni{aob-con3}
	\end{proof}}

\
\\\myanswer{Fourth rule:
	\begin{proof}
		\have[m]{nab}{\enot (\meta{A} \eor \meta{B})}
		\open
			\hypo[k]{a}{\meta{A}}
			\have{aob}{\meta{A} \eor \meta{B}}\oi{a}
			\have{nab1}{\ered}\ri{aob, nab}
		\close
		\have{na}{\enot \meta{A}}\ni{a-nab1}
		\open
			\hypo{b}{\meta{B}}
			\have{aob2}{\meta{A} \eor \meta{B}}\oi{b}
			\have{nab2}{\ered}\ri{aob2, nab}
		\close
		\have{nb}{\enot \meta{B}}\ni{b-nab2}
		\have{con}{\enot \meta{A} \eand \enot \meta{B}}\ai{na, nb}
	\end{proof}
}

\problempart
The proofs you offered in response to the practice exercises of \S\S\ref{s:Further}--\ref{s:ProofTheoreticConcepts} used derived rules. Replace the use of derived rules, in such proofs, with only basic rules. You will find some `repetition' in the resulting proofs; in such cases, offer a streamlined proof using only basic rules.  (This will give you a sense, both of the power of derived rules, and of how all the rules interact.)

\setcounter{chapter}{19}
\chapter{Soundness and Completeness}\setcounter{ProbPart}{0}

\practiceproblems
\noindent\problempart Use either a derivation or a truth table for each of the following. 
\begin{enumerate}%[label=(\arabic*)]
\item Show that $A \eif [((B \eand C) \eor D) \eif A]$ is a tautology.
\item Show that $A \eif (A \eif B)$ is not a tautology
\item Show that the sentence $A \eif \enot{A}$ is not a contradiction.
\item Show that the sentence $A \eiff \enot A$ is a contradiction. 
\item Show that the sentence $ \enot (W \eif (J \eor J)) $ is contingent
\item Show that the sentence $ \enot(X \eor (Y \eor Z)) \eor (X \eor (Y \eor Z))$ is not contingent
 \item Show that the sentence $B \eif \enot S$ is equivalent to the sentence $\enot \enot B \eif \enot S$
\item Show that the sentence $ \enot (X \eor O) $ is not equivalent to the sentence $X \eand O$
\item Show that the sentences $\enot(A \eor B)$, $C$, $C \eif A$  are jointly inconsistent.
\item Show that the sentences $\enot(A \eor B)$, $\enot{B}$, $B \eif A$ are jointly consistent
\item Show that $\enot(A \eor (B \eor C)) $ \therefore $ \enot{C}$ is valid.
\item Show that $\enot(A \eand (B \eor C))$ \therefore $ \enot{C}$ is invalid. 
\end{enumerate}


\noindent\problempart Use either a derivation or a truth table for each of the following. 
\begin{enumerate}%[label=(\arabic*)]
\item Show that $A \eif (B \eif A)$ is a tautology
\item Show that $\enot (((N \eiff Q) \eor Q) \eor N)$ is not a tautology
\item Show that $ Z \eor (\enot Z \eiff Z) $ is contingent
\item show that $ (L \eiff ((N \eif N) \eif L)) \eor H $ is not contingent
\item Show that $ (A \eiff A) \eand (B \eand \enot B)$ is a contradiction
\item Show that $ (B \eiff (C \eor B)) $ is not a contradiction.
\item Show that $ ((\enot X \eiff X) \eor X) $ is equivalent to $X$
\item Show that $F \eand (K \eand R) $ is not equivalent to $ (F \eiff (K \eiff R)) $
\item Show that the sentences $ \enot (W \eif W)$, $(W \eiff W) \eand W$, $E \eor (W \eif \enot (E \eand W))$ are inconsistent.
\item Show that the sentences  $\enot R \eor C $, $(C \eand R) \eif \enot R$, $(\enot (R \eor R) \eif R) $ are consistent.
\item Show that $\enot \enot (C \eiff \enot C), ((G \eor C) \eor G) \therefore ((G \eif C) \eand G) $ is valid.
\item Show that $ \enot \enot L,  (C \eif \enot L) \eif C) \therefore \enot C$ is invalid. 
\end{enumerate}
