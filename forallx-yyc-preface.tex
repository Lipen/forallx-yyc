\chapter{Preface}

As the title indicates, this is a textbook on formal logic.  Formal logic concerns the study of a certain kind of language which, like any language, can serve to express states of affairs.  It is a formal language, i.e., its expressions (such as sentences) are defined formally.  This makes it a very useful language for being very precise about the states of affairs its sentences describe. In particular, in formal logic is is impossible to be ambiguous. The study of these languages centres on the relationship of entailment between sentences, i.e., which sentences follow from which other sentences.  Entailment is central because by understanding it better we can tell when some states of affairs must obtain provided some other states of affairs obtain.  But entailment is not the only important notion. We will also consider the relationship of being consistent, i.e., of not being mutually contradictory.  These notions can be defined semantically, using precise definitions of entailment based on interpretations of the language---or proof-theoretically, using formal systems of deduction.

Formal logic is of course a central sub-discipline of philosophy, where the logical relationship of assumptions to conclusions reached from them is important.  Philosophers investigate the consequences of definitions and assumptions and evaluate these definitions and assumptions on the basis of their consequences. It is also important in mathematics and computer science. In mathematics, formal languages are used to describe not ``everyday'' states of affairs, but mathematical states of affairs. Mathematicians are also interested in the consequences of definitions and assumptions, and for them it is equally important to establish these consequences (which they call ``theorems'') using completely precise and rigorous methods. Formal logic provides such methods.  In computer science, formal logic is applied to describe the state and behaviours of computational systems, e.g., circuits, programs, databases, etc.  Methods of formal logic can likewise be used to establish consequences of such descriptions, such as whether a circuit is error-free, whether a program does what it's intended to do, whether a database is consistent or if something is true of the data in it.

The book is divided into eight parts. The first part deals introduces the topic and notions of logic in an informal way, without introducing a formal language yet.  Parts II--IV concern truth-functional languages. In it, sentences are formed from basic sentences using a number of connectives (`or', `and', `not', `if \dots then') which just combine sentences into more complicated ones.  We discuss logical notions such as entailment in two ways: semantically, using the method of truth tables (in Part~III) and proof-theoretically, using a system of formal derivations (in Part~IV).  Parts V--VII deal with a more complicated language, that of first-order logic. It includes, in addition to the connectives of truth-functional logic, also names, predicates, identity, and the so-called quantifiers. These additional elements of the language make it much more expressive that the truth-functional language, and we'll spend a fair amount of time investigating just how much one can express in it.  Again, logical notions for the language of first-order logic are defined semantically, using interpretations, and proof-theoretically, using a more complex version of the formal derivation system introduced in Part~IV.  Part~VIII covers an advanced topic: that of expressive adequacy of the truth-functional connectives.

In the appendices you'll find a discussion of alternative notations for the languages we discuss in this text, of alternative derivation systems, and a quick reference listing most of the important rules and definitions. The central terms are listed in a glossary at the very end.

This book is based on a text originally written by P.~D. Magnus and revised and expanded by Tim Button and independently by J.~Robert Loftis.  Aaron Thomas-Bolduc and Richard Zach have combined elements of these texts into the present version, changed some of the terminology and examples, and added material of their own.  The resulting text is licensed under a Creative Commons Attribution-ShareAlike 4.0 license.
