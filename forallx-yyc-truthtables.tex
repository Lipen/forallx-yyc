%!TEX root = forallxyyc.tex
\part{Truth tables}
\label{ch.TruthTables}
\addtocontents{toc}{\protect\mbox{}\protect\hrulefill\par}

\chapter{Characteristic truth tables}
\label{s:CharacteristicTruthTables}

Any non-atomic sentence of TFL is composed of atomic sentences with sentential connectives. The truth value of the compound sentence depends only on the truth value of the atomic sentences that comprise it. In order to know the truth value of `$(D \eand E)$', for instance, you only need to know the truth value of `$D$' and the truth value of `$E$'. 

We introduced five connectives in chapter \ref{ch.TFL}, so we simply need to explain how they map between truth values. For convenience, we will abbreviate `True' with `T' and `False' with `F'. (But just to be clear, the two truth values are True and False; the truth values are not \emph{letters}!)

\newglossaryentry{truth value}
                 {
                   name = truth value,
                   description = {One of the two logical values sentences can have: True and False}
                   }

\paragraph{Negation.} For any sentence \meta{A}: If \meta{A} is true, then \enot\meta{A} is false. If \enot\meta{A} is true, then \meta{A} is false. We can summarize this in the \emph{characteristic truth table} for negation:
\begin{center}
\begin{tabular}{c|c}
\meta{A} & \enot\meta{A}\\
\hline
T & F\\
F & T 
\end{tabular}
\end{center}

\paragraph{Conjunction.} For any sentences \meta{A} and \meta{B}, \meta{A}\eand\meta{B} is true if and only if both \meta{A} and \meta{B} are true. We can summarize this in the {characteristic truth table} for conjunction:
\begin{center}
\begin{tabular}{c c |c}
\meta{A} & \meta{B} & $\meta{A}\eand\meta{B}$\\
\hline
T & T & T\\
T & F & F\\
F & T & F\\
F & F & F
\end{tabular}
\end{center}
Note that conjunction is \emph{symmetrical}. The truth value for $\meta{A} \eand \meta{B}$ is always the same as the truth value for $\meta{B} \eand \meta{A}$.  

\paragraph{Disjunction.} Recall that `$\eor$' always represents inclusive or. So, for any sentences \meta{A} and \meta{B}, $\meta{A}\eor \meta{B}$ is true if and only if either \meta{A} or \meta{B} is true. We can summarize this in the {characteristic truth table} for disjunction:
\begin{center}
\begin{tabular}{c c|c}
\meta{A} & \meta{B} & $\meta{A}\eor\meta{B}$ \\
\hline
T & T & T\\
T & F & T\\
F & T & T\\
F & F & F
\end{tabular}
\end{center}
Like conjunction, disjunction is symmetrical. 

\paragraph{Conditional.} We're just going to come clean and admit it: Conditionals are a right old mess in TFL. Exactly how much of a mess they are is \emph{philosophically} contentious. We'll discuss a few of the subtleties  in \S\S\ref{s:IndicativeSubjunctive} and \ref{s:ParadoxesOfMaterialConditional}. For now, we are going to stipulate the following: $\meta{A}\eif\meta{B}$ is false if and only if \meta{A} is true and \meta{B} is false. We can summarize this with a characteristic truth table for the conditional.
\begin{center}
\begin{tabular}{c c|c}
\meta{A} & \meta{B} & $\meta{A}\eif\meta{B}$\\
\hline
T & T & T\\
T & F & F\\
F & T & T\\
F & F & T
\end{tabular}
\end{center}
The conditional is \emph{asymmetrical}. You cannot swap the antecedent and consequent without changing the meaning of the sentence, because $\meta{A}\eif\meta{B}$ has a very different truth table from $\meta{B}\eif\meta{A}$.

\paragraph{Biconditional.} Since a biconditional is to be the same as the conjunction of a conditional running in each direction, we will want the truth table for the biconditional to be:
\begin{center}
\begin{tabular}{c c|c}
\meta{A} & \meta{B} & $\meta{A}\eiff\meta{B}$\\
\hline
T & T & T\\
T & F & F\\
F & T & F\\
F & F & T
\end{tabular}
\end{center}
Unsurprisingly, the biconditional is symmetrical. 

\chapter{Truth-functional connectives}
\label{s:TruthFunctionality}

\section{The idea of truth-functionality}
Let's introduce an important idea. 
	\factoidbox{
		A connective is \define{truth-functional} iff the truth value of a sentence with that connective as its main logical operator is uniquely determined by the truth value(s) of the constituent sentence(s).
	}
\newglossaryentry{truth-functional connective}
{
name=truth-functional connective,
description={an operator that builds larger sentences out of smaller ones and fixes the \gls{truth value} of the resulting sentence based only on the truth value of the component sentences}
}
        
Every connective in TFL is truth-functional. The truth value of a negation is uniquely determined by the truth value of the unnegated sentence. The truth value of a conjunction is uniquely determined by the truth value of both conjuncts. The truth value of a disjunction is uniquely determined by the truth value of both disjuncts, and so on. To determine the truth value of some TFL sentence, we only need to know the truth value of its components. 

This is what gives TFL its name: it is \emph{truth-functional logic}.

In plenty of languages there are connectives that are not truth-functional. In English, for example, we can form a new sentence from any simpler sentence by prefixing it with `It is necessarily the case that\ldots'. The truth value of this new sentence is not fixed solely by the truth value of the original sentence. For consider two true sentences:
	\begin{earg}
		\item $2 + 2 = 4$
		\item Shostakovich wrote fifteen string quartets
	\end{earg}
Whereas it is necessarily the case that $2 + 2 = 4$, it is not \emph{necessarily} the case that Shostakovich wrote fifteen string quartets. If Shostakovich had died earlier, he would have failed to finish Quartet no.\ 15; if he had lived longer, he might have written a few more. So `It is necessarily the case that\ldots' is a connective of English, but it is not \emph{truth-functional}.


\section{Symbolising versus translating}
All of the connectives of TFL are truth-functional, but more than that: they really do nothing \emph{but} map us between truth values.  

When we symbolise a sentence or an argument in TFL, we ignore everything \emph{besides} the contribution that the truth values of a component might make to the truth value of the whole. There are subtleties to our ordinary claims that far outstrip their mere truth values. Sarcasm; poetry; snide implicature; emphasis; these are important parts of everyday discourse, but none of this is retained in TFL. As remarked in \S\ref{s:TFLConnectives}, TFL cannot capture the subtle differences between the following English sentences:
	\begin{earg}
		\item Dana is a logician and Dana is a nice person
		\item Although Dana is a logician, Dana is a nice person
		\item Dana is a logician despite being a nice person
		\item Dana is a nice person, but also a logician
		\item Dana's being a logician notwithstanding, he is a nice person
	\end{earg}
All of the above sentences will be symbolised with the same TFL sentence, perhaps `$L \eand N$'.

We keep saying that we use TFL sentences to \emph{symbolise} English sentences. Many other textbooks talk about \emph{translating} English sentences into TFL. However, a good translation should preserve certain facets of meaning, and---as we have just pointed out---TFL just cannot do that. This is why we will speak of \emph{symbolising} English sentences, rather than of \emph{translating} them.

This affects how we should understand our symbolisation keys. Consider a key like:
	\begin{ekey}
		\item[L] Dana is a logician.
		\item[N] Dana is a nice person.
	\end{ekey}
Other textbooks will understand this as a stipulation that the TFL sentence `$L$' should \emph{mean} that Dana is a logician, and that the TFL sentence `$N$' should \emph{mean} that Dana is a nice person, but TFL just is totally unequipped to deal with \emph{meaning}. The preceding symbolisation key is doing no more and no less than stipulating that the TFL sentence `$L$' should take the same truth value as the English sentence `Dana is a logician' (whatever that might be), and that the TFL sentence `$N$' should take the same truth value as the English sentence `Dana is a nice person' (whatever that might be). 
	\factoidbox{
		When we treat a TFL sentence as \emph{symbolising} an English sentence, we are stipulating that the TFL sentence is to take the same truth value as that English sentence.
	}


\section{Indicative versus subjunctive conditionals}\label{s:IndicativeSubjunctive}
We want to bring home the point that TFL can \emph{only} deal with truth functions by considering the case of the conditional. When we introduced the characteristic truth table for the material conditional in \S\ref{s:CharacteristicTruthTables}, we did not say anything to justify it. Let's now offer a justification, which follows Dorothy Edgington.\footnote{Dorothy Edgington, `Conditionals', 2006, in the \emph{Stanford Encyclopedia of Philosophy} (\url{http://plato.stanford.edu/entries/conditionals/}).} 

Suppose that Lara has drawn some shapes on a piece of paper, and coloured some of them in. We have not seen them, but nevertheless claim:
	\begin{quote}
		If any shape is grey, then that shape is also circular.
	\end{quote}
As it happens, Lara has drawn the following:
\begin{center}
\begin{tikzpicture}
	\node[circle, grey_shape] (cat1) {A};
	\node[right=10pt of cat1, diamond, phantom_shape] (cat2)  { } ;
	\node[right=10pt of cat2, circle, white_shape] (cat3)  {C} ;
	\node[right=10pt of cat3, diamond, white_shape] (cat4)  {D};
\end{tikzpicture}
\end{center}
In this case, our claim is surely true.  Shapes C and D are not grey, and so can hardly present \emph{counterexamples} to our claim. Shape A \emph{is} grey, but fortunately it is also circular. So my claim has no counterexamples. It must be true. That means that each of the following \emph{instances} of our claim must be true too:
	\begin{ebullet}
		\item If A is grey, then it is circular \hfill (true antecedent, true consequent)
		\item If C is grey, then it is circular\hfill (false antecedent, true consequent)
		\item If D is grey, then it is circular \hfill (false antecedent, false consequent)
	\end{ebullet}
However, if Lara had drawn a fourth shape, thus:
\begin{center}
\begin{tikzpicture}
	\node[circle, grey_shape] (cat1) {A};
	\node[right=10pt of cat1, diamond, grey_shape] (cat2)  {B};
	\node[right=10pt of cat2, circle, white_shape] (cat3)  {C};
	\node[right=10pt of cat3, diamond, white_shape] (cat4)  {D};
\end{tikzpicture}
\end{center}
then our claim would have be false. So it must be that this claim is false:
	\begin{ebullet}
		\item If B is grey, then it is circular \hfill (true antecedent, false consequent)
	\end{ebullet}
Now, recall that every connective of TFL has to be truth-functional. This means that merely the truth values of the antecedent and consequent must uniquely determine the truth value of the conditional as a whole. Thus, from the truth values of our four claims---which provide us with all possible combinations of truth and falsity in antecedent and consequent---we can read off the truth table for the material conditional.

What this argument shows is that `$\eif$' is the \emph{best} candidate for a truth-functional conditional. Otherwise put, \emph{it is the best conditional that TFL can provide}. But is it any good, as a surrogate for the conditionals we use in everyday language? Consider two sentences:
	\begin{earg}
		\item[\ex{brownwins1}] If Mitt Romney had won the 2012 election, then he would have been the 45th President of the USA.
		\item[\ex{brownwins2}] If Mitt Romney had won the 2012 election, then he would have turned into a helium-filled balloon and floated away into the night sky.
	\end{earg}
Sentence \ref{brownwins1} is true; sentence \ref{brownwins2} is false, but both have false antecedents and false consequents. So the truth value of the whole sentence is not uniquely determined by the truth value of the parts. Do not just blithely assume that you can adequately symbolise an English `if \dots, then \dots' with TFL's `$\eif$'. 

The crucial point is that sentences \ref{brownwins1} and \ref{brownwins2} employ \emph{subjunctive} conditionals, rather than \emph{indicative} conditionals. They ask us to imagine something contrary to fact---Mitt Romney lost the 2012 election---and then ask us to evaluate what \emph{would} have happened in that case. Such considerations just cannot be tackled using `$\eif$'.

We will say more about the difficulties with conditionals in \S\ref{s:ParadoxesOfMaterialConditional}. For now, we will content ourselves with the observation that `$\eif$' is the only candidate for a truth-functional conditional for TFL, but that many English conditionals cannot be represented adequately using `$\eif$'. TFL is an intrinsically limited language. 


\chapter{Complete truth tables}
\label{s:CompleteTruthTables}

So far, we have considered assigning truth values to TFL sentences indirectly. We have said, for example, that a TFL sentence such as `$B$' is to take the same truth value as the English sentence `Big Ben is in London' (whatever that truth value may be), but we can also assign truth values \emph{directly}. We can simply stipulate that `$B$' is to be true, or stipulate that it is to be false.
	\factoidbox{
		A \define{valuation} is any assignment of truth values to particular atomic sentences of TFL.
	}

        \newglossaryentry{valuation}
{
name=valuation,
description={An assignment of \glspl{truth value} to particular atomic \glspl{sentence of TFL}}
}

The power of truth tables lies in the following. Each row of a truth table represents a possible valuation. The entire truth table represents all possible valuations; thus the truth table provides us with a means to calculate the truth values of complex sentences, on each possible valuation. This is easiest to explain by example.

\section{A worked example}
Consider the sentence `$(H\eand I)\eif H$'. There are four possible ways to assign True and False to the atomic sentence `$H$' and `$I$'---four possible valuations---which we can represent as follows:
\begin{center}
\begin{tabular}{c c|d e e e f}
$H$&$I$&$(H$&\eand&$I)$&\eif&$H$\\
\hline
 T & T\\
 T & F\\
 F & T\\
 F & F
\end{tabular}
\end{center}
To calculate the truth value of the entire sentence `$(H \eand I) \eif H$', we first copy the truth values for the atomic sentences and write them underneath the letters in the sentence:
\begin{center}
\begin{tabular}{c c|d e e e f}
$H$&$I$&$(H$&\eand&$I)$&\eif&$H$\\
\hline
 T & T & {T} & & {T} & & {T}\\
 T & F & {T} & & {F} & & {T}\\
 F & T & {F} & & {T} & & {F}\\
 F & F & {F} & & {F} & & {F}
\end{tabular}
\end{center}
Now consider the subsentence `$(H\eand I)$'. This is a conjunction, $(\meta{A}\eand\meta{B})$, with `$H$' as \meta{A} and with `$I$' as \meta{B}. The characteristic truth table for conjunction gives the truth conditions for \emph{any} sentence of the form $(\meta{A}\eand\meta{B})$, whatever $\meta{A}$ and $\meta{B}$ might be. It represents the point that a conjunction is true iff both conjuncts are true. In this case, our conjuncts are just `$H$' and `$I$'. They are both true on (and only on) the first line of the truth table. Accordingly, we can calculate the truth value of the conjunction on all four rows.
\begin{center}
\begin{tabular}{c c|d e e e f}
 & & \meta{A} & \eand & \meta{B} & & \\
$H$&$I$&$(H$&\eand&$I)$&\eif&$H$\\
\hline
 T & T & T & {T} & T & & T\\
 T & F & T & {F} & F & & T\\
 F & T & F & {F} & T & & F\\
 F & F & F & {F} & F & & F
\end{tabular}
\end{center}
Now, the entire sentence that we are dealing with is a conditional, $\meta{A}\eif\meta{B}$, with `$(H \eand I)$' as \meta{A} and with `$H$' as \meta{B}. On the second row, for example, `$(H\eand I)$' is false and `$H$' is true. Since a conditional is true when the antecedent is false, we write a `T' in the second row underneath the conditional symbol. We continue for the other three rows and get this:
\begin{center}
\begin{tabular}{c c| d e e e f}
 & &  & \meta{A} &  &\eif &\meta{B} \\
$H$&$I$&$(H$&\eand&$I)$&\eif&$H$\\
\hline
 T & T &  & {T} &  &{T} & T\\
 T & F &  & {F} &  &{T} & T\\
 F & T &  & {F} &  &{T} & F\\
 F & F &  & {F} &  &{T} & F
\end{tabular}
\end{center}
The conditional is the main logical operator of the sentence, so the column of `T's underneath the conditional tells us that the sentence `$(H \eand I)\eif H$' is true regardless of the truth values of `$H$' and `$I$'. They can be true or false in any combination, and the compound sentence still comes out true. Since we have considered all four possible assignments of truth and falsity to `$H$' and `$I$'---since, that is, we have considered all the different \emph{valuations}---we can say that `$(H \eand I)\eif H$' is true on every valuation.

In this example, we have not repeated all of the entries in every column in every successive table. When actually writing truth tables on paper, however, it is impractical to erase whole columns or rewrite the whole table for every step. Although it is more crowded, the truth table can be written in this way:
\begin{center}
\begin{tabular}{c c| d e e e f}
$H$&$I$&$(H$&\eand&$I)$&\eif&$H$\\
\hline
 T & T & T & {T} & T & \TTbf{T} & T\\
 T & F & T & {F} & F & \TTbf{T} & T\\
 F & T & F & {F} & T & \TTbf{T} & F\\
 F & F & F & {F} & F & \TTbf{T} & F
\end{tabular}
\end{center}
Most of the columns underneath the sentence are only there for bookkeeping purposes. The column that matters most is the column underneath the \emph{main logical operator} for the sentence, since this tells you the truth value of the entire sentence. We have emphasised this, by putting this column in bold. When you work through truth tables yourself, you should similarly emphasise it (perhaps by underlining).

\section{Building complete truth tables}
A \define{complete truth table} has a line for every possible assignment of True and False to the relevant atomic sentences. Each line represents a \emph{valuation}, and a complete truth table has a line for all the different valuations. 

\newglossaryentry{complete truth table}
{
name=complete truth table,
description={A table that gives all the possible \glspl{truth value} for a \gls{sentence of TFL} or sentences in TFL, with a line for every possible \gls{valuation} of all atomic sentences}
}

The size of the complete truth table depends on the number of different atomic sentences in the table. A sentence that contains only one atomic sentence requires only two rows, as in the characteristic truth table for negation. This is true even if the same letter is repeated many times, as in the sentence
`$[(C\eiff C) \eif C] \eand \enot(C \eif C)$'.
The complete truth table requires only two lines because there are only two possibilities: `$C$' can be true or it can be false. The truth table for this sentence looks like this:
\begin{center}
\begin{tabular}{c| d e e e e e e e e e e e e e e f}
$C$&$[($&$C$&\eiff&$C$&$)$&\eif&$C$&$]$&\eand&\enot&$($&$C$&\eif&$C$&$)$\\
\hline
 T &    & T &  T  & T &   & T  & T & &\TTbf{F}&  F& &   T &  T  & T &   \\
 F &    & F &  T  & F &   & F  & F & &\TTbf{F}&  F& &   F &  T  & F &   \\
\end{tabular}
\end{center}
Looking at the column underneath the main logical operator, we see that the sentence is false on both rows of the table; i.e., the sentence is false regardless of whether `$C$' is true or false. It is false on every valuation.

A sentence that contains two atomic sentences requires four lines for a complete truth table, as in the characteristic truth tables for our binary connectives, and as in the complete truth table for `$(H \eand I)\eif H$'.

A sentence that contains three atomic sentences requires eight lines:
\begin{center}
\begin{tabular}{c c c|d e e e f}
$M$&$N$&$P$&$M$&\eand&$(N$&\eor&$P)$\\
\hline
%           M        &     N   v   P
T & T & T & T & \TTbf{T} & T & T & T\\
T & T & F & T & \TTbf{T} & T & T & F\\
T & F & T & T & \TTbf{T} & F & T & T\\
T & F & F & T & \TTbf{F} & F & F & F\\
F & T & T & F & \TTbf{F} & T & T & T\\
F & T & F & F & \TTbf{F} & T & T & F\\
F & F & T & F & \TTbf{F} & F & T & T\\
F & F & F & F & \TTbf{F} & F & F & F
\end{tabular}
\end{center}
From this table, we know that the sentence `$M\eand(N\eor P)$' can be true or false, depending on the truth values of `$M$', `$N$', and `$P$'.

A complete truth table for a sentence that contains four different atomic sentences requires 16 lines. Five letters, 32 lines. Six letters, 64 lines. And so on. To be perfectly general: If a complete truth table has $n$ different atomic sentences, then it must have $2^n$ lines.

In order to fill in the columns of a complete truth table, begin with the right-most atomic sentence and alternate between `T' and `F'. In the next column to the left, write two `T's, write two `F's, and repeat. For the third atomic sentence, write four `T's followed by four `F's. This yields an eight line truth table like the one above. For a 16 line truth table, the next column of atomic sentences should have eight `T's followed by eight `F's. For a 32 line table, the next column would have 16 `T's followed by 16 `F's, and so on.


\section{More about brackets}\label{s:MoreBracketingConventions}
Consider these two sentences:
	\begin{align*}
		((A \eand B) \eand C)\\
		(A \eand (B \eand C))
	\end{align*}
These are truth functionally equivalent. Consequently, it will never make any difference from the perspective of truth value -- which is all that TFL cares about (see \S\ref{s:TruthFunctionality}) -- which of the two sentences we assert (or deny). Even though the order of the brackets does not matter as to their truth, we should not just drop them. The expression
	\begin{align*}
		A \eand B \eand C
	\end{align*}
is ambiguous between the two sentences above.  The same observation holds for disjunctions. The following sentences are tautologically equivalent:
	\begin{align*}
		((A \eor B) \eor C)\\
		(A \eor (B \eor C))
	\end{align*}
But we should not simply write:
	\begin{align*}
		A \eor B \eor C
	\end{align*}
In fact, it is a specific fact about the characteristic truth table of $\eor$ and $\eand$ that guarantees that any two conjunctions (or disjunctions) of the same sentences are truth functionally equivalent, however you place the brackets. \emph{But be careful}. These two sentences have \emph{different} truth tables:
	\begin{align*}
		((A \eif B) \eif C)\\
		(A \eif (B \eif C))
	\end{align*}
So if we were to write:
	\begin{align*}
		A \eif B \eif C
	\end{align*}
it would be dangerously ambiguous. So we must not do the same with conditionals. Equally, these sentences have different truth tables:
	\begin{align*}
		((A \eor B) \eand C)\\
		(A \eor (B \eand C))
	\end{align*}
So if we were to write:
	\begin{align*}
		A \eor B \eand C
	\end{align*}
it would be dangerously ambiguous. \emph{Never write this.} The moral is: never drop brackets.

\practiceproblems\label{pr.TT.TTorC}
\problempart
Offer complete truth tables for each of the following:
\begin{earg}
\item $A \eif A$ %taut
\item $C \eif\enot C$ %contingent
\item $(A \eiff B) \eiff \enot(A\eiff \enot B)$ %tautology
\item $(A \eif B) \eor (B \eif A)$ % taut
\item $(A \eand B) \eif (B \eor A)$  %taut
\item $\enot(A \eor B) \eiff (\enot A \eand \enot B)$ %taut
\item $\bigl[(A\eand B) \eand\enot(A\eand B)\bigr] \eand C$ %contradiction
\item $[(A \eand B) \eand C] \eif B$ %taut
\item $\enot\bigl[(C\eor A) \eor B\bigr]$ %contingent
\end{earg}
\problempart
Check all the claims made in introducing the new notational conventions in \S\ref{s:MoreBracketingConventions}, i.e.\ show that:
\begin{earg}
	\item `$((A \eand B) \eand C)$' and `$(A \eand (B \eand C))$' have the same truth table
	\item `$((A \eor B) \eor C)$' and `$(A \eor (B \eor C))$' have the same truth table
	\item `$((A \eor B) \eand C)$' and `$(A \eor (B \eand C))$' do not have the same truth table
	\item `$((A \eif B) \eif C)$' and `$(A \eif (B \eif C))$' do not have the same truth table
\end{earg}
Also, check whether:
\begin{earg}
	\item[5.] `$((A \eiff B) \eiff C)$' and `$(A \eiff (B \eiff C))$' have the same truth table
\end{earg}

\problempart
Write complete truth tables for the following sentences and mark the column that represents the possible truth values for the whole sentence.

\begin{earg}

\item $\enot (S \eiff (P \eif S))$

%\begin{tabular}{c|c|ccccc}
%\cline{2-2}
%1.	&	\enot 	&	(S 	&	\eiff	&	(P 	&	\eif	&	S))	\\ 
%\cline{2-7}
%	& 	F 		&	T	&	T	&	T	&	T	&	T	\\
%	& 	F 		&	T	&	T	&	F	&	T	&	T	\\
%	& 	F 		&	F	&	T	&	T	&	F	&	F	\\
%	& 	T 		&	F	&	F	&	F	&	T	&	F	\\
%\cline{2-2}
%\end{tabular}


 \item $\enot [(X \eand Y) \eor (X \eor Y)]$

%\begin{tabular}{c|c|ccccccc}
%\cline{2-2}
%2.	&	\enot	&	 [(X 	&	\eand& 	Y) 	&	\eor 	&	(X 	&	\eor 	&	Y)] \\
%\cline{2-9}
%	&	F	&	T	&	T	&	T	&	T	&	T	&	T	&	T	\\
%	&	F	&	T	&	F	&	F	&	T	&	T	&	T	&	F	\\
%	&	F	&	F	&	F	&	T	&	T	&	F	&	T	&	T	\\
%	&	T	&	F	&	F	&	F	&	F	&	F	&	F	&	F	\\
%\cline{2-2}
%\end{tabular}


\item $(A \eif B) \eiff (\enot B\eiff \enot A)$
%\begin{tabular}{cccc|c|ccccc}
%\cline{5-5}
%3.	&	(A 	&	\eif	&	B)	&	 \eiff 	&	(\enot&	B 	&	\eiff 	&	 \enot 	& 	 A) \\
%\cline{2-10}
%	&	T	&	T	&	T	&	T		&	F	 &	T	&	T	&	F		&	T	\\	
%	&	T	&	F	&	F	&	T		&	T	 &	F	&	F	&	F		&	T	\\
%	&	F	&	T	&	T	&	F		&	F	 &	T	&	F	&	T		&	F	\\
%	&	F	&	T	&	F	&	T		&	T	 &	F	&	T	&	T		&	F	\\
%\cline{5-5}
%\end{tabular}

\item $[C \eiff (D \eor E)] \eand \enot C$

%\begin{tabular}{cccccc|c|cc}
%\cline{7-7}
%4.	&	[C 	&	\eiff 	&	(D 	&	\eor 	&	E)] 	&	\eand 	&	 \enot 	&	 C \\
%\cline{2-9}
%	&	T	&	T	&	T	&	T	&	T	&	F		&	F		&	T	\\
%	&	T	&	T	&	T	&	T	&	F	&	F		&	F		&	T	\\
%	&	T	&	T	&	F	&	T	&	T	&	F		&	F		&	T	\\
%	&	T	&	F	&	F	&	F	&	F	&	F		&	F		&	T	\\
%	&	F	&	F	&	T	&	T	&	T	&	F		&	T		&	F	\\
%	&	F	&	F	&	T	&	T	&	F	&	F		&	T		&	F	\\
%	&	F	&	F	&	F	&	T	&	T	&	F		&	T		&	F	\\
%	&	F	&	T	&	F	&	F	&	F	&	T		&	T		&	F	\\
%\cline{7-7}
%\end{tabular}

\item $\enot(G \eand (B \eand H)) \eiff (G \eor (B \eor H))$
%
%\begin{tabular}{ccccccc|c|ccccc}
%\cline{8-8}
%5.	&\enot&	(G 	&\eand &	(B 	&	 \eand 	&	 H))	&	\eiff 	&	(G 	& \eor 	& (B 	& \eor	& H))	\\
%\cline{2-13}
%	&F	   &	T	&	  T &	T	&	T		&	T	&	F	&	T	&	T	&	T	&	T	&	T	\\
%	&T	   &	T	&	  F &	T	&	F		&	F	&	T	&	T	&	T	&	T	&	T	&	F	\\	
%	&T	   &	T	&	 F  &	F	&	F		&	T	&	T	&	T	&	T	&	F	&	T	&	T	\\
%	&T	   &	T	&	 F  &	F	&	F		&	F	&	T	&	T	&	T	&	F	&	F	&	F	\\
%	&T	   &	F	&	F   &	T	&	T		&	T	&	T	&	F	&	T	&	T	&	T	&	T	\\
%	&T	   &	F	&	F   &	T	&	F		&	F	&	T	&	F	&	T	&	T	&	T	&	F	\\
%	&T	   &	F	&	F   &	F	&	F		&	T	&	T	&	F	&	T	&	F	&	T	&	T	\\
%	&T	   &	F	&	F   &	F	&	F		&	F	&	F	&	F	&	F	&	F	&	F	&	F	\\
%\cline{8-8}
%\end{tabular}

%\vspace{1em}

\end{earg}

\problempart
Write complete truth tables for the following sentences and mark the column that represents the possible truth values for the whole sentence.

\begin{earg}

\item	$(D \eand \enot D) \eif G $

%\vspace{1em}

%\begin{tabular}{ccccc|c|c}
%\cline{6-6}
%1.	&	(D 	&	 \eand 	& 	 \enot	&	 D) 	&	 \eif 	&	 G \\
%	&	T	&	F		&	F		&	T	&	T	&	T	\\
%	&	T	&	F		&	F		&	T	&	T	&	F	\\
%	&	F	&	F		&	T		&	F	&	T	&	T	\\
%	&	F	&	F		&	T		&	F	&	T	&	F	\\
%\cline{6-6}
%\end{tabular}
%\vspace{1em}


\item	$(\enot P \eor \enot M) \eiff M $

%\begin{tabular}{cccccc|c|c}
%\cline{7-7}
%2.	&	(\enot 	&	P 	&	\eor 	&	\enot 	& 	 M) 	& 	\eiff 	&	 M \\
%	&	F		&	T	&	F	&	F		&	T	&	T	&	T	\\
%	&	F		&	T	&	T	&	T		&	F	&	F	&	F	\\
%	&	T		&	F	&	T	&	F		&	T	&	T	&	T	\\
%	&	T		&	F	&	T	&	T		&	F	&	T	&	F	\\
%\cline{7-7}
%\end{tabular}
%\vspace{1em}



\item	$\enot \enot (\enot A \eand \enot B)  $

%\begin{tabular}{c|c|cccccc}
%\cline{2-2}
%3.	&	\enot		&	 \enot 	&	(\enot 	& 	 A 	& \eand 	& 	\enot 	&	 B)  \\
%	&	F		&	T		&	F		&	T	&	F	&	F		&	T	\\
%	&	F		&	T		&	F		&	T	&	F	&	T		&	F	\\
%	&	F		&	T		&	T		&	F	&	F	&	F		&	T	\\
%	&	T		&	F		&	T		&	F	&	T	&	T		&	F	\\
%\cline{2-2}
%\end{tabular}
%\vspace{1em}



\item 	$[(D \eand R) \eif I] \eif \enot(D \eor R) $

%\begin{tabular}{cccccc|c|cccc}
%\cline{7-7}
%4.	&	[(D 	& 	 \eand 	& 	 R)	& 	\eif 	&	I] 	&	\eif 	&	 \enot 	&	(D 	&	 \eor 	& R) \\
%	&	T	&	T		&	T	&	T	&	T	&	F	&	F		&	T	&	T		&T	\\
%	&	T	&	T		&	T	&	F	&	F	&	T	&	F		&	T	&	T		&T	\\
%	&	T	&	F		&	F	&	T	&	T	&	F	&	F		&	T	&	T		&F	\\
%	&	T	&	F		&	F	&	T	&	F	&	F	&	F		&	T	&	T		&F	\\
%	&	F	&	F		&	T	&	T	&	T	&	F	&	F		&	F	&	T		&T	\\
%	&	F	&	F		&	T	&	T	&	F	&	F	&	F		&	F	&	T		&T	\\
%	&	F	&	F		&	F	&	T	&	T	&	T	&	T		&	F	&	F		&F	\\
%	&	F	&	F		&	F	&	T	&	F	&	T	&	T		&	F	&	F		&F	\\
%\cline{7-7}
%\end{tabular}
%	
%\vspace{1em}


\item	$\enot [(D \eiff O) \eiff A] \eif (\enot D \eand O) $

%\begin{tabular}{ccccccc|c|cccc}
%\cline{8-8}
%5.	&	\enot 	&	[(D 	&	\eiff 	&	O) 	&	\eiff 	&	 A]	& 	\eif 	 &	(\enot 	& 	D 	 & 	 \eand &O) \\ 
%	&	F		&	T	&	T	&	T	&	T	&	T	&	T	&	F		&	T	&	F	&T	\\
%	&	T		&	T	&	T	&	T	&	F	&	F	&	F	&	F		&	T	&	F	&T	\\
%	&	T		&	T	&	F	&	F	&	F	&	T	&	F	&	F		&	T	&	F	&F	\\
%	&	F		&	T	&	F	&	F	&	T	&	F	&	T	&	F		&	T	&	F	&F	\\
%	&	T		&	F	&	F	&	T	&	F	&	T	&	T	&	T		&	F	&	T	&T	\\
%	&	F		&	F	&	F	&	T	&	T	&	F	&	T	&	T		&	F	&	T	&T	\\
%	&	F		&	F	&	T	&	F	&	T	&	T	&	T	&	T		&	F	&	F	&F	\\
%	&	T		&	F	&	T	&	F	&	F	&	F	&	T	&	T		&	F	&	F	&F	\\
%\cline{8-8}
%\end{tabular}
%\vspace{1em}
\end{earg}


If you want additional practice, you can construct truth tables for any of the sentences and arguments in the exercises for the previous chapter.


\chapter{Semantic concepts}
\label{s:SemanticConcepts}

In the previous section, we introduced the idea of a valuation and showed how to determine the truth value of any TFL sentence, on any valuation, using a truth table. In this section, we will introduce some related ideas, and show how to use truth tables to test whether or not they apply.


\section{Tautologies and contradictions}
In \S\ref{s:BasicNotions}, we explained \emph{necessary truth} and \emph{necessary falsity}. Both notions have surrogates in TFL. We will start with a surrogate for necessary truth.
	\factoidbox{
		$\meta{A}$ is a \define{tautology} iff it is true on every valuation.
	}

\newglossaryentry{tautology}
{
name=tautology,
description={A sentence that has only Ts in the column under the main logical operator of its \gls{complete truth table}; a sentence that is true on every \gls{valuation}}
}

We can determine whether a sentence is a tautology just by using truth tables. If the sentence is true on every line of a complete truth table, then it is true on every valuation, so it is a tautology. In the example of \S\ref{s:CompleteTruthTables}, `$(H \eand I) \eif H$' is a tautology. 

This is only, though, a \emph{surrogate} for necessary truth. There are some necessary truths that we cannot adequately symbolise in TFL. An example is `$2 + 2 = 4$'. This \emph{must} be true, but if we try to symbolise it in TFL, the best we can offer is an atomic sentence, and no atomic sentence is a tautology. Still, if we can adequately symbolise some English sentence using a TFL sentence which is a tautology, then that English sentence expresses a necessary truth.

We have a similar surrogate for necessary falsity:
	\factoidbox{
		$\meta{A}$ is a \define{contradiction} iff it is false on every valuation.
	}
\newglossaryentry{contradiction of TFL}
{
  name=contradiction (of TFL),
  text = contradiction,
description={A sentence that has only Fs in the column under the main logical operator of its \gls{complete truth table}; a sentence that is false on every \gls{valuation}}
}

We can determine whether a sentence is a contradiction just by using truth tables. If the sentence is false on every line of a complete truth table, then it is false on every valuation, so it is a contradiction. In the example of \S\ref{s:CompleteTruthTables}, `$[(C\eiff C) \eif C] \eand \enot(C \eif C)$' is a contradiction.


\section{Logical equivalence}
Here is a similar useful notion:
	\factoidbox{
		$\meta{A}$ and $\meta{B}$ are \define{logically equivalent} iff, for every valuation, their truth values agree, i.e.\ if there is no valuation in which they have opposite truth values.
	}
\newglossaryentry{logically equivalent}
{
  name=logical equivalence (in TFL),
  text = logically equivalent,
description={A property held by pairs of sentences if and only if the \gls{complete truth table} for those sentences has identical columns under the two main logical operators, i.e., if the sentences have the same truth value on every valuation}
}

We have already made use of this notion, in effect, in \S\ref{s:MoreBracketingConventions}; the point was that `$(A \eand B) \eand C$' and  `$A \eand (B \eand C)$' are logically equivalent. Again, it is easy to test for logical equivalence using truth tables. Consider the sentences `$\enot(P \eor Q)$' and `$\enot P \eand \enot Q$'. Are they logically equivalent? To find out, we construct a truth table.
\begin{center}
\begin{tabular}{c c|d e e f |d e e e f}
$P$&$Q$&\enot&$(P$&\eor&$Q)$&\enot&$P$&\eand&\enot&$Q$\\
\hline
 T & T & \TTbf{F} & T & T & T & F & T & \TTbf{F} & F & T\\
 T & F & \TTbf{F} & T & T & F & F & T & \TTbf{F} & T & F\\
 F & T & \TTbf{F} & F & T & T & T & F & \TTbf{F} & F & T\\
 F & F & \TTbf{T} & F & F & F & T & F & \TTbf{T} & T & F
\end{tabular}
\end{center}
Look at the columns for the main logical operators; negation for the first sentence, conjunction for the second. On the first three rows, both are false. On the final row, both are true. Since they match on every row, the two sentences are logically equivalent.


\section{Consistency}
In \S\ref{s:BasicNotions}, we said that sentences are jointly possible iff it is possible for all of them to be true at once. We can offer a surrogate for this notion too:
	\factoidbox{
		$\meta{A}_1, \meta{A}_2, \ldots, \meta{A}_n$ are \define{jointly logically consistent} iff there is some valuation which makes them all true.
	}

        \newglossaryentry{logical consistency in TFL}
{
  name=logical consistency (in TFL),
  text=jointly logically consistent,
description={A property held by sentences if and only if the \gls{complete truth table} for those sentences contains one line on which all the sentences are true, i.e., if some \gls{valuation} makes all the sentences true}
}

Derivatively, sentences are jointly logically inconsistent if there is no valuation that makes them all true. Again, it is easy to test for joint logical consistency using truth tables. 

\section{Entailment and validity}
The following idea is closely related to that of joint consistency:
	\factoidbox{
		The sentences $\meta{A}_1, \meta{A}_2, \ldots, \meta{A}_n$ \define{entail} the sentence $\meta{C}$ if there is no valuation of the atomic sentences which makes all of $\meta{A}_1, \meta{A}_2, \ldots, \meta{A}_n$ true and $\meta{C}$ false.
	}
       \newglossaryentry{logically valid in TFL}
{
  name=logical validity (in TFL),
  text = logically valid,
description={A property held by arguments if and only if the \gls{complete truth table} for the argument contains no rows where the \glspl{premise} are all true and the \gls{conclusion} false, i.e., if no \gls{valuation} makes all premises true and the conclusion false}
}
 
Again, it is easy to test this with a truth table. Let us check whether `$\enot L \eif (J \eor L)$' and `$\enot L$' entail `$J$', we simply need to check whether there is any valuation which makes both `$\enot L \eif (J \eor L)$' and `$\enot L$' true whilst making `$J$' false. So we use a truth table: 
\begin{center}
\begin{tabular}{c c|d e e e e f|d f| c}
$J$&$L$&\enot&$L$&\eif&$(J$&\eor&$L)$&\enot&$L$&$J$\\
\hline
%J   L   -   L      ->     (J   v   L)
 T & T & F & T & \TTbf{T} & T & T & T & \TTbf{F} & T & \TTbf{T}\\
 T & F & T & F & \TTbf{T} & T & T & F & \TTbf{T} & F & \TTbf{T}\\
 F & T & F & T & \TTbf{T} & F & T & T & \TTbf{F} & T & \TTbf{F}\\
 F & F & T & F & \TTbf{F} & F & F & F & \TTbf{T} & F & \TTbf{F}
\end{tabular}
\end{center}
The only row on which both`$\enot L \eif (J \eor L)$' and `$\enot L$' are true is the second row, and that is a row on which `$J$' is also true. So `$\enot L \eif (J \eor L)$' and `$\enot L$' entail `$J$'.

We now make an important observation:
	\factoidbox{
		If $\meta{A}_1, \meta{A}_2, \ldots, \meta{A}_n$ entail $\meta{C}$, then $\meta{A}_1, \meta{A}_2, \ldots, \meta{A}_n \therefore \meta{C}$ is valid.
	}
Here's why. If $\meta{A}_1, \meta{A}_2, \ldots, \meta{A}_n$ entail $\meta{C}$, then there is no valuation which makes all of $\meta{A}_1, \meta{A}_2, \ldots, \meta{A}_n$ true whilst making $\meta{C}$ false. This means that it is \emph{logically impossible} for $\meta{A}_1, \meta{A}_2, \ldots, \meta{A}_n$ all to be true whilst $\meta{C}$ is false. But this is just what it takes for an argument, with premises $\meta{A}_1, \meta{A}_2, \ldots, \meta{A}_n$ and conclusion $\meta{C}$, to be valid!

In short, we have a way to test for the validity of English arguments. First, we symbolise them in TFL, as having premises $\meta{A}_1, \meta{A}_2, \ldots, \meta{A}_n$, and conclusion $\meta{C}$. Then we test for entailment using truth tables. 


\section{The limits of these tests}\label{s:ParadoxesOfMaterialConditional}
We have reached an important milestone: a test for the validity of arguments! However, we should not get carried away just yet. It is important to understand the \emph{limits} of our achievement. We will illustrate these limits with three examples.

First, consider the argument: 
	\begin{earg}
		\item Daisy has four legs. So Daisy has more than two legs.
	\end{earg}
To symbolise this argument in TFL, we would have to use two different atomic sentences -- perhaps `$F$'  and `$T$' -- for the premise and the conclusion respectively. Now, it is obvious that `$F$' does not entail `$T$'. The English argument surely seems valid, though!

Second, consider the sentence:
	\begin{earg}
\setcounter{eargnum}{1}
		\item\label{n:JanBald} Jan is neither bald nor not-bald.
	\end{earg}
To symbolise this sentence in TFL, we would offer something like `$\enot J \eand \enot \enot J$'. This a contradiction (check this with a truth-table), but sentence \ref{n:JanBald} does not itself seem like a contradiction; for we might have happily go on to add `Jan is on the borderline of baldness'!

Third, consider the following sentence:
	\begin{earg}
\setcounter{eargnum}{2}	
		\item\label{n:GodParadox}	It's not the case that, if God exists, She answers malevolent prayers.
%	Aaliyah wants to kill Zebedee. She knows that, if she puts chemical A into Zebedee's water bottle, Zebedee will drink the contaminated water and die. Equally, Bathsehba wants to kill Zebedee. She knows that, if she puts chemical B into Zebedee's water bottle, then Zebedee will drink the contaminated water and die. But chemicals A and B neutralise each other; so that if both are added to the water bottle, then Zebedee will not die.
	\end{earg}
        Symbolising this in TFL, we would offer something like `$\enot (G \eif M)$'. Now, `$\enot (G \eif M)$' entails `$G$' (again, check this with a truth table). So if we symbolise sentence \ref{n:GodParadox} in TFL, it seems to entail that God exists. But that's strange: surely even an atheist can accept sentence \ref{n:GodParadox}, without contradicting herself!

        One lesson of this is that the symbolization of \ref{n:GodParadox} as `$\enot(G \eif M)$' shows that \ref{n:GodParadox} does not express what we intend. Perhaps we should rephrase it as
        	\begin{earg}
                  \setcounter{eargnum}{2}	
                \item\label{n:GodParadox2} If God exists, She does not answer malevolent prayers.
  \end{earg}
and symbolize \ref{n:GodParadox2} as `$G \eif \enot M$'.  Now, if atheists are right, and there is no God, then `$G$' is false and so `$G \eif \enot M$' is true, and the puzzle disappears. However, if `$G$' is false, `$G \eif M$', i.e.\ `If God exists, She answers malevolent prayers', is \emph{also} true!
                
In different ways, these four examples highlight some of the limits of working with a language (like TFL) that can \emph{only} handle truth-functional connectives. Moreover, these limits give rise to some interesting questions in philosophical logic. The case of Jan's baldness (or otherwise) raises the general question of what logic we should use when dealing with \emph{vague} discourse. The case of the atheist raises the question of how to deal with the (so-called) \emph{paradoxes of the material conditional}. Part of the purpose of this course is to equip you with the tools to explore these questions of \emph{philosophical logic}. But we have to walk before we can run; we have to become proficient in using TFL, before we can adequately discuss its limits, and consider alternatives. 

\section{The double-turnstile}
We are going to use the notion of entailment rather a lot in this course. It will help us, then, to introduce a symbol that abbreviates it. Rather than saying that the TFL sentences $\meta{A}_1, \meta{A}_2, \ldots$ and $\meta{A}_n$ together entail $\meta{C}$, we will abbreviate this by:
	$$\meta{A}_1, \meta{A}_2, \ldots, \meta{A}_n \entails \meta{C}$$
The symbol `$\entails$' is known as \emph{the double-turnstile}, since it looks like a turnstile with two horizontal beams.

Let me be clear. `$\entails$' is not a symbol of TFL. Rather, it is a symbol of our metalanguage, augmented English (recall the difference between object language and metalanguage from \S\ref{s:UseMention}). So the metalanguage sentence:
	\begin{ebullet}
		\item $P, P \eif Q \entails Q$
	\end{ebullet}
is just an abbreviation for the English sentence: 
	\begin{ebullet}
		\item The TFL sentences `$P$' and `$P \eif Q$' entail `$Q$'
	\end{ebullet}
Note that there is no limit on the number of TFL sentences that can be mentioned before the symbol `$\entails$'. Indeed, we can even consider the limiting case:
	$$\entails \meta{C}$$
This says that there is no valuation which makes all the sentences mentioned on the left hand side of `$\entails$' true whilst making $\meta{C}$ false. Since \emph{no} sentences are mentioned on the left hand side of `$\entails$' in this case, this just means that there is no valuation which makes $\meta{C}$ false. Otherwise put, it says that every valuation makes $\meta{C}$ true. Otherwise put, it says that $\meta{C}$ is a tautology. Equally:
	$$\meta{A} \entails$$
says that $\meta{A}$ is a contradiction.

\section{`$\entails$' versus `$\eif$'}
We now want to compare and contrast `$\entails$' and `$\eif$'. 

Observe: $\meta{A} \entails \meta{C}$ iff there is no valuation of the atomic sentences that makes $\meta{A}$ true and $\meta{C}$ false. 

Observe: $\meta{A} \eif \meta{C}$ is a tautology iff there is no valuation of the atomic sentences that makes $\meta{A} \eif \meta{C}$ false. Since a conditional is true except when its antecedent is true and its consequent false, $\meta{A} \eif \meta{C}$ is a tautology iff there is no valuation that makes $\meta{A}$ true and $\meta{C}$ false. 

Combining these two observations, we see that $\meta{A} \eif \meta{C}$  is a tautology iff  $\meta{A} \entails \meta{C}$. But there is a really, really important difference between `$\entails$' and `$\eif$':
	\factoidbox{`$\eif$' is a sentential connective of TFL.\\ `$\entails$' is a symbol of augmented English.
	}
Indeed, when `$\eif$' is flanked with two TFL sentences, the result is a longer TFL sentence. By contrast, when we use `$\entails$', we form a metalinguistic sentence that \emph{mentions} the surrounding TFL sentences. 


\practiceproblems
\problempart
Revisit your answers to \S\ref{s:CompleteTruthTables}\textbf{A}. Determine which sentences were tautologies, which were contradictions, and which were neither tautologies nor contradictions.
\solutions

\

\problempart
\label{pr.TT.consistent}
Use truth tables to determine whether these sentences are jointly consistent, or jointly inconsistent:
\begin{earg}
\item $A\eif A$, $\enot A \eif \enot A$, $A\eand A$, $A\eor A$ %consistent
\item $A\eor B$, $A\eif C$, $B\eif C$ %consistent
\item $B\eand(C\eor A)$, $A\eif B$, $\enot(B\eor C)$  %inconsistent
\item $A\eiff(B\eor C)$, $C\eif \enot A$, $A\eif \enot B$ %consistent
\end{earg}


\solutions
\problempart
\label{pr.TT.valid}
Use truth tables to determine whether each argument is valid or invalid.
\begin{earg}
\item $A\eif A \therefore A$ %invalid
\item $A\eif(A\eand\enot A) \therefore \enot A$ %valid
\item $A\eor(B\eif A) \therefore \enot A \eif \enot B$ %valid
\item $A\eor B, B\eor C, \enot A \therefore B \eand C$ %invalid
\item $(B\eand A)\eif C, (C\eand A)\eif B \therefore (C\eand B)\eif A$ %invalid
\end{earg}

\problempart Determine whether each sentence is a tautology, a contradiction, or a contingent sentence, using a complete truth table.
\begin{earg}
\item $\enot B \eand B$ \vspace{.5ex}%contra


\item $\enot D \eor D$ \vspace{.5ex}%taut


\item $(A\eand B) \eor (B\eand A)$\vspace{.5ex} %contingent


\item $\enot[A \eif (B \eif A)]$\vspace{.5ex} %contra


\item $A \eiff [A \eif (B \eand \enot B)]$ \vspace{.5ex}%contra


\item $[(A \eand B) \eiff B] \eif (A \eif B)$ \vspace{.5ex}% contingent. 

\end{earg}



\noindent\problempart
\label{pr.TT.equiv}
Determine whether each the following sentences are logically equivalent using complete truth tables. If the two sentences really are logically equivalent, write ``equivalent.'' Otherwise write, ``Not equivalent.'' 
\begin{earg}
\item $A$ and $\enot A$
\item $A \eand \enot A$ and $\enot B \eiff B$
\item $[(A \eor B) \eor C]$ and $[A \eor (B \eor C)]$
\item $A \eor (B \eand C)$ and $(A \eor B) \eand (A \eor C)$
\item $[A \eand (A \eor B)] \eif B$ and $A \eif B$\end{earg}


\problempart
\label{pr.TT.equiv2}
Determine whether each the following sentences are logically equivalent using complete truth tables. If the two sentences really are equivalent, write ``equivalent.'' Otherwise write, ``not equivalent.''
\begin{earg}
\item $A\eif A$ and $A \eiff A$
\item $\enot(A \eif B)$ and $\enot A \eif \enot B$
\item $A \eor B$ and $\enot A \eif B$
\item$(A \eif B) \eif C$ and $A \eif (B \eif C)$
\item $A \eiff (B \eiff C)$ and $A \eand (B \eand C)$
\end{earg}


\problempart
\label{pr.TT.consistent2}
Determine whether each collection of sentences is consistent or inconsistent using a complete truth table. 
\begin{earg}
\item $A \eand \enot B$, $\enot(A \eif B)$, $B \eif A$\vspace{.5ex} %Consistent

%\begin{tabular}{ccccccccccccccc} 
%1. 	&	A 					 & \eand 		&  \enot & B & & \enot  		& 	 (A	  & 	 \eif	 	 & 	 B)		 & 	 & 	 B	 	 & 	\eif 	 	 & 	A 	 	 & 	 Consistent \\ 
%\cline{2-5} \cline{7-10}\cline{12-14} 
%	& 	T 					 & 	 F	 		&  F	 & T & & F	 		& 	 T	  & 	 T	 	 & 	T 	 	 & 	 & 	 T	 	 & 	 T	 	 & T	 	 	&	  \\ 
%\cline{2-14}
%	& \multicolumn{1}{|r}{T}& 	\textbf{T}	 & T	 & F & & \textbf{T}	 & 	 T	 & 	 F	 	 & 	 F	 	 & 	 & 	 F	 	 & 	 \textbf{T}	 	 & 	 \multicolumn{1}{r|}{T}	 	 & 	  \\ 
%\cline{2-14}
%	& 	 F	 				 & 	 F	 & 	 F	 & T & 	& 	 F	 & 	 F	 & 	 T	 	 & 	 T	 	 & 	  & 	 T	 	 & 	 F	 	 & 	 F	 	 & 	  \\ 
%	& 	 F	  				& 	 F	 & 	 T	 & 	F&  & 	 F	 & 	 F	 & 	 T	 	 & 	 F	 	 & 	  & 	 F	 	 & 	 T	 	 & 	 F	 	 & 	  \\ 
%\end{tabular}

\item $A \eor B$, $A \eif \enot A$, $B \eif \enot B$ \vspace{.5ex}%inconsistent. 

%\begin{tabular}{ccccccccccccccc} 
%2. &A	 & \eor 	 & B 	 & 	 	 & A 	 & \eif 	 & 	\enot & A 	 & 	 	 & B 	 & \eif 	 & \enot	 & 	B 	 & 	Inconsistent \\ 
%\cline{2-4}\cline{6- 9} \cline{11-14}
%   &	T	 & 	 T	 &T  	 & 	 	 & T	 & 	 F	 & 	F 	 & T 	 & 	 	 & 	T 	 & 	F 	 & 	 F	 & 	T 	 & 	 \\ 
%   &	 T	& 	 T	 & F 	 & 	 	 & 	T 	 & 	 F	 & 	 F	 & 	 T	 & 	 	 & 	F 	 & 	 T	 & 	 T	 & 	 F	 & 	 \\ 
%   &	 F	& 	 T	 & 	 T	 & 	 	 & 	F 	 & 	 T	 & 	 T	 & 	F 	 & 	 	 & 	 T	 & 	 F	 & 	 F	 & 	 T	 & 	 \\ 
%   &	 F	& 	 F	 & 	 F	 & 	 	 & 	 F	 & 	 T	 & 	 T	 & 	 F	 & 	 	 & 	 F	 & 	 T	 & 	 T	 & 	 F	 & 	 \\ 
%\end{tabular}

\item $\enot(\enot A \eor B) $, $A \eif \enot C$, $A \eif (B \eif C)$\vspace{.5ex} %Inconsistent

%3. &\enot & (\enot & A & \eor &B) &  &A  & \eif 	 &\enot 	 &C & 	 & A &\eif 	& (B 	 &\eif 	& C)	 &Consistent \\ 
%\cline{2-6}\cline{8-11} \cline{13-17} 
%   &	F 	& 	F	 & 	T & T	 & T & 	  & T & F	 & 	 F&T 	 & 	 &T & T	 & T	 &T 	 &T 	 & \\ 
%   &	 F	& 	F	 & 	T & T	 & T & 	  & T & T	 & 	 T& F	 & 	 &T & F	 & T	 & F	 &F 	 & \\ 
% 
%  &	 T & 	F 	& 	T & F	 & F & 	  & T & F	 & 	 F& T	 & 	 &T & T	 & F	 & T	 &T 	 & \\ 
%\cline{2-17}
%   &	 \multicolumn{1}{|r}{{\color{red}T}}		&  F	 & 	T & F	 & 	F &  & 	T & {\color{red}T}	 & 	 T&F 	& 	 &T & {\color{red}T}	 & F	 & T	 &\multicolumn{1}{r|}{F} 	 & \\ 
%\cline{2-17}
%   &	 F	& 	T	 & 	F & T	 & 	T &  & 	F & T	 & 	 F& T	 & 	 &F	 & F	 & T	 & T	 &T 	 & \\ 
%   &	 F	& 	 T	& 	F & T	 & 	T &  & 	F & T	 & 	T & F 	& 	 &F	 & T	 & T	 &F 	 &F 	 & \\ 
%   &	 F	& 	 T	& 	F & T	 & 	F &  & 	F & T	 & 	F & T	 & 	 &F	 & T	 & F	 & T	 &T 	 & \\ 
%   &	 F	& 	 T	& 	F & T	 & 	F &  & 	F & T	 & 	T & F	 & 	 &F	 & T	 & F	 & T	 &F 	 & \\ 
%\end{tabular}
%


\item $A \eif B$, $A \eand \enot B$\vspace{.5ex} %Inconsistent

\item $A \eif (B \eif C)$, $(A \eif B) \eif C$, $A \eif C$\vspace{.5ex} % consistent. 

\end{earg}

\noindent\problempart
\label{pr.TT.consistent3}
Determine whether each collection of sentences is consistent or inconsistent, using a complete truth table. 
\begin{earg}
\item $\enot B$, $A \eif B$, $A$ \vspace{.5ex}%inconsistent.
\item $\enot(A \eor B)$, $A \eiff B$, $B \eif A$\vspace{.5ex} %Consistent
\item $A \eor B$, $\enot B$, $\enot B \eif \enot A$\vspace{.5ex} %Inconsistent
\item $A \eiff B$, $\enot B \eor \enot A$, $A \eif B$\vspace{.5ex} %consistent. 
\item $(A \eor B) \eor C$, $\enot A \eor \enot B$, $\enot C \eor \enot B$\vspace{.5ex} %consistent
\end{earg}




\noindent\problempart
\label{pr.TT.valid2}
Determine whether each argument is valid or invalid, using a complete truth table. 
\begin{earg}
\item $A\eif B$, $B \therefore  A$ %invalid

\item $A\eiff B$, $B\eiff C \therefore A\eiff C$ %valid

\item $A \eif B$, $A \eif C\therefore B \eif C$ %invalid. 

\item $A \eif B$, $B \eif A\therefore A \eiff B$ %valid. 

\end{earg}

\noindent\problempart
\label{pr.TT.valid3}
Determine whether each argument is valid or invalid, using a complete truth table. 
\begin{earg}
\item $A\eor\bigl[A\eif(A\eiff A)\bigr] \therefore  A $\vspace{.5ex}%invalid
\item $A\eor B$, $B\eor C$, $\enot B \therefore A \eand C$\vspace{.5ex} %valid
\item $A \eif B$, $\enot A\therefore \enot B$ \vspace{.5ex}%invalid
\item $A$, $B\therefore \enot(A\eif \enot B)$ \vspace{.5ex}%valid
\item $\enot(A \eand B)$, $A \eor B$, $A \eiff B\therefore C$ \vspace{.5ex}%valid 
\end{earg}

\solutions
\problempart
\label{pr.TT.concepts}
Answer each of the questions below and justify your answer.
\begin{earg}
\item Suppose that \meta{A} and \meta{B} are logically equivalent. What can you say about $\meta{A}\eiff\meta{B}$?
%\meta{A} and \meta{B} have the same truth value on every line of a complete truth table, so $\meta{A}\eiff\meta{B}$ is true on every line. It is a tautology.
\item Suppose that $(\meta{A}\eand\meta{B})\eif\meta{C}$ is neither a tautology nor a contradiction. What can you say about whether $\meta{A}, \meta{B} \therefore\meta{C}$ is valid?
%The sentence is false on some line of a complete truth table. On that line, \meta{A} and \meta{B} are true and \meta{C} is false. So the argument is invalid.
\item Suppose that $\meta{A}$, $\meta{B}$ and $\meta{C}$  are jointly inconsistent. What can you say about $(\meta{A}\eand\meta{B}\eand\meta{C})$?
\item Suppose that \meta{A} is a contradiction. What can you say about whether $\meta{A}, \meta{B} \entails \meta{C}$?
%Since \meta{A} is false on every line of a complete truth table, there is no line on which \meta{A} and \meta{B} are true and \meta{C} is false. So the argument is valid.
\item Suppose that \meta{C} is a tautology. What can you say about whether $\meta{A}, \meta{B}\entails \meta{C}$?
%Since \meta{C} is true on every line of a complete truth table, there is no line on which \meta{A} and \meta{B} are true and \meta{C} is false. So the argument is valid.
\item Suppose that \meta{A} and \meta{B} are logically equivalent. What can you say about $(\meta{A}\eor\meta{B})$?
%Not much. $(\meta{A}\eor\meta{B})$ is a tautology if \meta{A} and \meta{B} are tautologies; it is a contradiction if they are contradictions; it is contingent if they are contingent.
\item Suppose that \meta{A} and \meta{B} are \emph{not} logically equivalent. What can you say about $(\meta{A}\eor\meta{B})$?
%\meta{A} and \meta{B} have different truth values on at least one line of a complete truth table, and $(\meta{A}\eor\meta{B})$ will be true on that line. On other lines, it might be true or false. So $(\meta{A}\eor\meta{B})$ is either a tautology or it is contingent; it is \emph{not} a contradiction.
\end{earg}
\problempart 
Consider the following principle:
	\begin{ebullet}
		\item Suppose $\meta{A}$ and $\meta{B}$ are logically equivalent. Suppose an argument contains $\meta{A}$ (either as a premise, or as the conclusion). The validity of the argument would be unaffected, if we replaced $\meta{A}$ with $\meta{B}$.
	\end{ebullet}
Is this principle correct? Explain your answer.



\chapter{Truth table shortcuts}
With practice, you will quickly become adept at filling out truth tables. In this section, we want to give you some permissible shortcuts to help you along the way. 

\section{Working through truth tables}
You will quickly find that you do not need to copy the truth value of each atomic sentence, but can simply refer back to them. So you can speed things up by writing:
\begin{center}
\begin{tabular}{c c|d e e e e f}
$P$&$Q$&$(P$&\eor&$Q)$&\eiff&\enot&$P$\\
\hline
 T & T &  & T &  & \TTbf{F} & F\\
 T & F &  & T &  & \TTbf{F} & F\\
 F & T &  & T & & \TTbf{T} & T\\
 F & F &  & F &  & \TTbf{F} & T
\end{tabular}
\end{center}
You also know for sure that a disjunction is true whenever one of the disjuncts is true. So if you find a true disjunct, there is no need to work out the truth values of the other disjuncts. Thus you might offer:
\begin{center}
\begin{tabular}{c c|d e e e e e e f}
$P$&$Q$& $(\enot$ & $P$&\eor&\enot&$Q)$&\eor&\enot&$P$\\
\hline
 T & T & F & & F & F& & \TTbf{F} & F\\
 T & F &  F & & T& T& &  \TTbf{T} & F\\
 F & T & & &  & & & \TTbf{T} & T\\
 F & F & & & & & &\TTbf{T} & T
\end{tabular}
\end{center}
Equally, you know for sure that a conjunction is false whenever one of the conjuncts is false. So if you find a false conjunct, there is no need to work out the truth value of the other conjunct. Thus you might offer:
\begin{center}
\begin{tabular}{c c|d e e e e e e f}
$P$&$Q$&\enot &$(P$&\eand&\enot&$Q)$&\eand&\enot&$P$\\
\hline
 T & T &  &  & &  & & \TTbf{F} & F\\
 T & F &   &  &&  & & \TTbf{F} & F\\
 F & T & T &  & F &  & & \TTbf{T} & T\\
 F & F & T &  & F & & & \TTbf{T} & T
\end{tabular}
\end{center}
A similar short cut is available for conditionals. You immediately know that a conditional is true if either its consequent is true, or its antecedent is false. Thus you might present:
\begin{center}
\begin{tabular}{c c|d e e e e e f}
$P$&$Q$& $((P$&\eif&$Q$)&\eif&$P)$&\eif&$P$\\
\hline
 T & T & &  & & & & \TTbf{T} & \\
 T & F &  &  & && & \TTbf{T} & \\
 F & T & & T & & F & & \TTbf{T} & \\
 F & F & & T & & F & &\TTbf{T} & 
\end{tabular}
\end{center}
So `$((P \eif Q) \eif P) \eif P$' is a tautology. In fact, it is an instance of \emph{Peirce's Law}, named after Charles Sanders Peirce.

\section{Testing for validity and entailment}
When we use truth tables to test for validity or entailment, we are checking for \emph{bad} lines: lines where the premises are all true and the conclusion is false. Note:
	\begin{earg}
		\item[\textbullet] Any line where the conclusion is true is not a bad line. 
		\item[\textbullet] Any line where some premise is false is not a bad line. 
	\end{earg}
Since \emph{all} we are doing is looking for bad lines, we should bear this in mind. So: if we find a line where the conclusion is true, we do not need to evaluate anything else on that line: that line definitely isn't bad. Likewise, if we find a line where some premise is false, we do not need to evaluate anything else on that line. 

With this in mind, consider how we might test the following for validity:
	$$\enot L \eif (J \eor L), \enot L \therefore J$$
The \emph{first} thing we should do is evaluate the conclusion. If we find that the conclusion is \emph{true} on some line, then that is not a bad line. So we can simply ignore the rest of the line. So at our first stage, we are left with something like:
\begin{center}
\begin{tabular}{c c|d e e e e f |d f|c}
$J$&$L$&\enot&$L$&\eif&$(J$&\eor&$L)$&\enot&$L$&$J$\\
\hline
%J   L   -   L      ->     (J   v   L)
 T & T & &&&&&&&& {T}\\
 T & F & &&&&&&&& {T}\\
 F & T & &&?&&&&?&& {F}\\
 F & F & &&?&&&&?&& {F}
\end{tabular}
\end{center}
where the blanks indicate that we are not going to bother doing any more investigation (since the line is not bad) and the question-marks indicate that we need to keep investigating. 

The easiest premise to evaluate is the second, so we next do that:
\begin{center}
\begin{tabular}{c c|d e e e e f |d f|c}
$J$&$L$&\enot&$L$&\eif&$(J$&\eor&$L)$&\enot&$L$&$J$\\
\hline
%J   L   -   L      ->     (J   v   L)
 T & T & &&&&&&&& {T}\\
 T & F & &&&&&&&& {T}\\
 F & T & &&&&&&{F}&& {F}\\
 F & F & &&?&&&&{T}&& {F}
\end{tabular}
\end{center}
Note that we no longer need to consider the third line on the table: it will not be a bad line, because (at least) one of premises is false on that line. Finally, we complete the truth table:
\begin{center}
\begin{tabular}{c c|d e e e e f |d f|c}
$J$&$L$&\enot&$L$&\eif&$(J$&\eor&$L)$&\enot&$L$&$J$\\
\hline
%J   L   -   L      ->     (J   v   L)
 T & T & &&&&&&&& {T}\\
 T & F & &&&&&&&& {T}\\
 F & T & &&&&&&{F}& & {F}\\
 F & F & T &  & \TTbf{F} &  & F & & {T} & & {F}
\end{tabular}
\end{center}
The truth table has no bad lines, so the argument is valid. (Any valuation on which all the premises are true is a valuation on which the conclusion is true.)

It might be worth illustrating the tactic again. Let us check whether the following argument is valid
$$A\eor B, \enot (A\eand C), \enot (B \eand \enot D) \therefore (\enot C\eor D)$$
At the first stage, we determine the truth value of the conclusion. Since this is a disjunction, it is true whenever either disjunct is true, so we can speed things along a bit. We can then ignore every line apart from the few lines where the conclusion is false.
\begin{center}
\begin{tabular}[t]{c c c c | c|c|c|d e e f }
$A$ & $B$ & $C$ & $D$ & $A\eor B$ & $\enot (A\eand C)$ & $\enot (B\eand \enot D)$ & $(\enot$ &$C$& $\eor$ & $D)$\\
\hline
T & T & T & T & & & & &  &  \TTbf{T} & \\
T & T & T & F & ? & ? & ? & F & &  \TTbf{F} & \\
T & T & F & T &  & &   & & &  \TTbf{T} & \\
T & T & F & F &  &  &   & T & &  \TTbf{T} &\\
T & F & T & T &  &  &  & & &  \TTbf{T} & \\
T & F & T & F & ? & ? & ?  & F &  &  \TTbf{F} &\\
T & F & F & T & & & & & & \TTbf{T} &\\
T & F & F & F & & & & T &  & \TTbf{T} & \\
F & T & T & T & & & & & & \TTbf{T} & \\
F & T & T & F & ? & ? & ? & F &  & \TTbf{F} &\\
F & T & F & T & & &  & & & \TTbf{T} & \\
F & T & F & F & & & &T & & \TTbf{T} & \\
F & F & T & T & & & & & & \TTbf{T} & \\
F & F & T & F & ? & ? & ? & F & & \TTbf{F} & \\
F & F & F & T & & & & & & \TTbf{T} & \\
F & F & F & F & & & & T& & \TTbf{T} & \\
\end{tabular}
\end{center}
We must now evaluate the premises. We use shortcuts where we can:
\begin{center}
\begin{tabular}[t]{c c c c | d e f |d e e f |d e e e f |d e e f }
$A$ & $B$ & $C$ & $D$ & $A$ & $\eor$ & $B$ & $\enot$ & $(A$ &$\eand$ &$ C)$ & $\enot$ & $(B$ & $\eand$ & $\enot$ & $D)$ & $(\enot$ &$C$& $\eor$ & $D)$\\
\hline
T & T & T & T & & && & && & && & & & &  &  \TTbf{T} & \\
T & T & T & F & &\TTbf{T}& & \TTbf{F}& &T& & & & & & & F & &  \TTbf{F} & \\
T & T & F & T & & && & && & &&  & &   & & &  \TTbf{T} & \\
T & T & F & F & & && & && & &&  &  &   & T & &  \TTbf{T} & \\
T & F & T & T & & && & && & &&  &  &  & & &  \TTbf{T} & \\
T & F & T & F & &\TTbf{T}& &\TTbf{F}& &T& &  && & & & F & & \TTbf{F} & \\
T & F & F & T & & && & && & && & & & & & \TTbf{T} & \\
T & F & F & F & & && & && & && & & & T &  & \TTbf{T} & \\
F & T & T & T& & && & && & & & & & & & & \TTbf{T} & \\
F & T & T & F & &\TTbf{T}& & \TTbf{T}& & F& & \TTbf{F}& & T& T&  & F &  & \TTbf{F} & \\
F & T & F & T & & && & && & && & &  & & & \TTbf{T} & \\
F & T & F & F& & && & && & && & & &T & & \TTbf{T} & \\
F & F & T & T & & && & && & && & & & & & \TTbf{T} & \\
F & F & T & F & & \TTbf{F} & & & & & & &&  &  &  & F & & \TTbf{F} & \\
F & F & F & T & & && & && & && & & & & & \TTbf{T} & \\
F & F & F & F & & && & && & && & & & T& & \TTbf{T} & \\
\end{tabular}
\end{center}
If we had used no shortcuts, we would have had to write 256 `T's or `F's on this table. Using shortcuts, we only had to write 37. We have saved ourselves a \emph{lot} of work.

We have been discussing shortcuts in testing for logically validity, but exactly the same shortcuts can be used in testing for entailment. By employing a similar notion of \emph{bad} lines, you can save yourself a huge amount of work.

\practiceproblems
\problempart
\label{pr.TT.TTorC2}
Using shortcuts, determine whether each sentence is a tautology, a contradiction, or neither. 
\begin{earg}
\item $\enot B \eand B$ %contra
\item $\enot D \eor D$ %taut
\item $(A\eand B) \eor (B\eand A)$ %contingent
\item $\enot[A \eif (B \eif A)]$ %contra
\item $A \eiff [A \eif (B \eand \enot B)]$ %contra
\item $\enot(A\eand B) \eiff A$ %contingent
\item $A\eif(B\eor C)$ %contingent
\item $(A \eand\enot A) \eif (B \eor C)$ %tautology
\item $(B\eand D) \eiff [A \eiff(A \eor C)]$%contingent
\end{earg}


\chapter{Partial truth tables}\label{s:PartialTruthTable}

Sometimes, we do not need to know what happens on every line of a truth table. Sometimes, just a line or two will do. 

\paragraph{Tautology.} 
In order to show that a sentence is a tautology, we need to show that it is true on every valuation. That is to say, we need to know that it comes out true on every line of the truth table. So we need a complete truth table. 

To show that a sentence is \emph{not} a tautology, however, we only need one line: a line on which the sentence is false. Therefore, in order to show that some sentence is not a tautology, it is enough to provide a single valuation---a single line of the truth table---which makes the sentence false. 

Suppose that we want to show that the sentence `$(U \eand T) \eif (S \eand W)$' is \emph{not} a tautology. We set up a \define{partial truth table}:
\begin{center}
\begin{tabular}{c c c c |d e e e e e f}
$S$&$T$&$U$&$W$&$(U$&\eand&$T)$&\eif    &$(S$&\eand&$W)$\\
\hline
   &   &   &   &    &   &    &\TTbf{F}&    &   &   
\end{tabular}
\end{center}
We have only left space for one line, rather than 16, since we are only looking for one line on which the sentence is false. For just that reason, we have filled in `F' for the entire sentence. 

The main logical operator of the sentence is a conditional. In order for the conditional to be false, the antecedent must be true and the consequent must be false. So we fill these in on the table:
\begin{center}
\begin{tabular}{c c c c |d e e e e e f}
$S$&$T$&$U$&$W$&$(U$&\eand&$T)$&\eif    &$(S$&\eand&$W)$\\
\hline
   &   &   &   &    &  T  &    &\TTbf{F}&    &   F &   
\end{tabular}
\end{center}
In order for the `$(U\eand T)$' to be true, both `$U$' and `$T$' must be true.
\begin{center}
\begin{tabular}{c c c c|d e e e e e f}
$S$&$T$&$U$&$W$&$(U$&\eand&$T)$&\eif    &$(S$&\eand&$W)$\\
\hline
   & T & T &   &  T &  T  & T  &\TTbf{F}&    &   F &   
\end{tabular}
\end{center}
Now we just need to make `$(S\eand W)$' false. To do this, we need to make at least one of `$S$' and `$W$' false. We can make both `$S$' and `$W$' false if we want. All that matters is that the whole sentence turns out false on this line. Making an arbitrary decision, we finish the table in this way:
\begin{center}
\begin{tabular}{c c c c|d e e e e e f}
$S$&$T$&$U$&$W$&$(U$&\eand&$T)$&\eif    &$(S$&\eand&$W)$\\
\hline
 F & T & T & F &  T &  T  & T  &\TTbf{F}&  F &   F & F  
\end{tabular}
\end{center}
We now have a partial truth table, which shows that `$(U \eand T) \eif (S \eand W)$' is not a tautology. Put otherwise, we have shown that there is a valuation which makes `$(U \eand T) \eif (S \eand W)$' false, namely, the valuation which makes `$S$' false, `$T$' true, `$U$' true and `$W$' false. 

\paragraph{Contradiction.}
Showing that something is a contradiction requires a complete truth table: we need to show that there is no valuation which makes the sentence true; that is, we need to show that the sentence is false on every line of the truth table. 

However, to show that something is \emph{not} a contradiction, all we need to do is find a valuation which makes the sentence true, and a single line of a truth table will suffice. We can illustrate this with the same example.
\begin{center}
\begin{tabular}{c c c c|d e e e e e f}
$S$&$T$&$U$&$W$&$(U$&\eand&$T)$&\eif    &$(S$&\eand&$W)$\\
\hline
  &  &  &  &   &   &   &\TTbf{T}&  &  &
\end{tabular}
\end{center}
To make the sentence true, it will suffice to ensure that the antecedent is false. Since the antecedent is a conjunction, we can just make one of them false. For no particular reason, we choose to make `$U$' false; and then we can assign whatever truth value we like to the other atomic sentences.
\begin{center}
\begin{tabular}{c c c c|d e e e e e f}
$S$&$T$&$U$&$W$&$(U$&\eand&$T)$&\eif    &$(S$&\eand&$W)$\\
\hline
 F & T & F & F &  F &  F  & T  &\TTbf{T}&  F &   F & F
\end{tabular}
\end{center}

\paragraph{Truth functional equivalence.}
To show that two sentences are logically equivalent, we must show that the sentences have the same truth value on every valuation. So this requires a  complete truth table.

To show that two sentences are \emph{not} logically equivalent, we only need to show that there is a valuation on which they have different truth values. So this requires only a one-line partial truth table: make the table so that one sentence is true and the other false.

\paragraph{Consistency.}
To show that some sentences are jointly consistent, we must show that there is a valuation which makes all of the sentences true,so this requires only a partial truth table with a single line. 

To show that some sentences are jointly inconsistent, we must show that there is no valuation which makes all of the sentence true. So this requires a complete truth table: You must show that on every row of the table at least one of the sentences is false.

\paragraph{Validity.}
To show that an argument is valid, we must show that there is no valuation which makes all of the premises true and the conclusion false. So this  requires a complete truth table.  (Likewise for entailment.)

To show that argument is \emph{invalid}, we must show that there is a valuation which makes all of the premises true and the conclusion false. So this requires only a one-line partial truth table on which all of the premises are true and the conclusion is false. (Likewise for a failure of entailment.)


\
\\This table summarises what is required:

\begin{center}
\begin{tabular}{l l l}
%\cline{2-3}
 & \textbf{Yes} & \textbf{No}\\
 \hline
%\cline{2-3}
tautology? & complete truth table & one-line partial truth table\\
contradiction? &  complete truth table  & one-line partial truth table\\
%contingent? & two-line partial truth table & complete truth table\\
equivalent? & complete truth table & one-line partial truth table\\
consistent? & one-line partial truth table & complete truth table\\
valid? & complete truth table & one-line partial truth table\\
entailment? & complete truth table & one-line partial truth table\\
\end{tabular}
\end{center}
\label{table.CompleteVsPartial}


\practiceproblems
\solutions

\solutions
\problempart
\label{pr.TT.equiv3}
Use complete or partial truth tables (as appropriate) to determine whether these pairs of sentences are logically equivalent:
\begin{earg}
\item $A$, $\enot A$ %No
\item $A$, $A \eor A$ %Yes
\item $A\eif A$, $A \eiff A$ %Yes
\item $A \eor \enot B$, $A\eif B$ %No
\item $A \eand \enot A$, $\enot B \eiff B$ %Yes
\item $\enot(A \eand B)$, $\enot A \eor \enot B$ %Yes
\item $\enot(A \eif B)$, $\enot A \eif \enot B$ %No
\item $(A \eif B)$, $(\enot B \eif \enot A)$ %Yes
\end{earg}

\solutions
\problempart
\label{pr.TT.consistent4}
Use complete or partial truth tables (as appropriate) to determine whether these sentences are jointly consistent, or jointly inconsistent:
\begin{earg}
\item $A \eand B$, $C\eif \enot B$, $C$ %inconsistent
\item $A\eif B$, $B\eif C$, $A$, $\enot C$ %inconsistent
\item $A \eor B$, $B\eor C$, $C\eif \enot A$ %consistent
\item $A$, $B$, $C$, $\enot D$, $\enot E$, $F$ %consistent
\end{earg}

\solutions
\problempart
\label{pr.TT.valid4}
Use complete or partial truth tables (as appropriate) to determine whether each argument is valid or invalid:
\begin{earg}
\item $A\eor\bigl[A\eif(A\eiff A)\bigr] \therefore A$ %invalid
\item $A\eiff\enot(B\eiff A) \therefore A$ %invalid
\item $A\eif B, B \therefore A$ %invalid
\item $A\eor B, B\eor C, \enot B \therefore A \eand C$ %valid
\item $A\eiff B, B\eiff C \therefore A\eiff C$ %valid
\end{earg}

\problempart
\label{pr.TT.TTorC3}
Determine whether each sentence is a tautology, a contradiction, or a contingent sentence. Justify your answer with a complete or partial truth table where appropriate.

% truth tables in LaTeX generated by http://www.curtisbright.com/logic/. Be sure to give him a shout out.

\begin{earg}
\item  $A \eif \enot A$ \vspace{.5ex}							

%{\color{red}
%$
%\begin{array}{c|cccc}
%A&A&\eif&\enot&A\\\hline
%T&T&\mathbf{F}&F&T\\
%F&F&\mathbf{T}&T&F
%\end{array}
%$ 
%
%Contingent	 \vspace{6pt}
%}
%	T letter, 2 connectives
\item $A \eif (A \eand (A \eor B))$ \vspace{.5ex}	

%{\color{red}
%$
%\begin{array}{cc|ccc@{}ccc@{}ccc@{}c@{}c}
%A&B&A&\eif&(&A&\eand&(&A&\eor&B&)&)\\\hline
%T&T&T&\mathbf{T}&&T&T&&T&T&T&&\\
%T&F&T&\mathbf{T}&&T&T&&T&T&F&&\\
%F&T&F&\mathbf{T}&&F&F&&F&T&T&&\\
%F&F&F&\mathbf{T}&&F&F&&F&F&F&&
%\end{array}
%$
%
%Tautology \vspace{6pt}
%}
%			2 letters, 3 connectives

\item $(A \eif B) \eiff (B \eif A)$ 	\vspace{.5ex}				%
%
%{\color{red}
%$
%\begin{array}{cc|c@{}ccc@{}ccc@{}ccc@{}c}
%a&b&(&a&\rightarrow&b&)&\leftrightarrow&(&b&\rightarrow&a&)\\\hline
%T&T&&T&T&T&&\mathbf{T}&&T&T&T&\\
%T&F&&T&F&F&&\mathbf{F}&&F&T&T&\\
%F&T&&F&T&T&&\mathbf{F}&&T&F&F&\\
%F&F&&F&T&F&&\mathbf{T}&&F&T&F&
%\end{array}
%$
%
%Contingent \vspace{6pt}
%
%}
%		2 letters, 3 connectives

\item $A \eif \enot(A \eand (A \eor B)) $	\vspace{.5ex}	

%{\color{red}
%$
%\begin{array}{cc|cccc@{}ccc@{}ccc@{}c@{}c}
%a&b&a&\rightarrow&\enot&(&a&\eand&(&a&\eor&b&)&)\\\hline
%T&T&T&\mathbf{F}&F&&T&T&&T&T&T&&\\
%T&F&T&\mathbf{F}&F&&T&T&&T&T&F&&\\
%F&T&F&\mathbf{T}&T&&F&F&&F&T&T&&\\
%F&F&F&\mathbf{T}&T&&F&F&&F&F&F&&
%\end{array}
%$
%
%Contingent	\vspace{6pt}
%
%}
%
% 2 letters, 4 connectives

\item $\enot B \eif [(\enot A \eand A) \eor B]$\vspace{.5ex} 

%{\color{red}
%$
%\begin{array}{cc|cccc@{}c@{}cccc@{}ccc@{}c}
%a&b&\enot&b&\rightarrow&(&(&\enot&a&\eand&a&)&\eor&b&)\\\hline
%T&T&F&T&\mathbf{T}&&&F&T&F&T&&T&T&\\
%T&F&T&F&\mathbf{F}&&&F&T&F&T&&F&F&\\
%F&T&F&T&\mathbf{T}&&&T&F&F&F&&T&T&\\
%F&F&T&F&\mathbf{F}&&&T&F&F&F&&F&F&
%\end{array}
%$
%Contingent	 \vspace{6pt}
%
%}
%	2 letters, 5 connectives

\item $\enot(A \eor B) \eiff (\enot A \eand \enot B)$ \vspace{.5ex}

%{\color{red}
%$
%\begin{array}{cc|cc@{}ccc@{}ccc@{}ccccc@{}c}
%a&b&\enot&(&a&\eor&b&)&\leftrightarrow&(&\enot&a&\eand&\enot&b&)\\\hline
%T&T&F&&T&T&T&&\mathbf{T}&&F&T&F&F&T&\\
%T&F&F&&T&T&F&&\mathbf{T}&&F&T&F&T&F&\\
%F&T&F&&F&T&T&&\mathbf{T}&&T&F&F&F&T&\\
%F&F&T&&F&F&F&&\mathbf{T}&&T&F&T&T&F&
%\end{array}
%$
%
%Tautology \vspace{6pt}
%}
%2 letters, 6 connectives

\item $[(A \eand B) \eand C] \eif B$\vspace{.5ex}							
%
%{\color{red}
%$
%\begin{array}{ccc|c@{}c@{}ccc@{}ccc@{}ccc}
%a&b&c&(&(&a&\eand&b&)&\eand&c&)&\rightarrow&b\\\hline
%T&T&T&&&T&T&T&&T&T&&\mathbf{T}&T\\
%T&T&F&&&T&T&T&&F&F&&\mathbf{T}&T\\
%T&F&T&&&T&F&F&&F&T&&\mathbf{T}&F\\
%T&F&F&&&T&F&F&&F&F&&\mathbf{T}&F\\
%F&T&T&&&F&F&T&&F&T&&\mathbf{T}&T\\
%F&T&F&&&F&F&T&&F&F&&\mathbf{T}&T\\
%F&F&T&&&F&F&F&&F&T&&\mathbf{T}&F\\
%F&F&F&&&F&F&F&&F&F&&\mathbf{T}&F
%\end{array}
%$
%
%Tautology \vspace{6pt}
%}
%
%3 letters, 3 connectives

\item $\enot\bigl[(C\eor A) \eor B\bigr]$\vspace{.5ex} 						
%
%{\color{red}
%$
%\begin{array}{ccc|cc@{}c@{}ccc@{}ccc@{}c}
%a&b&c&\enot&(&(&c&\eor&a&)&\eor&b&)\\\hline
%T&T&T&\mathbf{F}&&&T&T&T&&T&T&\\
%T&T&F&\mathbf{F}&&&F&T&T&&T&T&\\
%T&F&T&\mathbf{F}&&&T&T&T&&T&F&\\
%T&F&F&\mathbf{F}&&&F&T&T&&T&F&\\
%F&T&T&\mathbf{F}&&&T&T&F&&T&T&\\
%F&T&F&\mathbf{F}&&&F&F&F&&T&T&\\
%F&F&T&\mathbf{F}&&&T&T&F&&T&F&\\
%F&F&F&\mathbf{T}&&&F&F&F&&F&F&
%\end{array}
%$
%
%Contingent \vspace{6pt}
%
%}
%	 	3 letters, 3 connectives

\item $\bigl[(A\eand B) \eand\enot(A\eand B)\bigr] \eand C$ \vspace{.5ex}	
%
%{\color{red}
%$
%\begin{array}{ccc|c@{}c@{}ccc@{}cccc@{}ccc@{}c@{}ccc}
%a&b&c&(&(&a&\eand&b&)&\eand&\enot&(&a&\eand&b&)&)&\eand&c\\\hline
%T&T&T&&&T&T&T&&F&F&&T&T&T&&&\mathbf{F}&T\\
%T&T&F&&&T&T&T&&F&F&&T&T&T&&&\mathbf{F}&F\\
%T&F&T&&&T&F&F&&F&T&&T&F&F&&&\mathbf{F}&T\\
%T&F&F&&&T&F&F&&F&T&&T&F&F&&&\mathbf{F}&F\\
%F&T&T&&&F&F&T&&F&T&&F&F&T&&&\mathbf{F}&T\\
%F&T&F&&&F&F&T&&F&T&&F&F&T&&&\mathbf{F}&F\\
%F&F&T&&&F&F&F&&F&T&&F&F&F&&&\mathbf{F}&T\\
%F&F&F&&&F&F&F&&F&T&&F&F&F&&&\mathbf{F}&F
%\end{array}
%$
%
%Contradiction \vspace{6pt}
%
%}
%
%% 	3 letters, 5 connectives
%
\item $(A \eand B) ]\eif[(A \eand C) \eor (B \eand D)]$ \vspace{.5ex}		
%
%{\color{red}
%$
%\begin{array}{cccc|c@{}c@{}ccc@{}c@{}ccc@{}c@{}ccc@{}ccc@{}ccc@{}c@{}c}
%a&b&c&d&(&(&a&\eand&b&)&)&\eif&(&(&a&\eand&c&)&\eor&(&b&\eand&d&)&)\\\hline
%T&T&T&T&&&T&T&T&&&\mathbf{T}&&&T&T&T&&T&&T&T&T&&\\
%T&T&F&F&&&T&T&T&&&\mathbf{F}&&&T&F&F&&F&&T&F&F&&\\
%\end{array}
%$
%
%Contingent \vspace{6pt}
%}
%
%	4 letters, 5 connectives
\end{earg}

\noindent\problempart
\label{pr.TT.TTorC4}
Determine whether each sentence is a tautology, a contradiction, or a contingent sentence. Justify your answer with a complete or partial truth table where appropriate.
\begin{earg}
\item  $\enot (A \eor A)$\vspace{.5ex}							%	Contradiction		1 letter, 2 connectives
\item $(A \eif B) \eor (B \eif A)$\vspace{.5ex}					%	Tautology			2 letters, 2 connectives
\item $[(A \eif B) \eif A] \eif A$\vspace{.5ex}					%	Tautology			2 letters, 3 connectives
\item $\enot[( A \eif B) \eor (B \eif A)]$\vspace{.5ex}			%	Contradiction		2 letters, 4 connectives
\item $(A \eand B) \eor (A \eor B)$\vspace{.5ex} 				%	Contingent		2 letters, 5 connectives
\item $\enot(A\eand B) \eiff A$\vspace{.5ex} 					%contingent			2 letters, 3 connectives
\item $A\eif(B\eor C)$\vspace{.5ex} 							%contingent			3 letters, 2 connectives
\item $(A \eand\enot A) \eif (B \eor C)$\vspace{.5ex} 			%tautology			3 letters, 4 connectives 
\item $(B\eand D) \eiff [A \eiff(A \eor C)]$\vspace{.5ex}			%contingent			4 letters, 4 connectives
\item $\enot[(A \eif B) \eor (C \eif D)]$\vspace{.5ex} 			% Contingent. 		4 letters, 4 connectives
\end{earg}



\noindent\problempart
Determine whether each the following pairs of sentences are logically equivalent using complete truth tables. If the two sentences really are logically equivalent, write ``equivalent.'' Otherwise write, ``not equivalent.''
\begin{earg}
\item $A$ and $A \eor A$
\item $A$ and $A \eand A$
\item $A \eor \enot B$ and $A\eif B$
\item $(A \eif B)$ and $(\enot B \eif \enot A)$
\item $\enot(A \eand B)$ and $\enot A \eor \enot B$
\item $ ((U \eif (X \eor X)) \eor U)$ and $\enot (X \eand (X \eand U))$
\item $ ((C \eand (N \eiff C)) \eiff C)$ and $(\enot \enot \enot N \eif C)$
\item $[(A \eor B) \eand C]$ and $[A \eor (B \eand C)]$
\item $((L \eand C) \eand I)$ and $L \eor C$
\end{earg}


\noindent\problempart
\label{pr.TT.consistent5}
Determine whether each collection of sentences is consistent or inconsistent. Justify your answer with a complete or partial truth table where appropriate.
\begin{earg}
\item $A\eif A$, $\enot A \eif \enot A$, $A\eand A$, $A\eor A$ \vspace{.5ex}%consistent
\item $A \eif \enot A$, $\enot A \eif A$\vspace{.5ex}%inconsistent. 
\item $A\eor B$, $A\eif C$, $B\eif C$\vspace{.5ex} %consistent
\item $A \eor B$, $A \eif C$, $B \eif C$, $\enot C$\vspace{.5ex} %	Inconsistent
\item $B\eand(C\eor A)$, $A\eif B$, $\enot(B\eor C)$\vspace{.5ex}  %inconsistent
\item $(A \eiff B) \eif B$,  $B \eif \enot (A \eiff B)$, $A \eor B$ \vspace{.5ex} %	Consistent
\item $A\eiff(B\eor C)$, $C\eif \enot A$, $A\eif \enot B$\vspace{.5ex} %consistent
\item  $A \eiff B$,  $\enot B \eor \enot A$,  $A \eif  B$ \vspace{.5ex}% Consistent
\item $A \eiff B$, $A \eif C$, $B \eif D$, $\enot(C \eor D)$\vspace{.5ex} %consitent
\item $\enot (A \eand \enot B)$,  $B \eif \enot A$, $\enot B$  \vspace{.5ex} %Consistent

\end{earg}

\noindent\problempart
\label{pr.TT.consistent6}
Determine whether each collection of sentences is consistent or inconsistent. Justify your answer with a complete or partial truth table where appropriate.
\begin{earg}
\item $A \eand B$, $C\eif \enot B$, $C$ \vspace{.5ex}%inconsistent
\item $A\eif B$, $B\eif C$, $A$, $\enot C$\vspace{.5ex} %inconsistent
\item $A \eor B$, $B\eor C$, $C\eif \enot A$\vspace{.5ex} %consistent
\item $A$, $B$, $C$, $\enot D$, $\enot E$, $F$\vspace{.5ex} %consistent
\item $A \eand (B \eor C)$, $\enot(A \eand C)$, $\enot(B \eand C)$ \vspace{.5ex}%consistent
\item $A \eif B$, $B \eif C$, $\enot(A \eif C)$ \vspace{.5ex} %inconsistent

%\begin{tabular}{ccc|ccc|ccccc}
%A 	&\eif	&B 	&B 	&\eif 	&C 	&\enot 	&(A 	&\eif 	&C) 	&Inconsistent\\
%\cline{1-10} 
%T 	&T 	&T 	&T 	&T 	&T 	&F 		&T 	&T 	&T 	& \\
%T&T&T&T&F&F&T&T&F&F&\\
%T&F&F&F&T&T&F&T&T&T&\\
%T&F&F&F&T&F&T&T&F&F&\\
%F&T&T&T&T&T&F&F&T&T&\\
%F&T&T&T&F&F&F&F&T&F&\\
%F&T&F&F&T&T&F&F&T&T&\\
%F&T&F&F&T&F&F&F&T&F&\\
%\end{tabular}

\end{earg}


\noindent\problempart Determine whether each argument is valid or invalid. Justify your answer with a complete or partial truth table where appropriate.
\label{pr.TT.valid5} 
\begin{enumerate}
\item $A\eif(A\eand\enot A)\therefore \enot A$% valid
\item $A \eor B$, $A \eif B$, $B \eif A \therefore  A \eiff B$  % Valid
\item $A\eor(B\eif A)\therefore \enot A \eif \enot B$ %valid
\item $A \eor B$, $A \eif B$, $ B \eif A \therefore  A \eand B$ %valid
\item $(B\eand A)\eif C$, $(C\eand A)\eif B\therefore (C\eand B)\eif A$ % invalid
\item $\enot (\enot A \eor \enot B)$, $A \eif \enot C \therefore  A \eif (B \eif C)$ % invalid.
\item $A \eand (B \eif C)$, $\enot C \eand (\enot B \eif \enot A)\therefore C \eand \enot C$ % valid
\item $A \eand B$, $\enot A \eif \enot C$, $B \eif \enot D \therefore  A \eor B$ % Invalid
\item $A \eif B\therefore (A \eand B) \eor (\enot A \eand \enot B)$ % invalid
\item $\enot A \eif B$,$ \enot B \eif C $,$ \enot C \eif A \therefore  \enot A \eif (\enot B \eor \enot C) $% Invalid

\end{enumerate}

\noindent\problempart Determine whether each argument is valid or invalid. Justify your answer with a complete or partial truth table where appropriate.
\label{pr.TT.valid6} 
\begin{enumerate}
\item $A\eiff\enot(B\eiff A)\therefore A$ % invalid
\item $A\eor B$, $B\eor C$, $\enot A\therefore B \eand C$ % invalid
\item $A \eif C$, $E \eif (D \eor B)$, $B \eif \enot D\therefore (A \eor C) \eor (B \eif (E \eand D))$ % invalid
\item $A \eor B$, $C \eif A$, $C \eif B\therefore A \eif (B \eif C)$ % invalid
\item $A \eif B$, $\enot B \eor A\therefore A \eiff B$ % valid
\end{enumerate}

\noindent\problempart
\label{pr.TT.concepts2}
Answer each of the questions below and justify your answer.
\begin{enumerate}
\item Suppose that \script{A} and \script{B} are logically equivalent. What can you say about $\script{A}\eiff\script{B}$?
%\script{A} and \script{B} have the same truth value on every line of a complete truth table, so $\script{A}\eiff\script{B}$ is true on every line. It is a tautology.
\item Suppose that $(\script{A}\eand\script{B})\eif\script{C}$ is contingent. What can you say about the argument ``\script{A}, \script{B}, $\therefore$\ \script{C}''?
%The sentence is false on some line of a complete truth table. On that line, \script{A} and \script{B} are true and \script{C} is false. So the argument is invalid.
\item Suppose that $\script{A},\script{B}, \script{C}$ are inconsistent. What can you say about $(\script{A}\eand\script{B}\eand\script{C})$?
%Since there is no line of a complete truth table on which all three sentences are true, the conjunction is false on every line. So it is a contradiction.
\item Suppose that \script{A} is a contradiction. What can you say about the argument \script{A}, \script{B} $\therefore $  \script{C}?
%Since \script{A} is false on every line of a complete truth table, there is no line on which \script{A} and \script{B} are true and \script{C} is false. So the argument is valid.
\item Suppose that \script{C} is a tautology. What can you say about the argument \script{A}, \script{B} $\therefore $ \script{C}''?
%Since \script{C} is true on every line of a complete truth table, there is no line on which \script{A} and \script{B} are true and \script{C} is false. So the argument is valid.
%\item Suppose that \script{A} and \script{B} are logically equivalent. What can you say about $(\script{A}\eor\script{B})$?
%Not much. $(\script{A}\eor\script{B})$ is a tautology if \script{A} and \script{B} are tautologies; it is a contradiction if they are contradictions; it is contingent if they are contingent.
\item Suppose that \script{A} and \script{B} are \emph{not} logically equivalent. What can you say about $(\script{A}\eor\script{B})$?
%\script{A} and \script{B} have different truth values on at least one line of a complete truth table, and $(\script{A}\eor\script{B})$ will be true on that line. On other lines, it might be true or false. So $(\script{A}\eor\script{B})$ is either a tautology or it is contingent; it is \emph{not} a contradiction.
\end{enumerate}
