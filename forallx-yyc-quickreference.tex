%!TEX root = forallxyyc.tex
\chapter[Quick reference]{Quick reference}
%\pagestyle{plain}
\section{Characteristic truth tables}
\label{app.CharacteristicTTs}

\begin{tabular}{c|c}
\meta{A} & \enot\meta{A}\\
\hline
T & F\\
F & T \\
\phantom{.}\\
\phantom{.}
\end{tabular}
\hfill
\begin{tabular}{c c|c|c|c|c}
\meta{A} & \meta{B} & $\meta{A}\eand\meta{B}$ & $\meta{A}\eor\meta{B}$ & $\meta{A}\eif\meta{B}$ & $\meta{A}\eiff\meta{B}$\\
\hline
T & T & T & T & T & T\\
T & F & F & T & F & F\\
F & T & F & T & T & F\\
F & F & F & F & T & T
\end{tabular}


\vfill

\section{Symbolization}
\begin{center}
\label{app.symbolization}
\begin{tabular*}{\textwidth}{rl}
\multicolumn{2}{c}{\textsc{Sentential Connectives}}\\ \\
It is not the case that $P$ & $\enot P$\\
Either $P$ or $Q$ & $(P \eor Q)$\\
Neither $P$ nor $Q$ & $\enot(P \eor Q)$\ or \ $(\enot P \eand \enot Q)$\\
Both $P$ and $Q$ & $(P \eand Q)$\\
If $P$ then $Q$ & $(P \eif Q)$\\
$P$ only if $Q$ & $(P \eif Q)$\\
$P$ if and only if $Q$ & $(P \eiff Q)$\\
$P$ unless $Q$ & $(P \eor Q)$\\
\\
\multicolumn{2}{c}{\label{SymbolizingPredicates}\textsc{Predicates}}\\ \\
All $F$s are $G$s & $\forall x(\atom{F}{x} \eif \atom{G}{x})$\\
Some $F$s are $G$s & $\exists x(\atom{F}{x} \eand \atom{G}{x})$\\
Not all $F$s are $G$s & $\enot\forall x(\atom{F}{x} \eif \atom{G}{x})$\ or\\
& $\exists x(\atom{F}{x} \eand \enot \atom{G}{x})$\\
No $F$s are $G$s & $\forall x(\atom{F}{x} \eif\enot \atom{G}{x})$\ or\\
& $\enot\exists x(\atom{F}{x} \eand \atom{G}{x})$\\
\\
\multicolumn{2}{c}{\textsc{Identity}}\\ \\
Only $c$ is $G$ & $\forall x(\atom{G}{x} \eiff x=c)$\\
Everything besides $c$ is $G$ & $\forall x(\enot x = c \eif \atom{G}{x} )$\\
%$j$ is more $R$ than anyone else. & $\forall x(x\neq j \eif Rjx)$\\
The $F$ is $G$ & $\exists x(\atom{F}{x} \eand \forall y(\atom{F}{y} \eif x=y) \eand \atom{G}{x} )$\\
It is not the case that\\
 the $F$ is $G$ & $\enot\exists x(\atom{F}{x} \eand \forall y(\atom{F}{y} \eif x=y) \eand \atom{G}{x} )$\\
The $F$ is non-$G$ & $\exists x(\atom{F}{x} \eand \forall y(\atom{F}{y} \eif x=y) \eand \enot \atom{G}{x} )$
\end{tabular*}
\end{center}






% BEGIN: symbolizing cardinality

\newpage
\section{Using identity to symbolize quantities}

\subsection*{There are at least \blank\ $F$s.}
\label{summary.atleast}

\begin{tabular*}{\textwidth}{rl}
one & $\exists x\,\atom{F}{x}$\\
two & $\exists x_1\exists x_2(\atom{F}{x_1} \eand \atom{F}{x_2} \eand \enot x_1  = x_2)$\\
three & $\exists x_1\exists x_2\exists x_3(\atom{F}{x_1} \eand \atom{F}{x_2} \eand \atom{F}{x_3} \eand {}$\\
& $\enot x_1 = x_2 \eand\enot x_1 = x_3 \eand \enot x_2 = x_3)$\\
four & $\exists x_1\exists x_2\exists x_3\exists x_4 (\atom{F}{x_1} \eand \atom{F}{x_2} \eand \atom{F}{x_3} \eand \atom{F}{x_4} \eand {}$\\
& $\enot x_1 = x_2 \eand \enot x_1 = x_3 \eand \enot x_1 = x_4 \eand {}$\\
& $ \enot x_2 = x_3 \eand \enot x_2 = x_4 \eand \enot x_3 = x_4)$\\
$n$ & $\exists x_1\ldots\exists x_n(\atom{F}{x_1} \eand \ldots \eand \atom{F}{x_n} \eand {}$\\
& $\enot x_1 = x_2 \eand\ldots\eand \enot x_{n-1} = x_n)$ 
\end{tabular*}

\subsection*{There are at most \blank\ $F$s.}
\label{summary.atmost}

One way to say `there are at most $n$ $F$s' is to put a negation sign in front of the symbolization for `there are at least $n+1$ $F$s'. Equivalently, we can offer:
\begin{tabular*}{\textwidth}{rl}
one & $\forall x_1\forall x_2\bigl[(\atom{F}{x_1} \eand \atom{F}{x_2}) \eif x_1=x_2\bigr]$\\
two & $\forall x_1\forall x_2\forall x_3\bigl[(\atom{F}{x_1} \eand \atom{F}{x_2} \eand \atom{F}{x_3}) \eif {}$\\ & $(x_1=x_2 \eor x_1=x_3 \eor x_2=x_3)\bigr]$\\
three & $\forall x_1\forall x_2\forall x_3\forall x_4\bigl[(\atom{F}{x_1} \eand \atom{F}{x_2} \eand \atom{F}{x_3} \eand \atom{F}{x_4}) \eif {}$\\
& $(x_1=x_2 \eor x_1=x_3 \eor x_1=x_4 \eor {}$\\
& $x_2=x_3 \eor x_2=x_4 \eor x_3=x_4)\bigr]$\\
$n$ & $\forall x_1\ldots\forall x_{n+1}
\bigl[(\atom{F}{x_1} \eand \ldots \eand \atom{F}{x_{n+1}}) \eif {}$\\
& $(x_1=x_2 \eor \ldots \eor x_n=x_{n+1})\bigr]$ 
\end{tabular*}


\subsection*{There are exactly \blank\ $F$s.}
\label{summary.exactly}

One way to say `there are exactly $n$ $F$s' is to conjoin two of the symbolizations above and say `there are at least $n$ $F$s and there are at most $n$ $F$s.' The following equivalent formulas are shorter:
\begin{tabular*}{\textwidth}{rl}
zero & $\forall x\,\enot \atom{F}{x}$\\
one & $\exists x\bigl[\atom{F}{x} \eand \forall y(\atom{F}{y} \eif x = y)\bigr]$\\
two & $\exists x_1\exists x_2\bigl[\atom{F}{x_1} \eand \atom{F}{x_2} \eand {}$\\
& $\enot x_1 = x_2 \eand \forall y\bigl(\atom{F}{y} \eif (y= x_1 \eor y = x_2)\bigr) \bigr]$\\
three & $\exists x_1\exists x_2\exists x_3\bigl[\atom{F}{x_1} \eand \atom{F}{x_2} \eand \atom{F}{x_3} \eand {}$\\
& $\enot x_1 =  x_2 \eand \enot  x_1 = x_3 \eand \enot x_2 = x_3 \eand {}$\\
& $\forall y\bigl(\atom{F}{y} \eif (y = x_1 \eor y = x_2 \eor y =  x_3)\bigr) \bigr]$\\
$n$ & $\exists x_1\ldots\exists x_n\bigl[\atom{F}{x_1} \eand\ldots\eand \atom{F}{x_n}  \eand {}$\\
&$ \enot x_1 = x_2 \eand\ldots\eand \enot x_{n-1}= x_n \eand \phantom{.}$\\
& $\forall y\bigl(\atom{F}{y} \eif (y= x_1 \eor \ldots \eor y= x_n)\bigr)\bigr]$ 
%\item[one] $\exists x\forall y\bigl[\atom{F}{x} \eand (\atom{F}{y} \eif y = x)\bigr]$
%\item[two] $\exists x\exists y\forall z\Bigl(\atom{F}{x} \eand \atom{F}{y} \eand \bigl[\atom{F}{z} \eif (z=x \eor z=y)\bigr] \eand x \neq y\Bigr)$
%\item[three] $\exists x_1\exists x_2\exists x_3\forall y\Bigl(\atom{F}{x_1} \eand \atom{F}{x_2} \eand \atom{F}{x_3} \eand [\atom{F}{y} \eif (y=x_1 \eor y=x_2 \eor y=x_3)] \eand x_1 \neq x_2 \eand x_1 \neq x_3 \eand x_2 \neq x_3\Bigr)$
%\item[n] $\exists x_1\cdots\exists x_n\forall y\Bigl(\atom{F}{x_1} \eand \cdots \eand \atom{F}{x_n} \eand \bigl[\atom{F}{y} \eif (y=x_1 \eor \cdots \eor y=x_n)\bigr] \eand x_1 \neq x_2 \eand\cdots\eand x_{n-1}\neq x_n\Bigr)$ 
\end{tabular*}


\label{ProofRules}
\newpage\section{Basic deduction rules for TFL}
\renewenvironment{proof}
	{\noindent\par\noindent\small$\begin{nd}}
	{\end{nd}$\noindent\normalsize\ignorespacesafterend}

%{\LARGE \textbf{Basic Rules of Proof}}
\begin{multicols}{2}
\subsection*{Reiteration}

\begin{proof}
	\have[m]{a}{\meta{A}}
	\have[\ ]{c}{\meta{A}} \by{R}{a}
\end{proof}

\subsection*{Conjunction}

\begin{proof}
	\have[m]{a}{\meta{A}}
	\have[n]{b}{\meta{B}}
	\have[\ ]{c}{\meta{A}\eand\meta{B}} \ai{a, b}

	\have[m]{ab}{\meta{A}\eand\meta{B}}
\\	\have[\ ]{a}{\meta{A}} \ae{ab}

	\have[m]{ab}{\meta{A}\eand\meta{B}}
\\	\have[\ ]{b}{\meta{B}} \ae{ab}
\end{proof}

\subsection*{Conditional}

\begin{proof}
	\open
		\hypo[i]{a}{\meta{A}}
		\have[j]{b}{\meta{B}}
	\close
	\have[\ ]{ab}{\meta{A}\eif\meta{B}}\ci{a-b}

	\have[m]{ab}{\meta{A}\eif\meta{B}}
\\	\have[n]{a}{\meta{A}}
	\have[\ ]{b}{\meta{B}} \ce{ab,a}
\end{proof}

\subsection*{Negation}

\begin{proof}
\open
	\hypo[i]{a}{\meta{A}}
	\have[j]{nb}{\ered}
\close
\have[\ ]{na}{\enot\meta{A}}\ni{a-nb}

\have[m]{na}{\enot\meta{A}}
\\ \have[n]{a}{\meta{A}}
\have[ ]{bot}{\ered}\ri{na, a}
\end{proof}

\subsection*{Indirect proof}

\begin{proof}
\open
	\hypo[i]{a}{\enot\meta{A}}
	\have[j]{nb}{\ered}
\close
\have[\ ]{na}{\meta{A}}\ip{a-nb}
\end{proof}


\subsection*{Explosion}

\begin{proof}
\have[m]{bot}{\ered}
\\\have[ ]{}{\meta{A}}\re{bot}
\end{proof}

\subsection*{Disjunction}

\begin{proof}
	\have[m]{a}{\meta{A}}
	\have[\ ]{ab}{\meta{A}\eor\meta{B}}\oi{a}

	\have[m]{a}{\meta{A}}
\\	\have[\ ]{ba}{\meta{B}\eor\meta{A}}\oi{a}

	\have[m]{ab}{\meta{A}\eor\meta{B}}
\\	\open
		\hypo[i]{a}{\meta{A}}
		\have[j]{c1}{\meta{C}}
	\close
	\open
		\hypo[k]{b}{\meta{B}}
		\have[l]{c2}{\meta{C}}
	\close
	\have[\ ]{c}{\meta{C}} \oe{ab,a-c1, b-c2}
\end{proof}

\subsection*{Biconditional}

\begin{proof}
	\open
		\hypo[i]{a1}{\meta{A}} 
		\have[j]{b1}{\meta{B}}
	\close
	\open
		\hypo[k]{b2}{\meta{B}}
		\have[l]{a2}{\meta{A}}
	\close
	\have[\ ]{ab}{\meta{A}\eiff\meta{B}}\bi{a1-b1,b2-a2}

	\have[m]{ab}{\meta{A}\eiff\meta{B}}
\\	\have[n]{a}{\meta{A}}
	\have[\ ]{b}{\meta{B}} \be{ab,a}

	\have[m]{ab}{\meta{A}\eiff\meta{B}}
\\	\have[n]{a}{\meta{B}}
	\have[\ ]{b}{\meta{A}} \be{ab,a}
\end{proof}

\end{multicols}

\newpage
\section{Derived rules for TFL}
\begin{multicols}{2}
\subsection*{Disjunctive syllogism}
\begin{proof}
	\have[m]{ab}{\meta{A} \eor \meta{B}}
	\have[n]{nb}{\enot \meta{A}}
	\have[\ ]{con}{\meta{B}}\by{DS}{ab, nb}

	\have[m]{ab}{\meta{A} \eor \meta{B}}
\\	\have[n]{nb}{\enot \meta{B}}
	\have[\ ]{con}{\meta{A}}\by{DS}{ab, nb}
\end{proof}

\subsection*{Modus Tollens}

\begin{proof}
	\have[m]{ab}{\meta{A}\eif\meta{B}}
	\have[n]{a}{\enot\meta{B}}
	\have[\ ]{b}{\enot\meta{A}} \by{MT}{ab,a}
\end{proof}

\subsection*{Double-negation elimination}
	\begin{proof}
		\have[m]{dna}{\enot \enot \meta{A}}
		\have[ ]{a}{\meta{A}}\dne{dna}
	\end{proof}


\subsection*{Excluded middle}
	\begin{proof}
		\open
			\hypo[i]{a}{\meta{A}}
			\have[j]{c1}{\meta{B}}
		\close
		\open
			\hypo[k]{b}{\enot\meta{A}}
			\have[l]{c2}{\meta{B}}
		\close
		\have[\ ]{ab}{\meta{B}}\tnd{a-c1,b-c2}
	\end{proof}

%
%\subsection*{Hypothetical Syllogism}
%
%\begin{proof}
%	\have[m]{ab}{\meta{A}\eif\meta{B}}
%	\have[n]{bc}{\meta{B}\eif\meta{C}}
%	\have[\ ]{ac}{\meta{A}\eif\meta{C}}\by{HS}{ab,bc}
%\end{proof}

\subsection*{De Morgan Rules}
\begin{proof}
	\have[m]{ab}{\enot (\meta{A} \eor \meta{B})}
	\have[\ ]{dm}{\enot \meta{A} \eand \enot \meta{B}}\dem{ab}

	\have[m]{ab}{\enot \meta{A} \eand \enot \meta{B}}
\\	\have[\ ]{dm}{\enot (\meta{A} \eor \meta{B})}\dem{ab}

	\have[m]{ab}{\enot (\meta{A} \eand \meta{B})}
\\	\have[\ ]{dm}{\enot \meta{A} \eor \enot \meta{B}}\dem{ab}

	\have[m]{ab}{\enot \meta{A} \eor \enot \meta{B}}
\\	\have[\ ]{dm}{\enot (\meta{A} \eand \meta{B})}\dem{ab}
\end{proof}
\end{multicols}

\newpage

\section{Basic deduction rules for FOL}

\begin{multicols}{2}
\subsection*{Universal elimination}

\begin{proof}
	\have[m]{a}{\forall \meta{x}\meta{A}(\ldots \meta{x} \ldots \meta{x}\ldots)}
	\have[\ ]{c}{\meta{A}(\ldots \meta{c} \ldots \meta{c}\ldots)} \Ae{a}
\end{proof}

\subsection*{Universal introduction}

\begin{proof}
	\have[m]{a}{\meta{A}(\ldots \meta{c} \ldots \meta{c}\ldots)}
	\have[\ ]{c}{\forall \meta{x}\meta{A}(\ldots \meta{x} \ldots \meta{x}\ldots)} \Ai{a}
\end{proof}

\medskip\begin{raggedright}
\meta{c} must not occur in any undischarged assumption

\meta{x} must not occur in\\ $\meta{A}(\ldots \meta{c} \ldots \meta{c}\ldots)$
\end{raggedright}

\subsection*{Existential introduction}

\begin{proof}
	\have[m]{a}{\meta{A}(\ldots \meta{c} \ldots \meta{c}\ldots)}
	\have[\ ]{c}{\exists \meta{x}\meta{A}(\ldots \meta{x} \ldots \meta{c}\ldots)}\Ei{a}
\end{proof}

\medskip\begin{raggedright}
\noindent \meta{x} must not occur in\\ $\meta{A}(\ldots \meta{c} \ldots \meta{c}\ldots)$
\end{raggedright}
%\noindent You can replace one or more instance of \meta{c} with \meta{x}.

\subsection*{Existential elimination}

\begin{proof}
	\have[m]{a}{\exists \meta{x}\meta{A}(\ldots \meta{x} \ldots \meta{x}\ldots)}
	\open	
		\hypo[i]{b}{\meta{A}(\ldots \meta{c} \ldots \meta{c}\ldots)}
		\have[j]{c}{\meta{B}}
	\close
	\have[\ ]{d}{\meta{B}}\Ee{a,b-c}
\end{proof}

\medskip\begin{raggedright}
\noindent \meta{c} must not occur in any undischarged assumption, in $\exists \meta{x}\meta{A}(\ldots \meta{x} \ldots \meta{x}\ldots)$, or in \meta{B}\end{raggedright}\vfill\columnbreak

\end{multicols}

\subsection*{Identity introduction}

\begin{proof}
	\have[\ \,\,\,]{x}{\meta{c}=\meta{c}} \by{=I}{}
\end{proof}


\subsection*{Identity elimination}

\begin{multicols}{2}
\begin{proof}
	\have[m]{e}{\meta{a}=\meta{b}}
	\have[n]{a}{\meta{A}(\ldots \meta{a} \ldots \meta{a}\ldots)}
	\have[\ ]{ea1}{\meta{A}(\ldots \meta{b} \ldots \meta{a}\ldots)} \by{=E}{e,a}
\end{proof}
\begin{proof}
	\have[m]{e}{\meta{a}=\meta{b}}
	\have[n]{a}{\meta{A}(\ldots \meta{b} \ldots \meta{b}\ldots)}
	\have[\ ]{ea2}{\meta{A}(\ldots \meta{a} \ldots \meta{b}\ldots)} \by{=E}{e,a}
\end{proof}
\end{multicols}

\begin{minipage}{\textwidth} % hack to keep section header with table
\section{Derived rules for FOL}

\begin{multicols}{2}
\begin{proof}
	\have[m]{ab}{\forall \meta{x}\enot \meta{A}}
	\have[\ ]{ac}{\enot \exists \meta{x} \meta{A}}\cq{m}

	\have[m]{ab}{\enot \exists \meta{x}  \meta{A}}
\\	\have[\ ]{ac}{\forall \meta{x}\enot\meta{A}}\cq{m}
\end{proof}
\begin{proof}
	\have[m]{ab}{\exists \meta{x}\enot\meta{A}}
	\have[\ ]{ac}{\enot \forall \meta{x} \meta{A}}\cq{m}

	\have[m]{ab}{\enot \forall \meta{x}  \meta{A}}
\\	\have[\ ]{ac}{\exists \meta{x}\enot \meta{A}}\cq{m}
\end{proof}
\end{multicols}
\end{minipage}
