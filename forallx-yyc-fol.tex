%!TEX root = forallxyyc.tex
\part{First-order logic}
\label{ch.FOL}
\addtocontents{toc}{\protect\mbox{}\protect\hrulefill\par}
\chapter{Building blocks of FOL}\label{s:FOLBuildingBlocks}

\section{The need to decompose sentences}
Consider the following argument, which is obviously valid in English:
\begin{quote}
\label{willard1}
Willard is a logician. All logicians wear funny hats. \therefore Willard wears a funny hat.
\end{quote}
To symbolise it in TFL, we might offer a symbolisation key:
\begin{ekey}
\item[L] Willard is a logician.
\item[A] All logicians wear funny hats.
\item[F] Willard wears a funny hat.
\end{ekey}
And the argument itself becomes:
$$L, A \therefore F$$
This is \emph{invalid} in TFL, but the original English argument is clearly valid.

The problem is not that we have made a mistake while symbolising the argument. This is the best symbolisation we can give \emph{in TFL}. The problem lies with TFL itself. `All logicians wear funny hats' is about both logicians and hat-wearing. By not retaining this structure in our symbolisation, we lose the connection between Willard's being a logician and Willard's wearing a hat.

The basic units of TFL are atomic sentences, and TFL cannot decompose these. To symbolise arguments like the preceding one, we will have to develop a new logical language which will allow us to \emph{split the atom}. We will call this language \emph{first-order logic}, or \emph{FOL}. 

The details of FOL will be explained throughout this chapter, but here is the basic idea for splitting the atom.

First, we have \emph{names}. In FOL, we indicate these with lowercase italic letters. For instance, we might let `$b$' stand for Bertie, or let `$i$' stand for Willard.

Second, we have predicates. English predicates are expressions like `\blank\ is a dog' or `\blank\ is a logician'. These are not complete sentences by themselves. In order to make a complete sentence, we need to fill in the gap. We need to say something like `Bertie is a dog' or `Willard is a logician'. In FOL, we indicate predicates with uppercase italic letters. For instance, we might let the FOL predicate `$D$' symbolise the English predicate `\blank\ is a dog'. Then the expression `$Db$' will be a sentence in FOL, which symbolises the English sentence `Bertie is a dog'. Equally, we might let the FOL predicate `$L$' symbolise the English predicate `\blank\ is a logician'. Then the expression `$Li$' will symbolise the English sentence `Willard is a logician'.

Third, we have quantifiers. For instance, `$\exists$' will roughly convey `There is at least one \ldots'. So we might symbolise the English sentence `there is a dog' with the FOL sentence `$\exists x Dx$', which we would naturally read out-loud as `there is at least one thing, x, such that x is a dog'.

That is the general idea, but FOL is significantly more subtle than TFL, so we will come at it slowly. 


\section{Names}
In English, a \emph{singular term} is a word or phrase that refers to a \emph{specific} person, place, or thing. The word `dog' is not a singular term, because there are a great many dogs. The phrase `Bertie' is a singular term, because it refers to a specific terrier. Likewise, the phrase `Philip's dog Bertie' is a singular term, because it refers to a specific little terrier. 

\emph{Proper names} are a particularly important kind of singular term. These are expressions that pick out individuals without describing them. The name `Emerson' is a proper name, and the name alone does not tell you anything about Emerson. Of course, some names are traditionally given to boys and other are traditionally given to girls. If `Hilary' is used as a singular term, you might guess that it refers to a woman. You might, though, be guessing wrongly. Indeed, the name does not necessarily mean that the person referred to is even a person: Hilary might be a giraffe, for all you could tell just from the name. 

In FOL, our \define{names} are lower-case letters `$a$' through to `$r$'. We can add subscripts if we want to use some letter more than once. So here are some singular terms in FOL:
	$$a,b,c,\ldots, r, a_1, f_{32}, j_{390}, m_{12}$$
These should be thought of along the lines of proper names in English, but with one difference. `Tim Button' is a proper name, but there are several people with this name. (Equally, there are  at least two people with the name `P.D.\ Magnus'.) We live with this kind of ambiguity in English, allowing context to individuate the fact that `Tim Button' refers to an author of this book, and not some other Tim. In FOL, we do not tolerate any such ambiguity. Each name must pick out \emph{exactly} one thing. (However, two different names may pick out the same thing.)

\newglossaryentry{name}{
  name = name,
  description = {a symbol of FOL used to pick out an object of the \gls{domain}}
  }

As with TFL, we can provide symbolisation keys. These indicate, temporarily, what a name will pick out. So we might offer:
	\begin{ekey}
		\item[e] Elsa
		\item[g] Gregor
		\item[m] Marybeth
	\end{ekey}


\section{Predicates}
The simplest predicates are properties of individuals. They are things you can say about an object. Here are some examples of English predicates:
	\begin{quote}
		\blank\ is a dog\\
		\blank\ is a member of Monty Python\\
		A piano fell on \blank
	\end{quote}
In general, you can think about predicates as things which combine with singular terms to make sentences. Conversely, you can start with sentences and make predicates out of them by removing terms. Consider the sentence, `Vinnie borrowed the family car from Nunzio.' By removing a singular term, we can obtain any of three different predicates:
	\begin{quote}
		\blank\ borrowed the family car from Nunzio\\
		Vinnie borrowed \blank\ from Nunzio\\
		Vinnie borrowed the family car from \blank
	\end{quote}
In FOL, \define{predicates} are capital letters $A$ through $Z$, with or without subscripts. We might write a symbolisation key for predicates thus:
	\begin{ekey}
		\item[Ax] \gap{x} is angry
		\item[Hx] \gap{x} is happy
%		\item[T_1xy] \gap{x} is as tall or taller than \gap{y}
%		\item[T_2xy] \gap{x} is as tough or tougher than \gap{y}
%		\item[Bxyz] \gap{y} is between \gap{x} and \gap{z}
	\end{ekey}
        (Why the subscripts on the gaps? We will return to this in \S\ref{s:MultipleGenerality}.)

\newglossaryentry{predicate}{
  name = predicate,
  description = {a symbol of FOL used to symbolise a property or relation}
}

If we combine our two symbolisation keys, we can start to symbolise some English sentences that use these names and predicates in combination. For example, consider the English sentences:
	\begin{earg}
		\item[\ex{terms1}] Elsa is angry.
		\item[\ex{terms2a}] Gregor and Marybeth are angry.
		\item[\ex{terms2}] If Elsa is angry, then so are Gregor and Marybeth.
	\end{earg}
Sentence \ref{terms1} is straightforward: we symbolise it by `$Ae$'.

Sentence \ref{terms2a}: this is a conjunction of two simpler sentences. The simple sentences can be symbolised just by `$Ag$' and `$Am$'. Then we help ourselves to our resources from TFL, and symbolise the entire sentence by `$Ag \eand Am$'. This illustrates an important point: FOL has all of the truth-functional connectives of TFL.

Sentence \ref{terms2}: this is a conditional, whose antecedent is sentence \ref{terms1} and whose consequent is sentence \ref{terms2a}, so we can symbolise this with `$Ae \eif (Ag \eand Am)$'.

\section{Quantifiers}
We are now ready to introduce quantifiers. Consider these sentences:
	\begin{earg}
		\item[\ex{q.a}] Everyone is happy.
%		\item[\ex{q.ac}] Everyone is at least as tough as Elsa.
		\item[\ex{q.e}] Someone is angry.
	\end{earg}
It might be tempting to symbolise sentence \ref{q.a} as `$He \eand Hg \eand Hm$'. Yet this would only say that Elsa, Gregor, and Marybeth are happy. We want to say that \emph{everyone} is happy, even those with no names. In order to do this, we introduce the `$\forall$' symbol. This is called the \define{universal quantifier}.

\newglossaryentry{universal quantifier}{
  name = universal quantifier,
  description = {the symbol $\forall$ of FOL used to symbolise generality; $\forall x\, Fx$ is true iff every member of the domain is~$F$}
}

A quantifier must always be followed by a \define{variable}. In FOL, variables are italic lowercase letters `$s$' through `$z$', with or without subscripts. So we might symbolise sentence \ref{q.a} as `$\forall x Hx$'.  The variable `$x$' is serving as a kind of placeholder. The expression `$\forall x$' intuitively means that you can pick anyone and put them in as `$x$'. The subsequent `$Hx$' indicates, of that thing you picked out, that it is happy. 

\newglossaryentry{variable}{
  name = variable,
  description = {a symbol of FOL used following quantifiers and as placeholders in atomic formulas; lowercase letters between $s$ and $z$}
}


It should be pointed out that there is no special reason to use `$x$' rather than some other variable. The sentences `$\forall x Hx$', `$\forall y Hy$', `$\forall z Hz$', and `$\forall x_5 Hx_5$' use different variables, but they will all be logically equivalent.

To symbolise sentence \ref{q.e}, we introduce another new symbol: the \define{existential quantifier}, `$\exists$'. Like the universal quantifier, the existential quantifier requires a variable. Sentence \ref{q.e} can be symbolised by `$\exists x Ax$'. Whereas `$\forall x Ax$' is read naturally as `for all x, x is angry', `$\exists x Ax$' is read naturally as `there is something, x, such that x is angry'. Once again, the variable is a kind of placeholder; we could just as easily have symbolised sentence \ref{q.e} with `$\exists z Az$', `$\exists w_{256} Aw_{256}$', or whatever.

\newglossaryentry{existential quantifier}{
  name = existential quantifier,
  description = {the symbol $\exists$ of FOL used to symbolise existence; $\exists x\, Fx$ is true iff at least one member of the domain is~$F$}
}

Some more examples will help. Consider these further sentences:
	\begin{earg}
		\item[\ex{q.ne}] No one is angry.
		\item[\ex{q.en}] There is someone who is not happy.
		\item[\ex{q.na}] Not everyone is happy.
	\end{earg}
Sentence \ref{q.ne} can be paraphrased as, `It is not the case that someone is angry'. We can then symbolise it using negation and an existential quantifier: `$\enot \exists x Ax$'. Yet sentence \ref{q.ne} could also be paraphrased as, `Everyone is not angry'. With this in mind, it can be symbolised using negation and a universal quantifier: `$\forall x \enot Ax$'. Both of these are acceptable symbolisations.  Indeed, it will transpire that, in general, $\forall x \enot\meta{A}$ is logically equivalent to $\enot\exists x\meta{A}$. (Notice that we have here returned to the practice of using `$\meta{A}$' as a metavariable, from \S\ref{s:UseMention}.) Symbolising a sentence one way, rather than the other, might seem more `natural' in some contexts, but it is not much more than a matter of taste.

Sentence \ref{q.en} is most naturally paraphrased as, `There is some x, such that x is not happy'. This then becomes `$\exists x \enot Hx$'. Of course, we could equally have written `$\enot\forall x Hx$', which we would naturally read as `it is not the case that everyone is happy'. That too would be a perfectly adequate symbolisation of sentence \ref{q.na}.


\section{Domains}
Given the symbolisation key we have been using, `$\forall xHx$' symbolises `Everyone is happy'.  Who is included in this \emph{everyone}? When we use sentences like this in English, we usually do not mean everyone now alive on the Earth. We certainly do not mean everyone who was ever alive or who will ever live. We usually mean something more modest: everyone now in the building, everyone enrolled in the ballet class, or whatever.

In order to eliminate this ambiguity, we will need to specify a \define{domain}. The domain is the collection of things that we are talking about. So if we want to talk about people in Chicago, we define the domain to be people in Chicago. We write this at the beginning of the symbolisation key, like this:
	\begin{ekey}
		\item[\text{domain}] people in Chicago
	\end{ekey}
The quantifiers \emph{range over} the domain. Given this domain, `$\forall x$' is to be read roughly as `Every person in Chicago is such that\ldots' and `$\exists x$' is to be read roughly as `Some person in Chicago is such that\ldots'. 

\newglossaryentry{domain}{
  name = domain,
  description = {the collection of objects assumed for a symbolisation in FOL, or that gives the range of the quantifiers in an \gls{interpretation}}
}

In FOL, the domain must always include at least one thing. Moreover, in English we can infer `something is angry' from `Gregor is angry'. In FOL, then, we will want to be able to infer `$\exists x Ax$' from `$Ag$'. So we will insist that each name must pick out exactly one thing in the domain. If we want to name people in places beside Chicago, then we need to include those people in the domain. 
	\factoidbox{
		A domain must have \emph{at least} one member. A name must pick out \emph{exactly} one member of the domain, but a member of the domain may be picked out by one name, many names, or none at all.
	}

Even allowing for a domain with just one member can produce some strange results. Suppose we have this as a symbolisation key:
\begin{ekey}
\item[\text{domain}] the Eiffel Tower
\item[Px] \gap{x} is in Paris.
\end{ekey}
The sentence $\forall x Px$ might be paraphrased in English as `Everything is in Paris.' Yet that would be misleading. It means that everything \emph{in the domain} is in Paris. This domain contains only the Eiffel Tower, so with this symbolisation key $\forall x Px$ just means that the Eiffel Tower is in Paris.

\subsection{Non-referring terms}

In FOL, each name must pick out exactly one member of the domain. A name cannot refer to more than one thing---it is a \emph{singular} term. Each name must still pick out \emph{something}. This is connected to a classic philosophical problem: the so-called problem of non-referring terms.

Medieval philosophers typically used sentences about the \emph{chimera} to exemplify this problem. Chimera is a mythological creature; it does not really exist. Consider these two sentences:
\begin{earg}
\item[\ex{chimera1}] Chimera is angry.
\item[\ex{chimera2}] Chimera is not angry.
\end{earg}
It is tempting just to define a name to mean `chimera.' The symbolisation key would look like this:
\begin{ekey}
\item[\text{domain}] creatures on Earth
\item[Ax] \gap{x} is angry.
\item[c] chimera
\end{ekey}
We could then symbolise sentence \ref{chimera1} as $Ac$ and sentence \ref{chimera2} as $\enot Ac$.

Problems will arise when we ask whether these sentences are true or false.

One option is to say that sentence \ref{chimera1} is not true, because there is no chimera. If sentence \ref{chimera1} is false because it talks about a non-existent thing, then sentence \ref{chimera2} is false for the same reason. Yet this would mean that $Ac$ and $\enot Ac$ would both be false. Given the truth conditions for negation, this cannot be the case.

Since we cannot say that they are both false, what should we do? Another option is to say that sentence \ref{chimera1} is \emph{meaningless} because it talks about a non-existent thing. So $Ac$ would be a meaningful expression in FOL for some interpretations but not for others. Yet this would make our formal language hostage to particular interpretations. Since we are interested in logical form, we want to consider the logical force of a sentence like $Ac$ apart from any particular interpretation. If $Ac$ were sometimes meaningful and sometimes meaningless, we could not do that.

This is the \emph{problem of non-referring terms}, and we will return to it later (see p.~\pageref{subsec.defdesc}.) The important point for now is that each name of FOL \emph{must} refer to something in the domain, although the domain can contain any things we like. If we want to symbolise arguments about mythological creatures, then we must define a domain that includes them. This option is important if we want to consider the logic of stories. We can symbolise a sentence like `Sherlock Holmes lived at 221B Baker Street' by including fictional characters like Sherlock Holmes in our domain.

\chapter{Sentences with one quantifier}
\label{s:MoreMonadic}

We now have all of the pieces of FOL. Symbolising more complicated sentences will only be a matter of knowing the right way to combine predicates, names, quantifiers, and connectives. There is a knack to this, and there is no substitute for practice.

%\section{Dealing with syncategorematic adjectives}
%When we encounter a sentence like 
%	\begin{earg}
%		\item[\ex{syn1}] Herbie is a white car
%	\end{earg}
%We can paraphrase this as `Herbie is white and Herbie is a car'. We can then use a symbolisation key like:
%	\begin{ekey}
%		\item[Wx] \gap{x} is white
%		\item[Cx] \gap{x} is a car
%		\item[h] Herbie
%	\end{ekey}
%This allows us to symbolise sentence \ref{syn1} as `$Wh \eand Ch$'. But now consider:
%	\begin{earg}
%		\item[\ex{syn2}] Damon Stoudamire is a short basketball player. 
%		\item[\ex{syn3}] Damon Stoudamire is a man.
%		\item[\ex{syn4}] Damon Stoudamire is a short man.
%	\end{earg}
%Following the case of Herbie, we might try to use a symbolisation key like:
%	\begin{ekey}
%		\item[Sx] \gap{x} is short
%		\item[Bx] \gap{x} is a basketball player
%		\item[Mx] \gap{x} is a man
%		\item[d] Damon Stoudamire
%	\end{ekey}
%Then we would symbolise sentence \ref{syn2} with `$Sd \eand Bd$', sentence \ref{syn3} with `$Md$' and sentence \ref{syn4} with `$Sd \eand Md$', but that would be a terrible mistake! This now suggests that sentences \ref{syn2} and \ref{syn3} together \emph{entail} sentence \ref{syn4}, but they do not. Standing at  5'10'', Damon Stoudamire is one of the shortest professional basketball players of all time, but he is nevertheless an averagely-tall man. The point is that sentence \ref{syn2} says that Damon is short \emph{qua} basketball player, even though he is of average height \emph{qua} man. So you will need to symbolise `\blank\ is a short basketball player' and `\blank\ is a short man' using completely different predicates. 

%Similar examples abound. All politicians are people, but some good politicians are (arguably) bad people. Someone might be an incompetent statesman, but a competent individual. And so it goes. The moral is: when you see two adjectives in a row, you need to ask yourself carefully whether they can be treated as a conjunction or not.


\section{Common quantifier phrases}
Consider these sentences:
	\begin{earg}
		\item[\ex{quan1}] Every coin in my pocket is a quarter.
		\item[\ex{quan2}] Some coin on the table is a dime.
		\item[\ex{quan3}] Not all the coins on the table are dimes.
		\item[\ex{quan4}] None of the coins in my pocket are dimes.
	\end{earg}
In providing a symbolisation key, we need to specify a domain. Since we are talking about coins in my pocket and on the table, the domain must at least contain all of those coins. Since we are not talking about anything besides coins, we let the domain be all coins. Since we are not talking about any specific coins, we do not need to deal with any names. So here is our key:
	\begin{ekey}
		\item[\text{domain}] all coins
		\item[Px] \gap{x} is in my pocket
		\item[Tx] \gap{x} is on the table
		\item[Qx] \gap{x} is a quarter
		\item[Dx] \gap{x} is a dime
	\end{ekey}
Sentence \ref{quan1} is most naturally symbolised using a universal quantifier. The universal quantifier says something about everything in the domain, not just about the coins in my pocket. Sentence \ref{quan1} can be paraphrased as `for any coin, \emph{if} that coin is in my pocket \emph{then} it is a quarter'. So we can symbolise it as `$\forall x(Px \eif Qx)$'.

Since sentence \ref{quan1} is about coins that are both in my pocket \emph{and} that are quarters, it might be tempting to symbolise it using a conjunction. However, the sentence `$\forall x(Px \eand Qx)$' would symbolise the sentence `every coin is both a quarter and in my pocket'. This obviously means something very different than sentence \ref{quan1}. And so we see:
	\factoidbox{
		A sentence can be symbolised as $\forall x (\meta{F}x \eif \meta{G}x)$ if it can be paraphrased in English as `every F is G'.
	}
Sentence \ref{quan2} is most naturally symbolised using an existential quantifier. It can be paraphrased as `there is some coin which is both on the table and which is a dime'. So we can symbolise it as `$\exists x(Tx \eand Dx)$'.

Notice that we needed to use a conditional with the universal quantifier, but we used a conjunction with the existential quantifier. Suppose we had instead written `$\exists x(Tx \eif Dx)$'. That would mean that there is some object in the domain of which `$(Tx \eif Dx)$' is true. Recall that, in TFL, $\meta{A} \eif \meta{B}$ is logically equivalent (in TFL) to $\enot\meta{A} \eor \meta{B}$. This equivalence will also hold in FOL. So `$\exists x(Tx \eif Dx)$' is true if there is some object in the domain, such that `$(\enot Tx \eor Dx)$' is true of that object. That is, `$\exists x (Tx \eif Dx)$' is true if some coin is \emph{either} not on the table \emph{or} is a dime. Of course there is a coin that is not on the table: there are coins lots of other places. So it is \emph{very easy} for `$\exists x(Tx \eif Dx)$' to be true. A conditional will usually be the natural connective to use with a universal quantifier, but a conditional within the scope of an existential quantifier tends to say something very weak indeed. As a general rule of thumb, do not put conditionals in the scope of existential quantifiers unless you are sure that you need one.
	\factoidbox{
		A sentence can be symbolised as $\exists x (\meta{F}x \eand \meta{G}x)$ if it can be paraphrased in English as `some F is G'.
	}		
Sentence \ref{quan3} can be paraphrased as, `It is not the case that every coin on the table is a dime'. So we can symbolise it by `$\enot \forall x(Tx \eif Dx)$'. You might look at sentence \ref{quan3} and paraphrase it instead as, `Some coin on the table is not a dime'. You would then symbolise it by `$\exists x(Tx \eand \enot Dx)$'. Although it is probably not immediately obvious yet, these two sentences are logically equivalent. (This is due to the logical equivalence between $\enot\forall x\meta{A}$ and $\exists x\enot\meta{A}$, mentioned in \S\ref{s:FOLBuildingBlocks}, along with the equivalence between $\enot(\meta{A}\eif\meta{B})$ and $\meta{A}\eand\enot\meta{B}$.)

Sentence \ref{quan4} can be paraphrased as, `It is not the case that there is some dime in my pocket'. This can be symbolised by `$\enot\exists x(Px \eand Dx)$'. It might also be paraphrased as, `Everything in my pocket is a non-dime', and then could be symbolised by `$\forall x(Px \eif \enot Dx)$'. Again the two symbolisations are logically equivalent; both are correct symbolisations of sentence \ref{quan4}.


\section{Empty predicates}

In \S\ref{s:FOLBuildingBlocks}, we emphasised that a name must pick out exactly one object in the domain. However, a predicate need not apply to anything in the domain. A predicate that applies to nothing in the domain is called an \define{empty predicate}. This is worth exploring.

\newglossaryentry{empty predicate}{
  name = {empty predicate},
  description = {a \gls{predicate} that applies to no object in the \gls{domain}}
}

Suppose we want to symbolise these two sentences:
	\begin{earg}
		\item[\ex{monkey1}] Every monkey knows sign language
		\item[\ex{monkey2}] Some monkey knows sign language
	\end{earg}
It is possible to write the symbolisation key for these sentences in this way:
	\begin{ekey}
		\item[\text{domain}] animals
		\item[Mx] \gap{x} is a monkey.
		\item[Sx] \gap{x} knows sign language.
	\end{ekey}
Sentence \ref{monkey1} can now be symbolised by `$\forall x(Mx \eif Sx)$'. Sentence \ref{monkey2} can be symbolised as `$\exists x(Mx \eand Sx)$'.

It is tempting to say that sentence \ref{monkey1} \emph{entails} sentence \ref{monkey2}. That is, we might think that it is impossible for it to be the case that every monkey knows sign language, without its also being the case that some monkey knows sign language, but this would be a mistake. It is possible for the sentence `$\forall x(Mx \eif Sx)$' to be true even though the sentence `$\exists x(Mx \eand Sx)$' is false.

How can this be? The answer comes from considering whether these sentences would be true or false \emph{if there were no monkeys}. If there were no monkeys at all (in the domain), then `$\forall x(Mx \eif Sx)$' would be \emph{vacuously} true: take any monkey you like---it knows sign language! But if there were no monkeys at all (in the domain), then `$\exists x(Mx \eand Sx)$' would be false.

Another example will help to bring this home. Suppose we extend the above symbolisation key, by adding:
	\begin{ekey}
		\item[Rx] \gap{x} is a refrigerator
	\end{ekey}
Now consider the sentence `$\forall x(Rx \eif Mx)$'. This symbolises `every refrigerator is a monkey'. This sentence is true, given our symbolisation key, which is counterintuitive, since we (presumably) do not want to say that there are a whole bunch of refrigerator monkeys. It is important to remember, though, that `$\forall x(Rx \eif Mx)$' is true iff any member of the domain that is a refrigerator is a monkey. Since the domain is \emph{animals}, there are no refrigerators in the domain. Again, then, the sentence is \emph{vacuously} true. 

If you were actually dealing with the sentence `All refrigerators are monkeys', then you would most likely want to include kitchen appliances in the domain. Then the predicate `$R$' would not be empty and the sentence `$\forall x(Rx \eif Mx)$' would be false.
	\factoidbox{
		When $\meta{F}$ is an empty predicate, a sentence $\forall x (\meta{F}x \eif \ldots)$ will be vacuously true.
	}


\section{Picking a domain}
The appropriate symbolisation of an English language sentence in FOL will depend on the symbolisation key. Choosing a key can be difficult. Suppose we want to symbolise the English sentence:
	\begin{earg}
		\item[\ex{pickdomainrose}] Every rose has a thorn.
	\end{earg}
We might offer this symbolisation key:
	\begin{ekey}
		\item[Rx] \gap{x} is a rose
		\item[Tx] \gap{x} has a thorn
	\end{ekey}
It is tempting to say that sentence \ref{pickdomainrose} should be symbolised as `$\forall x(Rx \eif Tx)$', but we have not yet chosen a domain. If the domain contains all roses, this would be a good symbolisation. Yet if the domain is merely \emph{things on my kitchen table}, then `$\forall x(Rx \eif Tx)$' would only come close to covering the fact that every rose \emph{on my kitchen table} has a thorn. If there are no roses on my kitchen table, the sentence would be trivially true. This is not what we want. To symbolise sentence \ref{pickdomainrose} adequately, we need to include all the roses in the domain, but now we have two options. 

First, we can restrict the domain to include all roses but \emph{only} roses. Then sentence \ref{pickdomainrose} can, if we like, be symbolised with `$\forall x Tx$'. This is true iff everything in the domain has a thorn; since the domain is just the roses, this is true iff every rose has a thorn. By restricting the domain, we have been able to symbolise our English sentence with a very short sentence of FOL. So this approach can save us trouble, if every sentence that we want to deal with is about roses.

Second, we can let the domain contain things besides roses: rhododendrons; rats; rifles; whatevers., and we will certainly need to include a more expansive domain if we simultaneously want to symbolise sentences like:
	\begin{earg}
		\item[\ex{pickdomaincowboy}] Every cowboy sings a sad, sad song.
	\end{earg}
Our domain must now include both all the roses (so that we can symbolise sentence \ref{pickdomainrose}) and all the cowboys (so that we can symbolise sentence \ref{pickdomaincowboy}). So we might offer the following symbolisation key:
	\begin{ekey}
		\item[\text{domain}] people and plants
		\item[Cx] \gap{x} is a cowboy
		\item[Sx] \gap{x} sings a sad, sad song
		\item[Rx] \gap{x} is a rose
		\item[Tx] \gap{x} has a thorn
	\end{ekey}
Now we will have to symbolise sentence \ref{pickdomainrose} with `$\forall x (Rx \eif Tx)$', since `$\forall x Tx$' would symbolise the sentence `every person or plant has a thorn'. Similarly, we will have to symbolise sentence \ref{pickdomaincowboy} with `$\forall x (Cx \eif Sx)$'. 

In general, the universal quantifier can be used to symbolise the English expression `everyone' if the domain only contains people. If there are people and other things in the domain, then `everyone' must be treated as `every person'.


\section{The utility of paraphrase}
When symbolising English sentences in FOL, it is important to understand the structure of the sentences you want to symbolise. What matters is the final symbolisation in FOL, and sometimes you will be able to move from an English language sentence directly to a sentence of FOL. Other times, it helps to paraphrase the sentence one or more times. Each successive paraphrase should move from the original sentence closer to something that you can easily symbolise directly in FOL.

For the next several examples, we will use this symbolisation key:
	\begin{ekey}
		\item[\text{domain}] people
		\item[Bx] \gap{x} is a bassist.
		\item[Rx] \gap{x} is a rock star.
		\item[k] Kim Deal
	\end{ekey}
Now consider these sentences:
	\begin{earg}
		\item[\ex{pronoun1}] If Kim Deal is a bassist, then she is a rock star.
		\item[\ex{pronoun2}] If a person is a bassist, then she is a rock star.
	\end{earg}
The same words appear as the consequent in sentences \ref{pronoun1} and \ref{pronoun2} (`$\ldots$ she is a rock star'), but they mean very different things. To make this clear, it often helps to paraphrase the original sentences, removing pronouns.

Sentence \ref{pronoun1} can be paraphrased as, `If Kim Deal is a bassist, then \emph{Kim Deal} is a rockstar'. This can obviously be symbolised as `$Bk \eif Rk$'.

Sentence \ref{pronoun2} must be paraphrased differently: `If a person is a bassist, then \emph{that person} is a rock star'. This sentence is not about any particular person, so we need a variable. As an intermediate step, we can paraphrase this as, `For any person x, if x is a bassist, then x is a rockstar'. Now this can be symbolised as `$\forall x (Bx \eif Rx)$'. This is the same sentence we would have used to symbolise `Everyone who is a bassist is a rock star'. On reflection, that is surely true iff sentence \ref{pronoun2} is true, as we would hope.

Consider these further sentences:
	\begin{earg}
		\item[\ex{anyone1}] If anyone is a bassist, then Kim Deal is a rock star.
		\item[\ex{anyone2}] If anyone is a bassist, then she is a rock star.
	\end{earg}
The same words appear as the antecedent in sentences \ref{anyone1} and \ref{anyone2}  (`If anyone is a bassist$\ldots$'), but it can be tricky to work out how to symbolise these two uses. Again, paraphrase will come to our aid. 

Sentence \ref{anyone1} can be paraphrased, `If there is at least one bassist, then Kim Deal is a rock star'. It is now clear that this is a conditional whose antecedent is a quantified expression; so we can symbolise the entire sentence with a conditional as the main logical operator: `$\exists x Bx \eif Rk$'.

Sentence \ref{anyone2} can be paraphrased, `For all people x, if x is a bassist, then x is a rock star'. Or, in more natural English, it can be paraphrased by `All bassists are rock stars'. It is best symbolised as `$\forall x(Bx \eif Rx)$', just like sentence \ref{pronoun2}.

The moral is that the English words `any' and `anyone' should typically be symbolised using quantifiers, and if you are having a hard time determining whether to use an existential or a universal quantifier, try paraphrasing the sentence with an English sentence that uses words \emph{besides} `any' or `anyone'.



\section{Quantifiers and scope}
Continuing the example, suppose we want to symbolise these sentences:
	\begin{earg}
		\item[\ex{qscope1}] If everyone is a bassist, then Lars is a bassist
		\item[\ex{qscope2}] Everyone is such that, if they are a bassist, then Lars is a bassist.
	\end{earg}
To symbolise these sentences, we will have to add a new name to the symbolisation key, namely:
	\begin{ekey}
		\item[l] Lars
	\end{ekey}
Sentence \ref{qscope1} is a conditional, whose antecedent is `everyone is a bassist', so we will symbolise it with `$\forall x Bx \eif Bl$'. This sentence is \emph{necessarily} true: if \emph{everyone} is indeed a bassist, then take any one you like---for example Lars---and he will be a bassist. 

Sentence \ref{qscope2}, by contrast, might best be paraphrased by `every person x is such that, if x is a bassist, then Lars is a bassist'. This is symbolised by `$\forall x (Bx \eif Bl)$'. This sentence is false; Kim Deal is a bassist. So `$Bk$' is true, but Lars is not a bassist, so `$Bl$' is false. Accordingly, `$Bk \eif Bl$' will be false, so `$\forall x (Bx \eif Bl)$' will be false as well. 

In short, `$\forall x Bx \eif Bl$' and `$\forall x (Bx \eif Bl)$' are very different sentences. We can explain the difference in terms of the \emph{scope} of the quantifier. The scope of quantification is very much like the scope of negation, which we considered when discussing TFL, and it will help to explain it in this way.

In the sentence `$\enot Bk \eif Bl$', the scope of `$\enot$' is just the antecedent of the conditional. We are saying something like: if `$Bk$' is false, then `$Bl$' is true. Similarly, in the sentence `$\forall x Bx \eif Bl$', the scope of `$\forall x$' is just the antecedent of the conditional. We are saying something like: if `$Bx$' is true of \emph{everything}, then `$Bl$' is also true. 

In the sentence `$\enot(Bk \eif Bl)$', the scope of `$\enot$' is the entire sentence. We are saying something like: `$(Bk \eif Bl)$' is false. Similarly, in the sentence `$\forall x (Bx \eif Bl)$', the scope of `$\forall x$' is the entire sentence. We are saying something like: `$(Bx \eif Bl)$' is true of \emph{everything}.

The moral of the story is simple. When you are using conditionals, be very careful to make sure that you have sorted out the scope correctly. 

\subsection{Ambiguous predicates}

Suppose we just want to symbolise this sentence:
\begin{earg}
\item[\ex{surgeon1}] Adina is a skilled surgeon.
\end{earg}
Let the domain be people, let $Kx$ mean `$x$ is a skilled surgeon', and let $a$ mean Adina. Sentence \ref{surgeon1} is simply $Ka$.


Suppose instead that we want to symbolise this argument:
\begin{quote}
The hospital will only hire a skilled surgeon. All surgeons are greedy. Billy is a surgeon, but is not skilled. Therefore, Billy is greedy, but the hospital will not hire him.
\end{quote}
We need to distinguish being a \emph{skilled surgeon} from merely being a \emph{surgeon}. So we define this symbolisation key:
\begin{ekey}
\item[\text{domain}] people
\item[Gx] \gap{x} is greedy.
\item[Hx] The hospital will hire \gap{x}.
\item[Rx] \gap{x} is a surgeon.
\item[Kx] \gap{x} is skilled.
\item[b] Billy
\end{ekey}

Now the argument can be symbolised in this way:
\begin{earg}
\label{surgeon2}
\item[] $\forall x\bigl[\enot (Rx \eand Kx) \eif \enot Hx\bigr]$
\item[] $\forall x(Rx \eif Gx)$
\item[] $Rb \eand \enot Kb$
\item[\therefore] $Gb \eand \enot Hb$
\end{earg}

Next suppose that we want to symbolise this argument:
\begin{quote}
\label{surgeon3}
Carol is a skilled surgeon and a tennis player. Therefore, Carol is a skilled tennis player.
\end{quote}
If we start with the symbolisation key we used for the previous argument, we could add a predicate (let $Tx$ mean `$x$ is a tennis player') and a name (let $c$ mean Carol). Then the argument becomes:
\begin{earg}
\item[] $(Rc \eand Kc) \eand Tc$
\item[\therefore] $Tc \eand Kc$
\end{earg}
This symbolisation is a disaster! It takes what in English is a terrible argument and symbolises it as a valid argument in FOL. The problem is that there is a difference between being \emph{skilled as a surgeon} and \emph{skilled as a tennis player}. Symbolising this argument correctly requires two separate predicates, one for each type of skill. If we let $K_1x$ mean `$x$ is skilled as a surgeon' and $K_2x$ mean `$x$ is skilled as a tennis player,' then we can symbolise the argument in this way:
\begin{earg}
\label{surgeon3correct}
\item[] $(Rc \eand K_1c) \eand Tc$
\item[\therefore] $Tc \eand K_2c$
\end{earg}
Like the English language argument it symbolises, this is invalid. %\nix{Notice that there is no logical connection between $K_1c$ and $Rc$. As symbols of FOL, they might be any one-place predicates. In English there is a connection between being a \emph{surgeon} and being a \emph{skilled surgeon}: Every skilled surgeon is a surgeon. In order to capture this connection, we symbolise `Carol is a skilled surgeon' as $Rc \eand K_1c$. This means: `Carol is a surgeon and is skilled as a surgeon.'}

The moral of these examples is that you need to be careful of symbolising predicates in an ambiguous way. Similar problems can arise with predicates like \emph{good}, \emph{bad}, \emph{big}, and \emph{small}. Just as skilled surgeons and skilled tennis players have different skills, big dogs, big mice, and big problems are big in different ways.

Is it enough to have a predicate that means `$x$ is a skilled surgeon', rather than two predicates `$x$ is skilled' and `$x$ is a surgeon'? Sometimes. As sentence \ref{surgeon1} shows, sometimes we do not need to distinguish between skilled surgeons and other surgeons.

Must we always distinguish between different ways of being skilled, good, bad, or big? No. As the argument about Billy shows, sometimes we only need to talk about one kind of skill. If you are symbolising an argument that is just about dogs, it is fine to define a predicate that means `$x$ is big.' If the domain includes dogs and mice, however, it is probably best to make the predicate mean `$x$ is big for a dog.'

\practiceproblems
\problempart
\label{pr.BarbaraEtc}
Here are the syllogistic figures identified by Aristotle and his successors, along with their medieval names:
\begin{ebullet}
	\item \textbf{Barbara.} All G are F. All H are G. So:  All H are F
	\item \textbf{Celarent.} No G are F. All H are G. So: No H are F
	\item \textbf{Ferio.} No G are F. Some H is G. So: Some H is not F
	\item \textbf{Darii.} All G are F. Some H is G. So: Some H is F.
	\item \textbf{Camestres.} All F are G. No H are G. So: No H are F.
	\item \textbf{Cesare.} No F are G. All H are G. So: No H are F.
	\item \textbf{Baroko.} All F are G. Some H is not G. So: Some H is not F.
	\item \textbf{Festino.} No F are G. Some H are G. So: Some H is not F.
	\item \textbf{Datisi.} All G are F. Some G is H. So: Some H is F.
	\item \textbf{Disamis.} Some G is F. All G are H. So: Some H is F.
	\item \textbf{Ferison.} No G are F. Some G is H. So: Some H is not F.
	\item \textbf{Bokardo.} Some G is not F. All G are H. So:  Some H is not F.
	\item \textbf{Camenes.} All F are G. No G are H So: No H is F.
	\item \textbf{Dimaris.} Some F is G. All G are H. So: Some H is F.
	\item \textbf{Fresison.} No F are G. Some G is H. So: Some H is not F.
\end{ebullet}
Symbolise each argument in FOL.

\

\problempart
\label{pr.FOLvegetarians}
Using the following symbolisation key:
\begin{ekey}
\item[\text{domain}] people
\item[Kx] \gap{x} knows the combination to the safe
\item[Sx] \gap{x} is a spy
\item[Vx] \gap{x} is a vegetarian
%\item[Txy] \gap{x} trusts \gap{y}.
\item[h] Hofthor
\item[i] Ingmar
\end{ekey}
symbolise the following sentences in FOL:
\begin{earg}
\item Neither Hofthor nor Ingmar is a vegetarian.
\item No spy knows the combination to the safe.
\item No one knows the combination to the safe unless Ingmar does.
\item Hofthor is a spy, but no vegetarian is a spy.
\end{earg}
\solutions
\problempart\label{pr.FOLalligators}
Using this symbolisation key:
\begin{ekey}
\item[\text{domain}] all animals
\item[Ax] \gap{x} is an alligator.
\item[Mx] \gap{x} is a monkey.
\item[Rx] \gap{x} is a reptile.
\item[Zx] \gap{x} lives at the zoo.
\item[a] Amos
\item[b] Bouncer
\item[c] Cleo
\end{ekey}
symbolise each of the following sentences in FOL:
\begin{earg}
\item Amos, Bouncer, and Cleo all live at the zoo. 
\item Bouncer is a reptile, but not an alligator. 
%\item If Cleo loves Bouncer, then Bouncer is a monkey. 
%\item If both Bouncer and Cleo are alligators, then Amos loves them both.
\item Some reptile lives at the zoo. 
\item Every alligator is a reptile. 
\item Any animal that lives at the zoo is either a monkey or an alligator. 
\item There are reptiles which are not alligators.
%\item Cleo loves a reptile.
%\item Bouncer loves all the monkeys that live at the zoo.
%\item All the monkeys that Amos loves love him back.
\item If any animal is an reptile, then Amos is.
\item If any animal is an alligator, then it is a reptile.
%\item Every monkey that Cleo loves is also loved by Amos.
%\item There is a monkey that loves Bouncer, but sadly Bouncer does not reciprocate this love.
\end{earg}

\problempart
\label{pr.FOLarguments}
For each argument, write a symbolisation key and symbolise the argument in FOL.
\begin{earg}
\item Willard is a logician. All logicians wear funny hats. So Willard wears a funny hat
\item Nothing on my desk escapes my attention. There is a computer on my desk. As such, there is a computer that does not escape my attention.
\item All my dreams are black and white. Old TV shows are in black and white. Therefore, some of my dreams are old TV shows.
\item Neither Holmes nor Watson has been to Australia. A person could see a kangaroo only if they had been to Australia or to a zoo. Although Watson has not seen a kangaroo, Holmes has. Therefore, Holmes has been to a zoo.
\item No one expects the Spanish Inquisition. No one knows the troubles I've seen. Therefore, anyone who expects the Spanish Inquisition knows the troubles I've seen.
\item All babies are illogical. Nobody who is illogical can manage a crocodile. Berthold is a baby. Therefore, Berthold is unable to manage a crocodile.
\end{earg}

%\problempart
%label{pr.QLarguments}
%For each argument, write a symbolisation key and symbolise the argument into FOL.
%\begin{earg}
%\item Nothing on my desk escapes my attention. There is a computer on my desk. As such, there is a computer that does not escape my attention.
%\item All my dreams are black and white. Old TV shows are in black and white. Therefore, some of my dreams are old TV shows.
%\item Neither Holmes nor Watson has been to Australia. A person could see a kangaroo only if they had been to Australia or to a zoo. Although Watson has not seen a kangaroo, Holmes has. Therefore, Holmes has been to a zoo.
%\item No one expects the Spanish Inquisition. No one knows the troubles I've seen. Therefore, anyone who expects the Spanish Inquisition knows the troubles I've seen.
%\item An antelope is bigger than a bread box. I am thinking of something that is no bigger than a bread box, and it is either an antelope or a cantaloupe. As such, I am thinking of a cantaloupe.
%\item All babies are illogical. Nobody who is illogical can manage a crocodile. Berthold is a baby. Therefore, Berthold is unable to manage a crocodile.
%\end{earg}


\chapter{Multiple generality}\label{s:MultipleGenerality}
So far, we have only considered sentences that require one-place predicates and one quantifier. The full power of FOL really comes out when we start to use many-place predicates and multiple quantifiers. For this insight, we largely have Gottlob Frege (1879) to thank, but also C.S.~Peirce.


\section{Many-placed predicates}
All of the predicates that we have considered so far concern properties that objects might have. Those predicates have one gap in them, and to make a sentence, we simply need to slot in one term. They are \define{one-place} predicates.

However, other predicates concern the \emph{relation} between two things. Here are some examples of relational predicates in English:
	\begin{quote}
		\blank\ loves \blank\\
		\blank\ is to the left of \blank\\
		\blank\ is in debt to \blank
	\end{quote}
These are \define{two-place} predicates. They need to be filled in with two terms in order to make a sentence. Conversely, if we start with an English sentence containing many singular terms, we can remove two singular terms, to obtain different two-place predicates. Consider the sentence `Vinnie borrowed the family car from Nunzio'. By deleting two singular terms, we can obtain any of three different two-place predicates
	\begin{quote}
		Vinnie borrowed \blank\ from \blank\\
		\blank\ borrowed the family car from \blank\\
		\blank\ borrowed \blank\ from Nunzio
	\end{quote}
and by removing all three singular terms, we  obtain a \define{three-place} predicate:
	\begin{quote}
		\blank\ borrowed \blank\ from \blank
	\end{quote}
Indeed, there is no in principle upper limit on the number of places that our predicates may contain.

Now there is a little foible with the above. We have used the same symbol, `\blank', to indicate a gap formed by deleting a term from a sentence. However (as Frege emphasised), these are \emph{different} gaps. To obtain a sentence, we can fill them in with the same term, but we can equally fill them in with different terms, and in various different orders. The following are all perfectly good sentences, and they all mean very different things:
	\begin{quote}
		Karl loves Karl\\
		Karl loves Imre\\
		Imre loves Karl\\
		Imre loves Imre
	\end{quote}
The point is that we need to keep track of the gaps in predicates, so that we can keep track of how we are filling them in. 

To keep track of the gaps, we will label them. The labelling conventions we will adopt are best explained by example. Suppose we want to symbolise the following sentences:
	\begin{earg}
%		\item[\ex{terms3}] Imre is at least as tall Karl.
%		\item[\ex{terms4}] Imre is shorter than Karl.
		\item[\ex{terms3}] Karl loves Imre.
		\item[\ex{terms4}] Imre loves himself.
		\item[\ex{terms5}] Karl loves Imre, but not vice versa.
		\item[\ex{terms6}] Karl is loved by Imre.
	\end{earg}
We will start with the following representation key:
	\begin{ekey}
		\item[\text{domain}] people
		\item[i] Imre
		\item[k] Karl
		\item[Lxy] \gap{x} loves \gap{y}
	\end{ekey}
%Sentence \ref{terms3} can now be symbolised by `$Tmd$'. Note the order of the names! 
%Sentence \ref{terms4} might seem as if it requires a new predicate. But there is obviously a connection connection between `shorter' and `taller.' We can paraphrase sentence \ref{terms4} using predicates already in our key: `It is not the case that Imre is as tall or taller than Karl'. We can now symbolise it as `$\enot Tmd$'.
Sentence \ref{terms3} will now be symbolised by `$Lki$'. 

Sentence \ref{terms4} can be paraphrased as `Imre loves Imre'. It can now be symbolised by `$Lii$'. 

Sentence \ref{terms5} is a conjunction. We might paraphrase it as `Karl loves Imre, and Imre does not love Karl'. It can now be symbolised by `$Lki \eand \enot Lik$'. 

Sentence \ref{terms6} might be paraphrased by `Imre loves Karl'. It can then be symbolised by `$Lik$'. Of course, this slurs over the difference in tone between the active and passive voice; such nuances are lost in FOL. 

This last example, though, highlights something important. Suppose we add to our symbolisation key the following:
	\begin{ekey}
		\item[Mxy] \gap{y} loves \gap{x}
	\end{ekey}
Here, we have used the same English word (`loves') as we used in our symbolisation key for `$Lxy$'. However, we have swapped the order of the \emph{gaps} around (just look closely at those little subscripts!) So `$Mki$' and `$Lik$' now \emph{both} symbolise `Imre loves Karl'. `$Mik$' and `$Lki$' now \emph{both} symbolise `Karl loves Imre'. Since love can be unrequited, these are very different claims. 

The moral is simple. When we are dealing with predicates with more than one place, we need to pay careful attention to the order of the places.


\section{The order of quantifiers}
Consider the sentence `everyone loves someone'. This is potentially ambiguous. It might mean either of the following:
	\begin{earg}
		\item[\ex{lovecycle}] For every person x, there is some person that x loves
		\item[\ex{loveconverge}] There is some particular person whom every person loves
	\end{earg}
Sentence \ref{lovecycle} can be symbolised by `$\forall x \exists y Lxy$', and would be true of a love-triangle. For example, suppose that our domain of discourse is restricted to Imre, Juan and Karl. Suppose also that Karl loves Imre but not Juan, that Imre loves Juan but not Karl, and that Juan loves Karl but not Imre. Then sentence \ref{lovecycle} is true. 

Sentence \ref{loveconverge} is symbolised by `$\exists y \forall x Lxy$'. Sentence \ref{loveconverge} is \emph{not} true in the situation just described. Again, suppose that our domain of discourse is restricted to Imre, Juan and Karl. This requires that all of Juan, Imre and Karl converge on (at least) one object of love. 

The point of the example is to illustrate that the order of the quantifiers matters a great deal. Indeed, to switch them around is called a \emph{quantifier shift fallacy}. Here is an example, which comes up in various forms throughout the philosophical literature:
	\begin{earg}
		\item[] For every person, there is some truth they cannot know. \hfill ($\forall \exists$)
		\item[\therefore] There is some truth that no person can know. \hfill ($\exists \forall$)
	\end{earg}
This argument form is obviously invalid. It's just as bad as:\footnote{Thanks to Rob Trueman for the example.}
	\begin{earg}
		\item[] Every dog has its day. \hfill ($\forall \exists$)
		\item[\therefore] There is a day for all the dogs. \hfill ($\exists \forall$)
	\end{earg}

   
The order of quantifiers is also important in definitions in mathematics.  For instance, there is a big difference between pointwise and uniform continuity of functions:
\begin{itemize}
\item A function $f$ \emph{pointwise continuous} if
\[
\forall \epsilon\forall x\forall y\exists \delta(\left|x - y\right| < \delta \to \left|f(x) - f(y)\right| < \epsilon)
\]
\item A function $f$ \emph{uniformly continuous} if
\[
\forall \epsilon\exists \delta\forall x\forall y(\left|x - y\right| < \delta \to \left|f(x) - f(y)\right| < \epsilon)
\]
\end{itemize}

The moral is: take great care with the order of quantification.


\section{Stepping-stones to symbolisation}
Once we have the possibility of multiple quantifiers and many-place predicates, representation in FOL can quickly start to become a bit tricky. When you are trying to symbolise a complex sentence, we recommend laying down several stepping stones. As usual, this idea is best illustrated by example. Consider this representation key:
\begin{ekey}
\item[\text{domain}] people and dogs
\item[Dx] \gap{x} is a dog
\item[Fxy] \gap{x} is a friend of \gap{y}
\item[Oxy] \gap{x} owns \gap{y}
\item[g] Geraldo
\end{ekey}
Now let's try to symbolise these sentences:
\begin{earg}
\item[\ex{dog2}] Geraldo is a dog owner.
\item[\ex{dog3}] Someone is a dog owner.
\item[\ex{dog4}] All of Geraldo's friends are dog owners.
\item[\ex{dog5}] Every dog owner is the friend of a dog owner.
\item[\ex{dog6}] Every dog owner's friend owns a dog of a friend.
\end{earg}
Sentence \ref{dog2} can be paraphrased as, `There is a dog that Geraldo owns'. This can be symbolised by `$\exists x(Dx \eand Ogx)$'.

Sentence \ref{dog3} can be paraphrased as, `There is some y such that y is a dog owner'. Dealing with part of this, we might write `$\exists y(y\text{ is a dog owner})$'. Now the fragment we have left as `$y$ is a dog owner' is much like sentence \ref{dog2}, except that it is not specifically about Geraldo. So we can symbolise sentence \ref{dog3} by:
$$\exists y \exists x(Dx \eand Oyx)$$
We should pause to clarify something here. In working out how to symbolise the last sentence, we wrote down `$\exists y(y\text{ is a dog owner})$'. To be very clear: this is \emph{neither} an FOL sentence \emph{nor} an English sentence: it uses bits of FOL (`$\exists$', `$y$') and bits of English (`dog owner'). It is really is \emph{just a stepping-stone} on the way to symbolising the entire English sentence with a FOL sentence. You should regard it as a bit of rough-working-out, on a par with the doodles that you might absent-mindedly draw in the margin of this book, whilst you are concentrating fiercely on some problem.  

Sentence \ref{dog4} can be paraphrased as, `Everyone who is a friend of Geraldo is a dog owner'. Using our stepping-stone tactic, we might write 
$$\forall x \bigl[Fxg \eif x \text{ is a dog owner}\bigr]$$
Now the fragment that we have left to deal with, `$x$ is a dog owner', is structurally just like sentence \ref{dog2}. However, it would be a mistake for us simply to write 
$$\forall x \bigl[Fxg \eif \exists x(Dx \eand Oxx)\bigr]$$
for we would here have a \emph{clash of variables}. The scope of the universal quantifier, `$\forall x$', is the entire conditional, so the `$x$' in `$Dx$' should be governed by that, but `$Dx$' also falls under the scope of the existential quantifier `$\exists x$', so the `$x$' in `$Dx$' should be governed by that. Now confusion reigns: which `$x$' are we talking about? Suddenly the sentence becomes ambiguous (if it is even meaningful at all), and logicians hate ambiguity. The broad moral is that a single variable cannot serve two quantifier-masters simultaneously. 

To continue our symbolisation, then, we must choose some different variable for our existential quantifier. What we want is something like:
$$\forall x\bigl[Fxg \eif\exists z(Dz \eand Oxz)\bigr]$$
This adequately symbolises sentence \ref{dog4}.

Sentence \ref{dog5} can be paraphrased as `For any x that is a dog owner, there is a dog owner who is a friend of x'. Using our stepping-stone tactic, this becomes 
$$\forall x\bigl[\mbox{$x$ is a dog owner}\eif\exists y(\mbox{$y$ is a dog owner}\eand Fyx)\bigr]$$
Completing the symbolisation, we end up with
$$\forall x\bigl[\exists z(Dz \eand Oxz)\eif\exists y\bigl(\exists z(Dz \eand Oyz)\eand Fyx\bigr)\bigr]$$
Note that we have used the same letter, `$z$', in both the antecedent and the consequent of the conditional, but that these are governed by two different quantifiers. This is ok: there is no clash here, because it is clear which quantifier that variable falls under. We might graphically represent the scope of the quantifiers thus:
$$\overbrace{\forall x\bigl[\overbrace{\exists z(Dz \eand Oxz)}^{\text{scope of 1st `}\exists z\text{'}}\eif \overbrace{\exists y(\overbrace{\exists z(Dz \eand Oyz)}^{\text{scope of 2nd `}\exists z\text{'}}\eand Fyx)\bigr]}^{\text{scope of `}\exists y\text{'}}}^{\text{scope of `}\forall x\text{'}}$$
This shows that no variable is being forced to serve two masters simultaneously.

Sentence \ref{dog6} is the trickiest yet. First we paraphrase it as `For any x that is a friend of a dog owner, x owns a dog which is also owned by a friend of x'. Using our stepping-stone tactic, this becomes:

\
\\$\forall x\bigl[x\text{ is a friend of a dog owner}\eif \phantom{x}$\\
\phantom{x}\hfill $x\text{ owns a dog which is owned by a friend of }x\bigr]$

\
\\Breaking this down a bit more:

\
\\$\forall x\bigl[\exists y(Fxy \eand y\text{ is a dog owner})\eif \phantom{x}$\\
\phantom{x}\hfill $\exists y(Dy \eand Oxy \eand y\text{ is owned by a friend of }x)\bigr]$

\
\\And a bit more: 
$$\forall x\bigl[\exists y(Fxy \eand \exists z(Dz \eand Oyz)) \eif \exists y(Dy \eand Oxy \eand \exists z(Fzx \eand Ozy))\bigr]$$
And we are done!

\section{Supressed quantifiers}

Logic can often help to get clear on the meanings of English claims,
especiially where the quantifiers are left implicit or their order is
ambiguous or unclear. The clarity of expression and thinking afforded
by FOL can give you a significant advantage in argument, as can be
seen in the following takedown by British political philosopher Mary
Astell (1666--1731) of her contemporary, the theologian William
Nicholls. In Discourse IV: The Duty of Wives to their Husbands of his
\textit{The Duty of Inferiors towards their Superiors, in Five
  Practical Discourses} (London 1701), Nicholls argued that women are
naturally inferior to men. In the preface to the 3rd edition of her
treatise \emph{Some Reflections upon Marriage, Occasion'd by the Duke
  and Duchess of Mazarine's Case; which is also considered,} Astell
responded as follows:
\begin{quotation}
'Tis true, thro' Want of Learning, and of that Superior Genius which
Men as Men lay claim to, she [Astell] was ignorant of the
\textit{Natural Inferiority} of our Sex, which our Masters lay down as
a Self-Evident and Fundamental Truth. She saw nothing in the Reason of
Things, to make this either a Principle or a Conclusion, but much to
the contrary; it being Sedition at least, if not Treason to assert it
in this Reign. 

For if by the Natural Superiority of their Sex, they mean that
\textit{every} Man is by Nature superior to \textit{every} Woman,
which is the obvious meaning, and that which must be stuck to if they
would speak Sense, it wou'd be a Sin in \textit{any} Woman to have
Dominion over \textit{any} Man, and the greatest Queen ought not to
command but to obey her Footman, because no Municipal Laws can
supersede or change the Law of Nature; so that if the Dominion of the
Men be such, the \textit{Salique Law,}\footnote{The Salique law was
  the common law of France which prohibited the crown be passed on to
  female heirs.} as unjust as \textit{English Men} have ever thought
it, ought to take place over all the Earth, and the most glorious
Reigns in the \textit{English, Danish, Castilian}, and other Annals,
were wicked Violations of the Law of Nature!

If they mean that \textit{some} Men are superior to \textit{some}
Women this is no great Discovery; had they turn'd the Tables they
might have seen that \textit{some} Women are Superior to \textit{some}
Men. Or had they been pleased to remember their Oaths of Allegiance
and Supremacy, they might have known that \textit{One} Women is
superior to \textit{All} the Men in these Nations, or else they have
sworn to very little purpose.\footnote{In 1706, England was ruled by
  Queen Anne.} And it must not be suppos'd, that their Reason and
Religion wou'd suffer them to take Oaths, contrary to the Laws of
Nature and Reason of things.\footnote{Mary Astell, \textit{Reflections
    upon Marriage}, 1706 Preface, iii--iv, and Mary Astell,
  \textit{Political Writings}, ed. Patricia Springborg, Cambridge
  University Press, 1996, 9--10.}
\end{quotation}
We can symbolise the different interpretations Astell offers of
Nicholls' claim that men are superior to women:
He either meant that every man is superior to every woman, i.e.,
\[
\forall x(Mx \eif \forall y(Wy \eif Sxy))
\]
or that some men are superior to some women,
\[
\exists x(Mx \eand \exists y(Wy \eand Sxy)).
\]
The latter is true, but so is
\[
\exists y(Wy \eand \exists x(Mx \eand Syx)).
\]
(some women are superior to some men), so that would be ``no great
discovery.''  In fact, since the Queen is superior to all her
subjects, it's even true that some woman is superior to every man,
i.e.,
\[
\exists y(Wy \land \forall x(Mx \eif Syx)).
\]
But this is incompatible with the ``obvious meaning'' of Nicholls'
claim, i.e., the first reading. So what Nicholls claims amounts to
treason against the Queen!

\practiceproblems
\solutions
\problempart
Using this symbolisation key:
\begin{ekey}
\item[\text{domain}] all animals
\item[Ax] \gap{x} is an alligator
\item[Mx] \gap{x} is a monkey
\item[Rx] \gap{x} is a reptile
\item[Zx] \gap{x} lives at the zoo
\item[Lxy] \gap{x} loves \gap{y}
\item[a] Amos
\item[b] Bouncer
\item[c] Cleo
\end{ekey}
symbolise each of the following sentences in FOL:
\begin{earg}
%\item Amos, Bouncer, and Cleo all live at the zoo. 
%\item Bouncer is a reptile, but not an alligator. 
\item If Cleo loves Bouncer, then Bouncer is a monkey. 
\item If both Bouncer and Cleo are alligators, then Amos loves them both.
%\item Some reptile lives at the zoo. 
%\item Every alligator is a reptile. 
%\item Any animal that lives at the zoo is either a monkey or an alligator. 
%\item There are reptiles which are not alligators.
\item Cleo loves a reptile.
\item Bouncer loves all the monkeys that live at the zoo.
\item All the monkeys that Amos loves love him back.
%\item If any animal is an reptile, then Amos is.
%\item If any animal is an alligator, then it is a reptile.
\item Every monkey that Cleo loves is also loved by Amos.
\item There is a monkey that loves Bouncer, but sadly Bouncer does not reciprocate this love.
\end{earg}

\problempart 
Using the following symbolisation key:
\begin{ekey}
\item[\text{domain}] all animals
\item[Dx] \gap{x} is a dog
\item[Sx] \gap{x} likes samurai movies
\item[Lxy] \gap{x} is larger than \gap{y}
\item[r] Rave
\item[s] Shane
\item[d] Daisy
\end{ekey}
symbolise the following sentences in FOL:
\begin{earg}
\item Rave is a dog who likes samurai movies.
\item Rave, Shane, and Daisy are all dogs.
\item Shane is larger than Rave, and Daisy is larger than Shane.
\item All dogs like samurai movies.
\item Only dogs like samurai movies.
\item There is a dog that is larger than Shane.
\item If there is a dog larger than Daisy, then there is a dog larger than Shane.
\item No animal that likes samurai movies is larger than Shane.
\item No dog is larger than Daisy.
\item Any animal that dislikes samurai movies is larger than Rave.
\item There is an animal that is between Rave and Shane in size.
\item There is no dog that is between Rave and Shane in size.
\item No dog is larger than itself.
\item Every dog is larger than some dog.
\item There is an animal that is smaller than every dog.
\item If there is an animal that is larger than any dog, then that animal does not like samurai movies.
\end{earg}

\problempart
\label{pr.QLcandies}
Using the symbolisation key given, symbolise each English-language sentence into FOL.
\begin{ekey}
\item[\text{domain}] candies
\item[Cx] \gap{x} has chocolate in it.
\item[Mx] \gap{x} has marzipan in it.
\item[Sx] \gap{x} has sugar in it.
\item[Tx] Boris has tried \gap{x}.
\item[Bxy] \gap{x} is better than \gap{y}.
\end{ekey}
\begin{earg}
\item Boris has never tried any candy.
\item Marzipan is always made with sugar.
\item Some candy is sugar-free.
\item The very best candy is chocolate.
\item No candy is better than itself.
\item Boris has never tried sugar-free chocolate.
\item Boris has tried marzipan and chocolate, but never together.
%\item Boris has tried nothing that is better than sugar-free marzipan.
\item Any candy with chocolate is better than any candy without it.
\item Any candy with chocolate and marzipan is better than any candy that lacks both.
\end{earg}

\problempart
Using the following symbolisation key:
\begin{ekey}
\item[\text{domain}] people and dishes at a potluck
\item[Rx] \gap{x} has run out.
\item[Tx] \gap{x} is on the table.
\item[Fx] \gap{x} is food.
\item[Px] \gap{x} is a person.
\item[Lxy] \gap{x} likes \gap{y}.
\item[e] Eli
\item[f] Francesca
\item[g] the guacamole
\end{ekey}
symbolise the following English sentences in FOL:
\begin{earg}
\item All the food is on the table.
\item If the guacamole has not run out, then it is on the table.
\item Everyone likes the guacamole.
\item If anyone likes the guacamole, then Eli does.
\item Francesca only likes the dishes that have run out.
\item Francesca likes no one, and no one likes Francesca.
\item Eli likes anyone who likes the guacamole.
\item Eli likes anyone who likes the people that he likes.
\item If there is a person on the table already, then all of the food must have run out.
\end{earg}


\solutions
\problempart
\label{pr.FOLballet}
Using the following symbolisation key:
\begin{ekey}
\item[\text{domain}] people
\item[Dx] \gap{x} dances ballet.
\item[Fx] \gap{x} is female.
\item[Mx] \gap{x} is male.
\item[Cxy] \gap{x} is a child of \gap{y}.
\item[Sxy] \gap{x} is a sibling of \gap{y}.
\item[e] Elmer
\item[j] Jane
\item[p] Patrick
\end{ekey}
symbolise the following sentences in FOL:
\begin{earg}
\item All of Patrick's children are ballet dancers.
\item Jane is Patrick's daughter.
\item Patrick has a daughter.
\item Jane is an only child.
\item All of Patrick's sons dance ballet.
\item Patrick has no sons.
\item Jane is Elmer's niece.
\item Patrick is Elmer's brother.
\item Patrick's brothers have no children.
\item Jane is an aunt.
\item Everyone who dances ballet has a brother who also dances ballet.
\item Every woman who dances ballet is the child of someone who dances ballet.
\end{earg}


\chapter{Identity}
\label{sec.identity}

Consider this sentence:
\begin{earg}
\item[\ex{else1}] Pavel owes money to everyone
\end{earg}
Let the domain be people; this will allow us to symbolise `everyone' as a universal quantifier. Offering the symbolisation key:
	\begin{ekey}
		\item[Oxy] \gap{x} owes money to \gap{y}
		\item[p] Pavel
	\end{ekey}
we can symbolise sentence \ref{else1} by `$\forall x Opx$'. But this has a (perhaps) odd consequence. It requires that Pavel owes money to every member of the domain (whatever the domain may be). The domain certainly includes Pavel. So this entails that Pavel owes money to himself. 

Perhaps we meant to say:
	\begin{earg}
		\item[\ex{else1b}] Pavel owes money to everyone \emph{else}
		\item[\ex{else1c}] Pavel owes money to everyone \emph{other than} Pavel
		\item[\ex{else1d}] Pavel owes money to everyone \emph{except} Pavel himself
	\end{earg}
but we do not know how to deal with the italicised words yet. The solution is to add another symbol to FOL. 

\section{Adding identity}

The symbol `$=$' is a two-place predicate. Since it is to have a special meaning, we will write it a bit differently: we put it between two terms, rather than out front. And it \emph{does} have a very particular meaning. We \emph{always} adopt the following symbolisation key:
	\begin{ekey}
		\item[x=y] \gap{x} is identical to \gap{y}
	\end{ekey}
This does not mean \emph{merely} that the objects in question are indistinguishable, or that all of the same things are true of them. Rather, it means that the objects in question are \emph{the very same} object.

Now suppose we want to symbolise this sentence:
\begin{earg}
\item[\ex{else2}] Pavel is Mister Checkov.
\end{earg}
Let us add to our symbolisation key:
	\begin{ekey}
		\item[c] Mister Checkov
	\end{ekey}
Now sentence \ref{else2} can be symbolised as `$p=c$'. This means that the names `$p$' and `$c$' both name the same thing.

We can also now deal with sentences \ref{else1b}--\ref{else1d}. All of these sentences can be  paraphrased as `Everyone who is not Pavel is owed money by Pavel'. Paraphrasing some more, we get: `For all x, if x is not Pavel, then x is owed money by Pavel'. Now that we are armed with our new identity symbol, we can symbolise this as `$\forall x (\enot x = p \eif Opx)$'.

This last sentence contains the formula `$\enot x = p$'. That might look a bit strange, because the symbol that comes immediately after the `$\enot$' is a variable, rather than a predicate, but this is not a problem. We are simply negating the entire formula, `$x = p$'. 

In addition to sentences that use the word `else', `other than' and `except', identity will be helpful when symbolising some sentences that contain the words `besides' and `only.' Consider these examples:

\begin{earg}
\item[\ex{else3}] No one besides Pavel owes money to Hikaru.
\item[\ex{else4}] Only Pavel owes Hikaru money.
\end{earg}
Let `$h$' name Hikaru. Sentence \ref{else3} can be paraphrased as, `No one who is not Pavel owes money to Hikaru'. This can be symbolised by `$\enot\exists x(\enot x = p \eand Oxh)$'. Equally, sentence \ref{else3} can be paraphrased as `for all x, if x owes money to Hikaru, then x is Pavel'. It can then be symbolised as `$\forall x (Oxh \eif x = p)$'.

Sentence \ref{else4} can be treated similarly, but there is one subtlety here. Do either sentence \ref{else3} or \ref{else4} entail that Pavel himself owes money to Hikaru? 

\section{There are at least\ldots}
We can also use identity to say how many things there are of a particular kind. For example, consider these sentences:
\begin{earg}
\item[\ex{atleast1}] There is at least one apple
\item[\ex{atleast2}] There are at least two apples
\item[\ex{atleast3}] There are at least three apples
\end{earg}
We will use the symbolisation key:
	\begin{ekey}
		\item[Ax] \gap{x} is an apple
	\end{ekey}
Sentence \ref{atleast1} does not require identity. It can be adequately symbolised by `$\exists x Ax$': There is an apple; perhaps many, but at least one.

It might be tempting to also symbolise sentence \ref{atleast2} without identity. Yet consider the sentence `$\exists x \exists y(Ax \eand Ay)$'. Roughly, this says that there is some apple $x$ in the domain and some apple $y$ in the domain. Since nothing precludes these from being one and the same apple, this would be true even if there were only one apple. In order to make sure that we are dealing with \emph{different} apples, we need an identity predicate. Sentence \ref{atleast2} needs to say that the two apples that exist are not identical, so it can be symbolised by `$\exists x \exists y((Ax \eand Ay) \eand \enot x = y)$'.

Sentence \ref{atleast3} requires talking about three different apples. Now we need three existential quantifiers, and we need to make sure that each will pick out something different: `$\exists x \exists y\exists z[((Ax \eand Ay) \eand Az) \eand ((\enot x = y \eand \enot y = z) \eand \enot x = z)]$'.

\section{There are at most\ldots}
Now consider these sentences:
\begin{earg}
	\item[\ex{atmost1}] There is at most one apple
	\item[\ex{atmost2}] There are at most two apples
\end{earg}
Sentence \ref{atmost1} can be paraphrased as, `It is not the case that there are at least \emph{two} apples'. This is just the negation of sentence \ref{atleast2}: 
$$\enot \exists x \exists y[(Ax \eand Ay) \eand \enot x = y]$$
But sentence \ref{atmost1} can also be approached in another way. It means that if you pick out an object and it's an apple, and then you pick out an object and it's also an apple, you must have picked out the same object both times. With this in mind, it can be symbolised by
$$\forall x\forall y\bigl[(Ax \eand Ay) \eif x=y\bigr]$$
The two sentences will turn out to be logically equivalent.

In a similar way, sentence \ref{atmost2} can be approached in two equivalent ways. It can be paraphrased as, `It is not the case that there are \emph{three} or more distinct apples', so we can offer:
$$\enot \exists x \exists y\exists z(Ax \eand Ay \eand Az \eand \enot x = y \eand \enot y = z \eand \enot x = z)$$
Alternatively we can read it as saying that if you pick out an apple, and an apple, and an apple, then you will have picked out (at least) one of these objects more than once. Thus:
$$\forall x\forall y\forall z\bigl[(Ax \eand Ay \eand Az) \eif (x=y \eor x=z \eor y=z)\bigr]$$


\section{There are exactly\ldots}
We can now consider precise statements, like:
\begin{earg}
\item[\ex{exactly1}] There is exactly one apple.
\item[\ex{exactly2}] There are exactly two apples.
\item[\ex{exactly3}] There are exactly three apples.
\end{earg}
Sentence \ref{exactly1} can be paraphrased as, `There is \emph{at least} one apple and there is \emph{at most} one apple'. This is just the conjunction of sentence \ref{atleast1} and sentence \ref{atmost1}. So we can offer:
$$\exists x Ax \eand \forall x\forall y\bigl[(Ax \eand Ay) \eif x=y\bigr]$$
But it is perhaps more straightforward to paraphrase sentence \ref{exactly1} as, `There is a thing x which is an apple, and everything which is an apple is just x itself'. Thought of in this way, we offer: 
$$\exists x\bigl[Ax \eand \forall y(Ay \eif x= y)\bigr]$$
Similarly, sentence \ref{exactly2} may be paraphrased as, `There are \emph{at least} two apples, and there are \emph{at most} two apples'. Thus we could offer 
\begin{multline*}
  \exists x \exists y((Ax \eand Ay) \eand \enot x = y) \eand {}\\
  \forall x\forall y\forall z\bigl[((Ax \eand Ay) \eand Az) \eif ((x=y \eor x=z) \eor y=z)\bigr]
\end{multline*}
More efficiently, though, we can paraphrase it as `There are at least two different apples, and every apple is one of those two apples'. Then we offer:
$$\exists x\exists y\bigl[((Ax \eand Ay) \eand \enot x = y) \eand \forall z(Az \eif ( x= z \eor y = z)\bigr]$$
Finally, consider these sentence:
\begin{earg}
\item[\ex{exactly2things}] There are exactly two things
\item[\ex{exactly2objects}] There are exactly two objects
\end{earg}
It might be tempting to add a predicate to our symbolisation key, to symbolise the English predicate `\blank\ is a thing' or `\blank\ is an object', but this is unnecessary. Words like `thing' and `object' do not sort wheat from chaff: they apply trivially to everything, which is to say, they apply trivially to every thing. So we can symbolise either sentence with either of the following:
	\begin{center}
		$\exists x \exists y \enot x = y \eand \enot \exists x \exists y \exists z ((\enot x = y \eand \enot y = z) \eand \enot x = z)$\\
		
		$\exists x \exists y \bigl[\enot x = y \eand \forall z(x=z \eor y = z)\bigr]$
	\end{center}

\practiceproblems

%\solutions
%\problempart
%\label{pr.FOLcandies}
%Using the following symbolisation key:
%\begin{ekey}
%\item[\text{domain}] candies
%\item[Cx] \gap{x} has chocolate in it.
%\item[Mx] \gap{x} has marzipan in it.
%\item[Sx] \gap{x} has sugar in it.
%\item[Tx] Boris has tried \gap{x}.
%\item[Bxy] \gap{x} is better than \gap{y}.
%\end{ekey}
%symbolise the following English sentences in FOL:
%\begin{earg}
%\item Boris has never tried any candy.
%\item Marzipan is always made with sugar.
%\item Some candy is sugar-free.
%\item The very best candy is chocolate.
%\item No candy is better than itself.
%\item Boris has never tried sugar-free chocolate.
%\item Boris has tried marzipan and chocolate, but never together.
%%\item Boris has tried nothing that is better than sugar-free marzipan.
%\item Any candy with chocolate is better than any candy without it.
%\item Any candy with chocolate and marzipan is better than any candy that lacks both.
%\end{earg}



\problempart Explain why:
	\begin{ebullet}
		\item   `$\exists x \forall y(Ay \eiff x= y)$' is a good symbolisation of `there is exactly one apple'.
		\item `$\exists x \exists y \bigl[\enot x = y \eand \forall z(Az \eiff (x= z \eor y = z)\bigr]$' is a good symbolisation of `there are exactly two apples'.
	\end{ebullet}		


\chapter{Definite descriptions}\label{subsec.defdesc}
Consider sentences like:
	\begin{earg}
		\item[\ex{traitor1}] Nick is the traitor.
		\item[\ex{traitor2}] The traitor went to Cambridge.
		\item[\ex{traitor3}] The traitor is the deputy 
	\end{earg}
These are definite descriptions: they are meant to pick out a \emph{unique} object. They should be contrasted with \emph{indefinite} descriptions, such as `Nick  is \emph{a} traitor'. They should equally be contrasted with \emph{generics}, such as `\emph{The} whale is a mammal' (it's inappropriate to ask \emph{which} whale). The question we face is: how should we deal with definite descriptions in FOL?


\section{Treating definite descriptions as terms}
One option would be to introduce new names whenever we come across a definite description. This is probably not a great idea. We know that \emph{the} traitor---whoever it is---is indeed \emph{a} traitor. We want to preserve that information in our symbolisation.

A second option would be to use a \emph{new} definite description operator, such as `$\maththe$'. The idea would be to symbolise `the F' as `$\maththe xFx$'; or to symbolise `the G' as `$\maththe xGx$', etc. Expression of the form $\maththe \meta{x} \meta{A}\meta{x}$ would then behave like names. If we followed this path, then using the following symbolisation key:
	\begin{ekey}
		\item[\text{domain}] people
		\item[Tx] \gap{x} is a traitor
		\item[Dx] \gap{x} is a deputy
		\item[Cx] \gap{x} went to Cambridge
		\item[n] Nick
	\end{ekey}
We could symbolise sentence \ref{traitor1} with `$\maththe x Tx = n$', sentence \ref{traitor2} with `$C\maththe xTx$', and sentence \ref{traitor3} with `$\maththe x Tx = \maththe x Dx$'. 

However, it would be nice if we didn't have to add a new symbol to FOL. And indeed, we might be able to make do without one.

\section{Russell's analysis}
Bertrand Russell offered an analysis of definite descriptions. Very briefly put, he observed that, when we say `the F' in the context of a definite description, our aim is to pick out the \emph{one and only} thing that is F (in the appropriate context). Thus Russell analysed the notion of a definite description as follows:\footnote{Bertrand Russell, `On Denoting', 1905, \emph{Mind 14}, pp.\ 479--93; also Russell, \emph{Introduction to Mathematical Philosophy}, 1919, London: Allen and Unwin, ch.\ 16.}
	\begin{align*}
		\text{the F is G \textbf{iff} }&\text{there is at least one F, \emph{and}}\\
	&\text{there is at most one F, \emph{and}}\\	
	&\text{every F is G}
\end{align*}
Note a very important feature of this analysis: \emph{`the' does not appear on the right-side of the equivalence.} Russell is aiming to provide an understanding of definite descriptions in terms that do not presuppose them. 

Now, one might worry that we can say `the table is brown' without implying that there is one and only one table in the universe. But this is not (yet) a fantastic counterexample to Russell's analysis. The domain of discourse is likely to be restricted by context (e.g.\ to objects in my line of sight).

If we accept Russell's analysis of definite descriptions, then we can symbolise sentences of the form `the F is G' using our strategy for numerical quantification in FOL. After all, we can deal with the three conjuncts on the right-hand side of Russell's analysis as follows:
	$$\exists x Fx \eand \forall x \forall y ((Fx \eand Fy) \eif x = y) \eand \forall x (Fx \eif Gx)$$
In fact, we could express the same point rather more crisply, by recognising that the first two conjuncts just amount to the claim that there is \emph{exactly} one F, and that the last conjunct tells us that that object is F. So, equivalently, we could offer:
	$$\exists x \bigl[(Fx \eand \forall y (Fy \eif x = y)) \eand Gx\bigr]$$
Using these sorts of techniques, we can now symbolise sentences \ref{traitor1}--\ref{traitor3} without using any new-fangled fancy operator, such as `$\maththe$'. 

Sentence \ref{traitor1} is exactly like the examples we have just considered. So we would symbolise it by `$\exists x (Tx \eand \forall y(Ty \eif x = y) \eand x = n)$'. 

Sentence \ref{traitor2} poses no problems either: `$\exists x (Tx \eand \forall y(Ty \eif x = y) \eand Cx)$'.

Sentence \ref{traitor3} is a little trickier, because it links two definite descriptions. But, deploying  Russell's analysis, it can be paraphrased by `there is exactly one traitor, x, and there is exactly one deputy, y, and x = y'. So we can symbolise it by: 
$$\exists x \exists y \bigl(\bigl[Tx \eand \forall z(Tz \eif x = z)\bigr] \eand \bigl[Dy \eand \forall z(Dz \eif y = z)\bigr] \eand x = y\bigr)$$
Note that we have made sure that the formula `$x = y$' falls within the scope of both quantifiers!

\section{Empty definite descriptions}
One of the nice features of Russell's analysis is that it allows us to handle \emph{empty} definite descriptions neatly. 

France has no king at present. Now, if we were to introduce a name, `$k$', to name the present King of France, then everything would go wrong: remember from \S\ref{s:FOLBuildingBlocks} that a name must always pick out  some object in the domain, and whatever we choose as our domain, it will contain no present kings of France. 

Russell's analysis neatly avoids this problem. Russell tells us to treat definite descriptions using predicates and quantifiers, instead of names. Since predicates can be empty (see \S\ref{s:MoreMonadic}), this means that no difficulty now arises when the definite description is empty. 

Indeed, Russell's analysis helpfully highlights two ways to go wrong in a claim involving a definite description. To adapt an example from Stephen Neale (1990),\footnote{Neale, \emph{Descriptions}, 1990, Cambridge: MIT Press.}  suppose Alex claims:
	\begin{earg}
		\item[\ex{kingdate}] I am dating the present king of France.
	\end{earg}
Using the following symbolisation key:
	\begin{ekey}
		\item[a] Alex
		\item[Kx] \gap{x} is a present king of France
		\item[Dxy] \gap{x} is dating \gap{y}
	\end{ekey}
Sentence \ref{kingdate} would be symbolised by `$\exists x (\forall y(Ky \eiff  x = y) \eand Dax)$'. Now, this can be false in (at least) two ways, corresponding to these two different sentences:
	\begin{earg}
		\item[\ex{outernegation}] There is no one who is both the present King of France and  such that he and Alex are dating.
		\item[\ex{innernegation}] There is a unique present King of France, but Alex is not dating him.
	\end{earg}
Sentence \ref{outernegation} might be paraphrased by `It is not the case that: the present King of France and Alex are dating'. It will then be symbolised by `$\enot \exists x\bigl[(Kx \eand \forall y(Ky \eif  x = y)) \eand Dax \bigr]$'. We might call this \emph{outer} negation, since the negation governs the entire sentence. Note that it will be true if there is no present King of France.

Sentence \ref{innernegation} can be symbolised by `$\exists x ((Kx \eand \forall y(Ky \eif x = y)) \eand \enot Dax)$'. We might call this \emph{inner} negation, since the negation occurs within the scope of the definite description. Note that its truth requires that there is a present King of France, albeit one who is not dating Alex.

\section{The adequacy of Russell's analysis}
How good is Russell's analysis of definite descriptions? This question has generated a substantial philosophical literature, but we will restrict ourselves to two observations.

One worry focusses on Russell's treatment of empty definite descriptions. If there are no Fs, then on Russell's analysis, both `the F is G' is and  `the F is non-G' are false. P.F.\ Strawson suggested that such sentences should not be regarded as false, exactly.\footnote{P.F.\ Strawson, `On Referring', 1950, \emph{Mind 59}, pp.\ 320--34.} Rather, they involve presupposition failure, and need to be regarded as \emph{neither} true \emph{nor} false. 

If we agree with Strawson here, we will need to revise our logic. For, in our logic, there are only two truth values (True and False), and every sentence is assigned exactly one of these truth values. 

But there is room to disagree with Strawson. Strawson is appealing to some linguistic intuitions, but it is not clear that they are very robust. For example: isn't it just \emph{false}, not `gappy', that Tim is dating the present King of France?

Keith Donnellan raised a second sort of worry, which (very roughly) can be brought out by thinking about a case of mistaken identity.\footnote{Keith Donnellan, `Reference and Definite Descriptions', 1966, \emph{Philosophical Review 77}, pp.\ 281--304.} Two men stand in the corner: a very tall man drinking what looks like a gin martini; and a very short man drinking what looks like a pint of water. Seeing them, Malika says:
	\begin{earg}
		\item[\ex{gindrinker}] The gin-drinker is very tall!
	\end{earg}
Russell's analysis will have us render Malika's sentence as:
	\begin{earg}
		\item[\ref{gindrinker}$'$.] There is exactly one gin-drinker [in the corner], and whoever is a gin-drinker [in the corner] is very tall.
	\end{earg}
Now suppose that the very tall man is actually drinking \emph{water} from a martini glass; whereas the very short man is drinking a pint of (neat) gin. By Russell's analysis, Malika has said something false, but don't we want to say that Malika has said something \emph{true}? 

Again, one might wonder how clear our intuitions are on this case. We can all agree that Malika intended to pick out a particular man, and say something true of him (that he was tall). On Russell's analysis, she actually picked out a different man (the short one), and consequently said something false of him. But  maybe advocates of Russell's analysis only need to explain \emph{why} Malika's intentions were frustrated, and so why she said something false. This is easy enough to do:  Malika said something false because she had false beliefs about the men's drinks; if Malika's beliefs about the drinks had been true,  then she would have said something true.\footnote{Interested parties should read Saul Kripke, `Speaker Reference and Semantic Reference', 1977, in French et al (eds.), \emph{Contemporary Perspectives in the Philosophy of Language}, Minneapolis: University of Minnesota Press, pp.\ 6-27.}

To say much more here would lead us into deep philosophical waters. That would be no bad thing, but for now it would distract us from the immediate purpose of learning formal logic. So, for now, we will stick with Russell's analysis of definite descriptions, when it comes to putting things into FOL. It is certainly the best that we can offer, without significantly revising our logic, and it is quite defensible as an analysis. 

\practiceproblems

\problempart
Using the following symbolisation key:
\begin{ekey}
\item[\text{domain}] people
\item[Kx] \gap{x} knows the combination to the safe.
\item[Sx] \gap{x} is a spy.
\item[Vx] \gap{x} is a vegetarian.
\item[Txy] \gap{x} trusts \gap{y}.
\item[h] Hofthor
\item[i] Ingmar
\end{ekey}
symbolise the following sentences in FOL:
\begin{earg}
\item Hofthor trusts a vegetarian.
\item Everyone who trusts Ingmar trusts a vegetarian.
\item Everyone who trusts Ingmar trusts someone who trusts a vegetarian.
\item Only Ingmar knows the combination to the safe.
\item Ingmar trusts Hofthor, but no one else.
\item The person who knows the combination to the safe is a vegetarian.
\item The person who knows the combination to the safe is not a spy.
\end{earg}


\solutions
\problempart
\label{pr.FOLcards}
Using the following symbolisation key:
\begin{ekey}
\item[\text{domain}] cards in a standard deck
\item[Bx] \gap{x} is black.
\item[Cx] \gap{x} is a club.
\item[Dx] \gap{x} is a deuce.
\item[Jx] \gap{x} is a jack.
\item[Mx] \gap{x} is a man with an axe.
\item[Ox] \gap{x} is one-eyed.
\item[Wx] \gap{x} is wild.
\end{ekey}
symbolise each sentence in FOL:
\begin{earg}
\item All clubs are black cards.
\item There are no wild cards.
\item There are at least two clubs.
\item There is more than one one-eyed jack.
\item There are at most two one-eyed jacks.
\item There are two black jacks.
\item There are four deuces.
\item The deuce of clubs is a black card.
\item One-eyed jacks and the man with the axe are wild.
\item If the deuce of clubs is wild, then there is exactly one wild card.
\item The man with the axe is not a jack.
\item The deuce of clubs is not the man with the axe.
\end{earg}

\

\problempart Using the following symbolisation key:
\begin{ekey}
\item[\text{domain}] animals in the world
\item[Bx] \gap{x} is in Farmer Brown's field.
\item[Hx] \gap{x} is a horse.
\item[Px] \gap{x} is a Pegasus.
\item[Wx] \gap{x} has wings.
\end{ekey}
symbolise the following sentences in FOL:
\begin{earg}
\item There are at least three horses in the world.
\item There are at least three animals in the world.
\item There is more than one horse in Farmer Brown's field.
\item There are three horses in Farmer Brown's field.
\item There is a single winged creature in Farmer Brown's field; any other creatures in the field must be wingless.
\item The Pegasus is a winged horse.
\item The animal in Farmer Brown's field is not a horse.
\item The horse in Farmer Brown's field does not have wings.
\end{earg}

\problempart
In this chapter, we symbolised `Nick is the traitor' by `$\exists x (Tx \eand \forall y(Ty \eif x = y) \eand x = n)$'. Two equally good symbolisations would be:
	\begin{ebullet}
		\item $Tn \eand \forall y(Ty \eif n = y)$
		\item $\forall y(Ty \eiff y = n)$
	\end{ebullet}
Explain why these would be equally good symbolisations.


\chapter{Sentences of FOL}\label{s:FOLSentences}
We know how to represent English sentences in FOL. The time has finally come to define the notion of a \emph{sentence} of FOL.

\section{Expressions}
There are six kinds of symbols in FOL:

\begin{description}
\item[Predicates] $A,B,C,\ldots,Z$, or 
with subscripts, as needed: $A_1, B_1,Z_1,A_2,A_{25},J_{375},\ldots$
\item[Names] $a,b,c,\ldots, r$, or
with subscripts, as needed $a_1, b_{224}, h_7, m_{32},\ldots$
\item[Variables] $s, t, u, v, w, x,y,z$, or
with subscripts, as needed $x_1, y_1, z_1, x_2,\ldots$
\item[Connectives]  $\enot,\eand,\eor,\eif,\eiff$
\item[Brackets] ( , )
\item[Quantifiers]  $\forall, \exists$
\end{description}
We define an \define{expression of FOL} as any string of symbols of FOL. Take any of the symbols of FOL and write them down, in any order, and you have an expression.

\section{Terms and formulas}
\label{s:TermsFormulas}

In \S\ref{s:TFLSentences}, we went straight from the statement of the vocabulary of TFL to the definition of a sentence of TFL. In FOL, we will have to go via an intermediary stage: via the notion of a \define{formula}. The intuitive idea is that a formula is any sentence, or anything which can be turned into a sentence by adding quantifiers out front. But this will take some unpacking.

We start by defining the notion of a term.
	\factoidbox{
		A \define{term} is any name or any variable. }
So, here are some terms:
	$$a, b, x, x_1 x_2, y, y_{254}, z$$
Next we need to define atomic formulas.
	\factoidbox{
		\begin{enumerate}
		\item If $\meta{R}$ is an $n$-place predicate and $\meta{t}_1, \meta{t}_2, \ldots, \meta{t_n}$ are terms, then $\meta{R t}_1 \meta{t}_2 \ldots \meta{t}_n$ is an atomic formula.
		\item If $\meta{t}_1$ and $\meta{t}_2$ are terms, then $\meta{t}_1 = \meta{t}_2$ is an atomic formula.
		\item Nothing else is an atomic formula.
		\end{enumerate}
	}

\newglossaryentry{term}{
  name = term,
  description = {either a \gls{name} or a \gls{variable}}
}

\newglossaryentry{formula}{
  name = formula,
  description = {an expression of FOL built according to the recursive rules in \S\ref{s:TermsFormulas}}
}
        
        
The use of script letters here follows the conventions laid down in \S\ref{s:UseMention}. So, `$\meta{R}$' is not itself a predicate of FOL. Rather, it is a symbol of our metalanguage (augmented English) that we use to talk about any predicate of FOL. Similarly, `$\meta{t}_1$' is not a term of FOL, but a symbol of the metalanguage that we can use to talk about any term of FOL. So, where `$F$' is a one-place predicate, `$G$' is a three-place predicate, and `$S$' is a six-place predicate, here are some atomic formulas:
	\begin{center}
		$x = a$\\
		$a = b$\\
		$Fx$\\
		$Fa$\\
		$Gxay$\\
		$Gaaa$\\
		$Sx_1 x_2 a b y x_1$\\
		$Sby_{254} z a a z$
	\end{center}
Once we know what atomic formulas are, we can offer recursion clauses to define arbitrary formulas. The first few clauses are exactly the same as for TFL.
	\factoidbox{
	\begin{enumerate}
		\item Every atomic formula is a formula. 
		\item If \meta{A} is a formula, then $\enot\meta{A}$ is a formula.
		\item If \meta{A} and \meta{B} are formulas, then $(\meta{A}\eand\meta{B})$ is a formula.
		\item If \meta{A} and \meta{B} are formulas, then $(\meta{A}\eor\meta{B})$ is a formula.
		\item If \meta{A} and \meta{B} are formulas, then $(\meta{A}\eif\meta{B})$ is a formula.
		\item If \meta{A} and \meta{B} are formulas, then $(\meta{A}\eiff\meta{B})$ is a formula.
		\item If \meta{A} is a formula, \meta{x} is a variable, \meta{A} contains at least one occurrence of \meta{x}, and \meta{A} contains neither $\forall \meta{x}$ nor $\exists \meta{x}$, then $\forall\meta{x}\meta{A}$ is a formula.
		\item If \meta{A} is a formula, \meta{x} is a variable, \meta{A} contains at least one occurrence of \meta{x}, and \meta{A} contains neither $\forall \meta{x}$ nor $\exists \meta{x}$, then $\exists\meta{x}\meta{A}$ is a formula.
		\item Nothing else is a formula.
	\end {enumerate}
	}
So, assuming again that `$F$' is a one-place predicate, `$G$' is a three-place predicate and `$H$' is a six place-predicate, here are some formulas:
	\begin{center}
		$Fx$\\
		$Gayz$\\
		$Syzyayx$\\
		$(Gayz \eif Syzyayx)$\\
		$\forall z (Gayz \eif Syzyayx)$\\
		$Fx \eiff \forall z (Gayz \eif Syzyayx)$\\
		$\exists y (Fx \eiff \forall z (Gayz \eif Syzyayx))$\\
		$\forall x \exists y (Fx \eiff \forall z (Gayz \eif Syzyayx))$		\end{center}
However, this is \emph{not} a formula:
	\begin{center}
		$\forall x \exists x Gxxx$
	\end{center}
Certainly `$Gxxx$' is a formula, and `$\exists x Gxxx$' is therefore also a formula. But we cannot form a new formula by putting `$\forall x$' at the front. This violates the constraints on clause 7 of our recursive definition: `$\exists x Gxxx$' contains at least one occurrence of `$x$', but it already contains `$\exists x$'.

These constraints have the effect of ensuring that variables only serve one master at any one time (see \S\ref{s:MultipleGenerality}). In fact, we can now give a formal definition of scope, which incorporates the definition of the scope of a quantifier. Here we follow the case of TFL, though we note that a logical operator can be either a connective or a quantifier:
	\factoidbox{
		The \define{main logical operator} in a formula is the operator that was introduced last, when that formula was constructed using the recursion rules.
		
\bigskip

		The \define{scope} of a logical operator in a formula is the subformula for which that operator is the main logical operator.
	}
So we can graphically illustrate the scope of the quantifiers in the preceding example thus:
	$$\overbrace{\forall x \overbrace{\exists y (Fx \eiff \overbrace{\forall z (Gayz \eif Syzyayx)}^{\text{scope of `}\forall z\text{'}}}^{\text{scope of `}\exists y\text{'}})}^{\text{scope of `$\forall x$'}}$$

\newglossaryentry{main logical operator}{
  name = main logical operator,
  description = {the operator used last in the construction of a \gls{formula}}
}

\newglossaryentry{scope}{
  name = scope,
  description = {the subformula of a \gls{formula} of FOL for which the \gls{main logical operator} is the operator}
}


\section{Sentences}
Recall that we are largely concerned in logic with assertoric sentences: sentences that can be either true or false. Many formulas are not sentences. Consider the following symbolisation key:
	\begin{ekey}
		\item[\text{domain}] people
		\item[Lxy] \gap{x} loves \gap{y}
		\item[b] Boris
	\end{ekey}
Consider the atomic formula `$Lzz$'. All atomic formula are formulas, so `$Lzz$' is a formula, but can it be true or false? You might think that it will be true just in case the person named by `$z$' loves themself, in the same way that `$Lbb$' is true just in case Boris (the person named by `$b$') loves himself. \emph{However, `$z$' is a variable, and does not name anyone or any thing.}

Of course, if we put an existential quantifier out front, obtaining `$\exists zLzz$', then this would be true iff someone loves herself. Equally, if we wrote `$\forall z Lzz$', this would be true iff everyone loves themself. The point is that we need a quantifier to tell us how to deal with a variable. 

Let's make this idea precise.
	\factoidbox{
		A \define{bound variable} is an occurrence of a variable \meta{x} that is within the scope of either $\forall\meta{x}$ or $\exists\meta{x}$. 
		
		\bigskip
		
		A \define{free variable} is any variable that is not bound.
	}

\newglossaryentry{bound variable}{
  name = bound variable,
  description = {an occurrence of a variable in a \gls{formula} which is in the scope of a quantifier followed by the same variable}
}

\newglossaryentry{free variable}{
  name = free variable,
  description = {an occurrence of a variable in a \gls{formula} which is not a \gls{bound variable}}
}

        
For example, consider the formula
	$$\forall x(Ex \eor Dy) \eif \exists z(Ex \eif Lzx)$$
The scope of the universal quantifier `$\forall x$' is `$\forall x (Ex \eor Dy)$', so the first `$x$' is bound by the universal quantifier. However, the second and third occurrence of `$x$' are free. Equally, the `$y$' is free. The scope of the existential quantifier `$\exists z$' is `$(Ex \eif Lzx)$', so `$z$' is bound. 

Finally we can say the following.	
	\factoidbox{	
		A \define{sentence} of FOL is any formula of FOL that contains no free variables.
	}

\newglossaryentry{sentence of FOL}{
  name = sentence of FOL,
  description = {a \gls{formula} of FOL which has no \glspl{bound variable}}
}



\section{Bracketing conventions}

We will adopt the same notational conventions governing brackets that we did for TFL (see \S\ref{s:TFLSentences} and \S\ref{s:MoreBracketingConventions}.) 

First, we may omit the outermost brackets of a formula. 

Second, we may use square brackets, `[' and `]', in place of brackets to increase the readability of formulas. 

%Third, we may omit brackets between each pair of conjuncts when writing long series of conjunctions. 

%Fourth, we may omit brackets between each pair of disjuncts when writing long series of disjunctions.

\practiceproblems
\problempart
\label{pr.freeFOL}
Identify which variables are bound and which are free.
\begin{earg}
\item $\exists x Lxy \eand \forall y Lyx$
\item $\forall x Ax \eand Bx$
\item $\forall x (Ax \eand Bx) \eand \forall y(Cx \eand Dy)$
\item $\forall x\exists y[Rxy \eif (Jz \eand Kx)] \eor Ryx$
\item $\forall x_1(Mx_2 \eiff Lx_2x_1) \eand \exists x_2 Lx_3x_2$
\end{earg}
